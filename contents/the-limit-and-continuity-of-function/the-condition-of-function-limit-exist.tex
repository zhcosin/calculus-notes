
\section{函数极限存在的条件}
\label{sec:the-condition-of-function-limit-exist}

\begin{theorem}[函数极限与数列极限的关系]
  在$x_0$的某空心邻域内有定义的函数$f(x)$,当$x \to x_0$时存在极限的充分必要条件是,对于任意一个在这空心邻域内取值并以$x_0$为极限的数列$x_n$,$\lim_{n \to \infty}f(x_n)$都存在而且都相等。
\end{theorem}

\begin{proof}[证明]
  先证必要性,如果$\lim_{x \to x_0} = A$,那么对于任意小的正实数$\varepsilon > 0$,都存在另一个正实数$\delta > 0$,使得对这邻域内满足$|x-x_0|<\delta$的实数$x$都成立不等式$|f(x)-A|<\varepsilon$,那么对于任意一个也在这空心邻域内取值并以$x_0$为极限的数列$x_n$,因为它以$x_0$为极限,所以对于这个$\delta>0$,就必然能够从某一项$x_N$开始,后面的所有项都满足$|x_n-x_0|<\delta$,于是就有$|f(x_n)-A|<\varepsilon$,这就表明$f(x_n)$当$n \to \infty$时以$A$为极限,必要性得证。

  再证充分性,如果对于任意一个在这空心邻域内取值并收敛到$x_0$的数列$x_n$,对应的函数值数列$f(x_n)$都收敛到同一实数$A$,我们将证明,函数$f(x)$在$x \to x_0$时也必将收敛到$A$. 采用反证法,假使函数$f(x)$当$x \to x_0$时不以数$A$为极限,那么必然存在某个$\varepsilon_0>0$,使得无论把另一个正实数$\delta>0$限制得多么小,总有满足$|x-x_0|<\delta$的实数$x$能够使得$|f(x)-A| \geqslant \varepsilon_0$成立,于是先取$\delta=1$,得出一个符合这条件的实数$x_1$,然而取$\delta=\min\{\frac{1}{2}, |x_1-x_0|\}>0$,又可以选出$x_2$,依次这样下去,逐个令$\delta_n=\min\{\frac{1}{n}, |x_{n-1}-x_0|\}$,就可以挑选出$x_{n+1}$,这样就作出一个数列$x_n$,由$|x_n-x_0|<\delta_n<\frac{1}{n}$可知$x_n$收敛到$x_0$,但是由于$|f(x_n)-A| \geqslant \varepsilon$恒成立,可知数列$f(x_n)$并不收敛到$A$,这样,我们就证明了如果函数$f(x)$当$x \to x_0$时不以$A$为极限,那么就可以构造出一个以$x_0$为极限的数列$x_n$,使得$f(x_n)$也不以$A$为极限,这与我们的条件是矛盾的,所以充分性得证。
\end{proof}

事实上,如果任意以$x_0$为极限的数列$x_n$,函数值数列$f(x_n)$都收敛的话,这些极限值也必然相同,这是因为,如若不然,假如两个数列$x_n$和$r_n$分别以$A$和$B$为极限,那么在这两个数列中交错的取项构成另一数列$s_n$,显然$s_n$也以$x_0$为极限,而函数值数列$f(s_n)$中的奇数下标子列和偶数下标子列分别以$A$和$B$为极限,由条件知$f(s_n)$应有极限,所以$A=B$.有了这结论,上述定理中的条件可以适当减弱。

与数列的单调有界定理相仿,我们有以下定理
\begin{theorem}
  如果函数$f(x)$在$x_0$的某左空心邻域内单调递增且有上界,则函数$f(x)$在$x_0$处的左极限存在,右极限也有类似的结论。
\end{theorem}

\begin{proof}[证明]
证明很简单,只要在这左邻域内任取一单调增加并以$x_0$为极限的数列$x_n$,则函数值数列$f(x_n)$亦必是单调增加的数列,而它又有上界,所以它有极限,设这极限为$A$,则易证$A$便是函数在这左空心邻域内的上确界,那么对于无论多么小的正实数$\varepsilon$,都存在正整数$N$,使得当$n>N$时恒有$|f(x_n)-A|<\varepsilon$成立,于是取$\delta = x_0-x_{N+1}>0$,则对于任意满足$x_0-\delta<x<x_0$的实数$x$,有$x_N+1<x<x_0$,因而$A-\varepsilon<f(x_{N+1})<f(x) \leqslant A$,这表明$A$就是$f(x)$在$x_0$处的左极限。
\end{proof}

仿照数列的柯西收敛准则,有
\begin{theorem}
  函数$f(x)$在$x_0$的某空心邻域内有定义,则它在该存在极限的充分必要条件是: 任给无论多么小的正实数$\varepsilon$,恒存在另一正实数$\delta$,使得对任意满足$|x-x_0|<\delta$的两个实数$x_1$和$x_2$都成立$|f(x_1)-f(x_2)|<\varepsilon$.
\end{theorem}

\begin{proof}[证明]
  必要性是容易证明的,略去,下证充分性,取正实数数列$\varepsilon_n=1/n(n=1,2,\ldots)$,存在另一单调递减的正实数数列$\delta_n$,使得对于任意满足$|x_1-x_0|<\delta_n$和$|x_2-x_0|<\delta_n$的$x_1,x_2$都成立$|f(x_1)-f(x_2)|<\varepsilon_n$,于是函数$f(x)$在$x_0$的以$\delta_n$为半径的空心邻域内的函数值都必将限于一个长度为$\varepsilon$的闭区间$[m_n,M_n]$上,显然这个闭区间序列符合闭区间套定理的条件,所以有唯一实数$A$从属于所有的闭区间,显然这实数$A$也是函数$f(x)$在$x_0$处的极限。
\end{proof}

%%% Local Variables:
%%% mode: latex
%%% TeX-master: "../../calculus-note"
%%% End:
