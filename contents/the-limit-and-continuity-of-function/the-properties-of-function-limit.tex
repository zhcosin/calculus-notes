
\section{函数极限的性质}
\label{sec:the-properties-of-function-limit}

与数列极限的性质相仿,函数极限具有类似的性质,以下定理都以$x\to x_0$为例,但它们对于自变量趋于无穷大时的极限也是成立的。

\begin{property}[唯一性]
  函数极限$\lim_{x \to x_0}f(x)$若存在必唯一.
\end{property}

\begin{theorem}[局部有界性]
  设函数$f(x)$在$x_0$的某空心邻域内有定义,若$\lim_{x \to x_0}f(x)$存在(非无穷的有限值),则$f(x)$在$x_0$的某个空心邻域内有界。
\end{theorem}

\begin{theorem}[局部保号性]
  若函数$f(x)$在$x \to x_0$处的极限存在为$A$,则对于任意$r<A$都存在$x_0$的某个空心邻域内,在这邻域内恒有$f(x)>r$,同样,对于任意$r>A$,都存在$x_0$的某空心邻域,在这邻域内恒有$f(x)<r$.
\end{theorem}

\begin{theorem}[保不等式性]
 如果函数$f(x)$和$g(x)$都在$x_0$的某个空心邻内有定义,且在这邻域内上恒满足$f(x) \geqslant g(x)$,那么如果当$x \to x_0$时两个函数分别有极限$A$和$B$,则必有$A \geqslant B$.
\end{theorem}

\begin{theorem}[夹逼定理]
  在$x_0$的某空心邻域内有定义的三个函数$f(x)$、$g(x)$和$h(x)$,如果在这邻域内恒满足$f(x) \leqslant h(x) \leqslant g(x)$,那么如果当$x \to x_0$时$f(x)$和$g(x)$都收敛到$A$,那么这时$h(x)$也必收敛,且也收敛到$A$.
\end{theorem}

\begin{theorem}[四则运算法则]
  如果在$x_0$的某空心邻域内有定义的二函数$f(x)$和$g(x)$在$x \to x_0$时分别收敛到$A$和$B$,那么这由两个函数的和、差、积、商作成的新函数也收敛,并分别收敛到原先两个极限值和和、差、积、商,在商的情况下,要求分母不为零.
\end{theorem}

写成公式就是,在$\lim_{x \to x_0} f(x)$和$\lim_{x \to x_0}g(x)$都存在的前提下,有
\begin{eqnarray*}
  \lim_{x \to x_0} (f(x) \pm g(x)) & = & \lim_{x \to x_0} f(x) \pm \lim_{x \to x_0} g(x)  \\
  \lim_{x \to x_0} f(x)g(x) & = & \lim_{x \to x_0} f(x) \cdot \lim_{x \to x_0} g(x)  \\
  \lim_{x \to x_0} \frac{f(x)}{g(x)} & = & \frac{\lim_{x \to x_0} f(x)}{\lim_{x \to x_0} g(x)}  
\end{eqnarray*}

关于复合函数的极限,有如下结论
\begin{theorem}
  \label{theorem:limit-of-combine-function}
  设有如下函数极限
  \[ \lim_{x \to x_0}g(x) = u_0, \  \lim_{u \to u_0}f(u) = y_0 \]
  则有
  \[ \lim_{x \to x_0} f(g(x))=y_0 \]
\end{theorem}

\begin{proof}[证明]
  因为当$u \to u_0$时,$f(u) \to y_0$,所以对于无论多么小的正实数$\varepsilon$,总存在另一正实数$\delta$,使得当$|u-u_0|<\delta$时恒有$|f(u)-y_0|<\varepsilon$,而又由于当$x \to x_0$时$g(x) \to u_0$,所以对于前面提到的$\delta$,存在另一正实数$r$,使得当$|x-x_0|<r$时恒有$|g(x)-u_0|<\delta$,从而有$|f(g(x))-y_0|<\varepsilon$,所以最终得到的结论就是:对于无论多么小的正实数$\varepsilon$,总存在另一正实数$r$,使得当$|x-x_0|<r$时,恒有$|f(g(x))-y_0|<\varepsilon$,这就证得结论。
\end{proof}

%%% Local Variables:
%%% mode: latex
%%% TeX-master: "../../calculus-note"
%%% End:
