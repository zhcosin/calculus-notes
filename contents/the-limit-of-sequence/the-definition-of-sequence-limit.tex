
\section{数列极限的概念}
\label{sec:the-definition-of-sequence-limit}

\subsection{极限的定义}
\label{sec:definition-number-sequence-limit}

在中学数学中,我们熟知反比例函数$f(x)=1/x$的图象在向无穷远处延伸时,它会无限的向横轴靠近,当$x$取正值并无限的增大时,其第一象限的一支会无限的向$x$轴正半轴靠近,但无论$x$取多大,因为$1/x>0$,所以它始终不会与$x$轴相交,这给了我们一种“无限接近但是不会相等”的直观感受。

同样的情况还有许多,我们就准备来详细的讨论下这种“无限接近又不相等”的现象。

上面反比例函数的例子是针对函数而言的,我们先从较为简单的数列开始,同样可以得到数列$x_n=1/n$这个数列,在$n$取正整数并无限增大时,数列的值无限的接近零,但却总是大于零,我们来从这种现象中提取 \emph{极限} 的概念。

首先要指出的是这里“无限接近但不等于”中的“不等于”其实是无关紧要的,例如在数列$1/n$中把下标为偶数的项全部换成零,那么这个"无限"接近并没有被破坏,而且它仍然给我们以极限的印象,只是它在下标增大的过程中,可以无限次的取极限值,而且从这个例子中还可得知,数列的单调性也不是必要的。

我们先给出一个初步的定义: 如果数列$x_n$在随着$n$的无限增大过程中可以无限的接近一个常数$A$,则称$A$是这数列当$n$趋于无穷大时的极限。

这个定义不会令人满意,因为作为数学上的一个定义,它需要具备精确性,而这个定义中出现了“无限接近”这样含义模糊不清的描述,利用这个定义,我们很难说明一个给定常数是否是一个数列的极限,我们需要将它严格化。

所谓数列$x_n$“无限接近”于常数$A$,自然指的是差值$|x_n-A|$可以任意的小,所以我们进行第一步严格化:把数列$x_n$无限接近常数$A$严格化成差值$|x_n-A|$可以任意小,于是极限的定义可以重新叙述为: 对于数列$x_n$和常数$A$,如果数列$x_n$当$n$无限增大时差值$|x_n-A|$可以任意的小,则称常数$A$是数列$x_n$当$n$趋向于无穷大时的极限。

然后我们考虑如何刻画“可以任意的小”,那就是说,差值$|x_n-A|$可以小于任意的正实数$\varepsilon$,而不管这正实数$\varepsilon$有多小。初看起来,“可以小于任意的正实数”,似乎只要存在正整数$n$,使得$|x_n-A|<\varepsilon$就可以了,也就是如下的极限定义: 对于数列$x_n$和常数$A$,如果对于无论多么小的正实数$\varepsilon$,总存在正整数$N$,使得$|x_N-A|<\varepsilon$成立,则称常数$A$是数列$x_n$在下标趋于无穷大时的极限。

这个定义看上去似乎非常符合$1/n$这个数列的特征,不管多么小的正实数$\varepsilon$,总能找到使$1/n < \varepsilon$成立的$n$,只要$n$取的足够大。然而这个定义却有一个严重的问题,我们把数列$1/n$中下标为偶数的项全部换成1,所得到的新数列显然不应该有极限,因为它的奇数下标项趋于零而偶数下标项恒为1,按我们的直观感受,它的值并不无限靠近零,也不无限靠近1,0和1都不应该是它的极限,但是按照上面的定义,它却是符合条件的!

问题出在哪呢,仍然以这个把偶数下标项都替换为1的新数列为例,显然,存在正整数$N$使得$|x_n-A|<\varepsilon$这个条件,是无法保证数列的全部项都向常数$A$靠近的,它只能保证数列中有一部分项会向常数$A$靠近,刚才这个例子也说明了这一点,所以我们需要一个更强的能保证数列的所有项都要向常数$A$靠近,我们把存在正整数$N$使得$|x_N-A|<\varepsilon$成立,改为存在正整数$N$,使得$n>N$时$|x_n-A|<\varepsilon$恒成立,这样一来这个新数列就不满足这条件了,而原来的数列$1/n$却满足这条件。

这个新的条件,利用$n>N$时$|x_n-A|<\varepsilon$恒成立,来保证了数列向$A$靠近的总体趋势。这就是我们最终的极限定义,这个定义,从模糊到精确,别看在这几段话就给出了,实际上在历史上经过了几代数学家的努力,最后才由德国数学家魏尔斯特拉斯(Weierstrass, 1815.10.31-1897.2.19)在总结前人成果的基础上给出,这个精确定义,是人类智慧的结晶。

\begin{definition}
  对于实数数列${a_n}$和实数$a$,如果对于任意小的正实数$\varepsilon$,都存在某一下标$N$,使得该数列在这之后的所有项(即$n>N$)都满足
  \begin{equation}
    \label{eq:the-definition-of-sequence-limit}
    |a_n-a|<\varepsilon
  \end{equation}
  则称该数列存在极限,实数$a$称为该数列的极限。也称该数列为收敛数列,并且收敛到实数$a$,记为
  \begin{equation}
    \label{eq:limit-definition-for-number-sequence}
    \lim_{n \to \infty}x_n = a
  \end{equation}
\end{definition}

极限为零的数列称为无穷小数列,简称\emph{无穷小}。如果数列不存在有限的极限,称为数列为\emph{发散数列}。

如果数列无论对于多大的实数$M>0$,总能从某项开始,后续的全部项都有$a_n>M$,则称数列为\emph{正无穷大}。类似的也有负无穷大和(绝对值)无穷大的概念。

\subsection{一些例子}
\label{sec:some-example-for-number-sequence}

\begin{example}
  设实数$a>1$,则
  \[ \lim_{n \to \infty} \frac{1}{a^n} = 0 \]

  对于无论多么小的正实数$\varepsilon$,为了找到极限定义中所要求的$N$,考虑不等式
  \[ \frac{1}{a^n} < \varepsilon \]
  也就是$a^n>1/\varepsilon$,设$a=1+\lambda$,则$\lambda>0$,按二项式定理有\footnote{我们这里并没有从$a^n>1/\varepsilon$中直接使用对数来得出$n>\log_a{(1/\varepsilon)}$,这是因为尽管中学数学中已经学过对数概念,但那时还没有给出无理指数幂的定义,所以指数的定义是不完整的,因此我们无法确认,对于底数$a$,正实数$1/\varepsilon$的对数是否存在,以后我们将在\autoref{sec:the-power-of-real-with-rational-exponent}中专门讨论指数的定义和值域问题。}
  \[ a^n = (1+\lambda)^n = 1 + n\lambda + \frac{n(n-1)}{2!}\lambda^2+\cdots+\lambda^n > 1+n \lambda \]
  所以只要$1+n\lambda>1/\varepsilon$,便能保证$a^n>1/\varepsilon$成立,也就是只需要$n > (1/\varepsilon-1) / (a-1)$就行了,所以只要选择$N>(1/\varepsilon-1)/(a-1)$就行了,这就证得了此极限。
\end{example}

\begin{example}
  \label{example:limit-of-n-sqrt-a-when-a-greater-than-1}
  设实数$a>1$且$a \neq 1$,则 $\lim_{n \to \infty} \sqrt[n]{a} = 1$

  \begin{proof}[证明一]
    利用乘法公式$x^n-1=(x-1)(x^{n-1}+x^{n-2}+\cdots+1)$可得
    \[ \sqrt[n]{a}-1 = \frac{a-1}{(\sqrt[n]{a})^{n-1}+(\sqrt[n]{a})^{n-2}+\cdots+1} < \frac{1}{n}(a-1) \]
   于是对于任意正实数$\varepsilon$,只要取$N>\frac{a-1}{\varepsilon}$便能保证$n>N$时有$0<\sqrt[n]{a}-1<\varepsilon$,所以这极限得证。
  \end{proof}

  \begin{proof}[证明二]
    设$z_n=\sqrt[n]{a}-1$,则
    \[ a = (1+z_n)^n = 1+ nz_n+\frac{1}{2!}z_n^2+\cdots+z_n^n > 1+ n z_n \]
    所以得到
    \[ 0<z_n<\frac{1}{n}(a-1) \]
    下同证明一.
  \end{proof}
\end{example}


%%% Local Variables:
%%% mode: latex
%%% TeX-master: "../../calculus-note"
%%% End:
