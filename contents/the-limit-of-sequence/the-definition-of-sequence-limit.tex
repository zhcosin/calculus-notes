
\section{数列极限的概念}
\label{sec:the-definition-of-sequence-limit}

\subsection{极限的定义}
\label{sec:definition-number-sequence-limit}

\begin{definition}
  对于实数数列${a_n}$,如果存在实数$a$,使得对于任意小的正实数$\varepsilon$,都存在某一下标$N$,使得该数列在这之后的所有项(即$n>N$)都满足
  \begin{equation}
    \label{eq:the-definition-of-sequence-limit}
    |a_n-a|<\varepsilon
  \end{equation}
  则称该数列存在极限,实数$a$称为该数列的极限。也称该数列为收敛数列,并且收敛到实数$a$,记为
  \begin{equation}
    \label{eq:limit-definition-for-number-sequence}
    \lim_{n \to \infty}x_n = a
  \end{equation}
\end{definition}

极限为零的数列称为无穷小数列,简称\emph{无穷小}。如果数列不存在有限的极限,称为数列为\emph{发散数列}。

如果数列无论对于多大的实数$M>0$,总能从某项开始,后续的全部项都有$a_n>M$,则称数列为\emph{正无穷大}。类似的也有负无穷大和(绝对值)无穷大的概念。

\subsection{一些例子}
\label{sec:some-example-for-number-sequence}

\begin{example}
  设实数$a>1$,则
  \[ \lim_{n \to \infty} \frac{1}{a^n} = 0 \]

  对于无论多么小的正实数$\varepsilon$,为了找到极限定义中所要求的$N$,考虑不等式
  \[ \frac{1}{a^n} < \varepsilon \]
  也就是$a^n>1/\varepsilon$,设$a=1+\lambda$,则$\lambda>0$,按二项式定理有\footnote{我们这里并没有从$a^n>1/\varepsilon$中直接使用对数来得出$n>\log_a{(1/\varepsilon)}$,这是因为尽管中学数学中已经学过对数概念,但那时还没有给出无理指数幂的定义,所以指数的定义是不完整的,因此我们无法确认,对于底数$a$,正实数$1/\varepsilon$的对数是否存在,以后我们将在\autoref{sec:the-power-of-real-with-rational-exponent}中专门讨论指数的定义和值域问题。}
  \[ a^n = (1+\lambda)^n = 1 + n\lambda + \frac{n(n-1)}{2!}\lambda^2+\cdots+\lambda^n > 1+n \lambda \]
  所以只要$1+n\lambda>1/\varepsilon$,便能保证$a^n>1/\varepsilon$成立,也就是只需要$n > (1/\varepsilon-1) / (a-1)$就行了,所以只要选择$N>(1/\varepsilon-1)/(a-1)$就行了,这就证得了此极限。
\end{example}

\begin{example}
  \label{example:limit-of-n-sqrt-a-when-a-greater-than-1}
  设实数$a>1$且$a \neq 1$,则 $\lim_{n \to \infty} \sqrt[n]{a} = 1$

  \begin{proof}[证明一]
    利用乘法公式$x^n-1=(x-1)(x^{n-1}+x^{n-2}+\cdots+1)$可得
    \[ \sqrt[n]{a}-1 = \frac{a-1}{(\sqrt[n]{a})^{n-1}+(\sqrt[n]{a})^{n-2}+\cdots+1} < \frac{1}{n}(a-1) \]
   于是对于任意正实数$\varepsilon$,只要取$N>\frac{a-1}{\varepsilon}$便能保证$n>N$时有$0<\sqrt[n]{a}-1<\varepsilon$,所以这极限得证。
  \end{proof}

  \begin{proof}[证明二]
    设$z_n=\sqrt[n]{a}-1$,则
    \[ a = (1+z_n)^n = 1+ nz_n+\frac{1}{2!}z_n^2+\cdots+z_n^n > 1+ n z_n \]
    所以得到
    \[ 0<z_n<\frac{1}{n}(a-1) \]
    下同证明一.
  \end{proof}
\end{example}


%%% Local Variables:
%%% mode: latex
%%% TeX-master: "../../book"
%%% End:
