
\section{数列极限存在的条件}
\label{sec:condition-of-limit-exists}

极限的定义中用到了极限值,所以它只能用于判断一个指定常数是否某个数列的极限,而无法针对一个数列回答它是否收敛,所以我们还需要开发一些判定条件,从数列本身的性质来推断它是否收敛。

\subsection{有界数列与确界定理}
\label{sec:bound-number-sequence}

对于一个实数集,如果存在实数$M$,使得集合中的全部数$x$都满足$x \leqslant M$,则称实数$M$是这数集的一个\emph{上界},如果不等式是反向的,则称这实数是这数集的一个\emph{下界},显然,如果$M$是某个数集的上界,则比$M$大的所有实数也都是这数集的上界,对下界亦有类似结论。

如果数集既有上界又有下界,则称数集\emph{有界},有界数集的所有项的数值能够被某个区间所全部包含。

有界的另一种表述是,存在正实数$M>0$,使得数集的全部数$x$都满足$|x| \leqslant M$,这与前述说法是等价的。

\begin{theorem}[收敛数列的有界性]
  收敛数列必有界。
\end{theorem}

\begin{proof}[证明]
  这其实从定义就可以得出了,随便取一个$\epsilon>0$,即知数列从某项起全部落在区间$(a-\epsilon, a+\epsilon)$内,这里$a$是数列极限,再扩大此区间把前面的那些项(有限个)包含进来,于是数列便有界。
\end{proof}

\begin{definition}
对于一个有上界的实数集,如果某个实数$M$满足: (1)它是这数集的上界. (2)对于无论多么小的正实数$\epsilon$,总存在数集中的数$x$使得$x>M-\epsilon$,则称实数$M$是这数集的\emph{上确界},类似的有\emph{下确界}的定义.
\end{definition}

显然,上确界是最小的上界,下确界是最大的下界。

\begin{theorem}[确界定理]
  如果数集有上界,则必有上确界,如果有下界,则必有下确界。
\end{theorem}

确界定理的证明依赖于实数的完备性,这放在后续章节来完成。

\subsection{单调有界定理}
\label{sec:monotone-bound-theorem}

\begin{theorem}
  单调递增有上界的数列必定收敛,而且收敛到它的上确界。单调递减有下界的数列也类似。
\end{theorem}

\begin{proof}[证明]
  只证明单调递增有上界的情况,假如数列$x_n$就是这样的数列,它的上确界是$M$,则数列中的全部项都满足$x_n \leqslant M$,另外,对于任意小的正实数$\epsilon$,由上确界定义,总存在某个$x_N$满足$x_N>M-\epsilon$,再由单调性即知对于$n>N$恒有$M-\epsilon < x_n \leqslant M < M+\epsilon$,所以$M$就是这数列的极限。
\end{proof}

\subsection{一个重要的数列极限}
\label{sec:a-import-sequence-limit}

这一小节我们来证明下面这个数列有极限:
\[ x_n=\left( 1+\frac{1}{n} \right)^n \]

\begin{proof}[证明一]\footnote{这个证明来自参考文献\cite{olympic-math}.}
  由多元均值不等式,把$x_n$看成$n$个$(1+1/n)$的乘积,再添加上一个因数1构成$n+1$个数的乘积,有
  \[ \left( 1+\frac{1}{n} \right)^n = 1 \cdot \left( 1+\frac{1}{n} \right)^n < \left( \frac{1+n\left( 1+\frac{1}{n} \right)}{n+1} \right)^{n+1} = \left( 1+\frac{1}{n+1} \right)^{n+1} \]
  这便表明它是递增的。

  下证它是有上界的,把$n+1$拆分成$\frac{5}{6}$和$n$个$1+\frac{1}{6n}$,由均值不等式得
  \[ n+1 = \frac{5}{6} + n \left( 1+\frac{1}{6n} \right) > (n+1)\sqrt[n+1]{\frac{5}{6} \cdot \left( 1+\frac{1}{6n} \right)^n} \]
  整理即得
  \[ \left( 1+\frac{1}{6n} \right)^n < \frac{5}{6} \]
  所以
  \[ \left( 1+\frac{1}{6n} \right)^{6n} < \left( \frac{5}{6} \right)^6 < 3 \]
  于是由单调性便知
  \[ \left( 1+\frac{1}{n} \right)^n < 3 \]
  所以数列$x_n$单调增加且有上界,故此存在极限。
\end{proof}

\begin{proof}[证明二]\footnote{这个证明来自于参考文献\cite{math-analysis}.}
  把$x_n$按二项式定理展开得
  \begin{eqnarray*}
    x_n & = & \sum_{i=0}^n C_n^i \frac{1}{n^i} \\
    & = & \sum_{i=0}^n \frac{1}{i!}\left( 1-\frac{1}{n} \right) \left( 1-\frac{2}{n} \right)\cdots \left( 1-\frac{i-1}{n} \right)
  \end{eqnarray*}
  易见对于$x_{n+1}$而言,在上式的基础上会多出$i=n+1$的一个正项,并且其它项是把上式中每一个项中的每一个因子$1-\frac{i}{n}$更换为更大的因子$1-\frac{i}{n+1}$,所以$x_{n+1}>x_n$,这是一个递增的数列.

  将每一个项中的所有$(1-i/n)$因子全部放大为1,则有
  \[  x_n < 1+\frac{1}{1!}+\frac{1}{2!}+\cdots+\frac{1}{n!} \]
  接下来有两种放缩方式都可以证明它有上限:
  \[ \frac{1}{k!} < \frac{1}{k(k-1)} = \frac{1}{k-1} - \frac{1}{k} \]
  和
  \[ \frac{1}{k!} < \frac{1}{2^{k-1}} \]
  于是
  \[ x_n < 2 + \sum_{i=2}^n \left( \frac{1}{i-1}-\frac{1}{i} \right) = 3-\frac{1}{n} < 3 \]
  或者
  \[ x_n < 2 + \sum_{i=2}^n \frac{1}{2^{i-1}} = 3-\frac{1}{2^{n-1}} < 3 \]
  所以数列单调递增有上界,故此有极限.
\end{proof}


\subsection{柯西收敛准则}
\label{sec:cauchy-converage-principle}

\begin{theorem}[柯西收敛准则]
  数列$x_n$收敛的充分必要条件是,对于任意正实数$\epsilon$,总存在正整数$N>0$,使得任意$n_1>N$和任意$n_2>N$及任意恒有$|x_{n_1}-x_{n_2}| < \epsilon$。
\end{theorem}

\begin{proof}[证明]
  只证明必要性,充分性的证明放在实数完备性那一节。

  如果数列$x_n$收敛到$x$,那么对于任意正实数$\epsilon$,都有正整数$N$,使得$n>N$时恒有$|x_n-x|<\epsilon / 2$,于是对于任意$n_1>N$及$n_2>N$,便有$|x_{n_1}-x_{n_2}|=|(x_{n_1}-x)- (x_{n_2}-x)|\leqslant |x_{n_1}-x|+|x_{n_2}-x|<\epsilon / 2+\epsilon / 2 = \epsilon$。必要性得证。
\end{proof}

\begin{example}
  前$n$个正整数的平方倒数和
  \[ S_n = 1 + \frac{1}{2^2} + \cdots + \frac{1}{n^2} \]
  对它的片段有
  \begin{eqnarray*}
    S_{m+p}-S_m  & = & \frac{1}{(m+1)^2} + \frac{1}{(m+2)^2} + \cdots + \frac{1}{(m+p)^2} \\
                 & < & \frac{1}{m(m+1)} + \frac{1}{(m+1)(m+2)} + \cdots + \frac{1}{(m+p-1)(m+p)} \\
                 & = & \left( \frac{1}{m} - \frac{1}{m+1} \right) + \left( \frac{1}{m+1} - \frac{1}{m+2} \right) + \cdots + \left( \frac{1}{m+p-1} - \frac{1}{m+p} \right) \\
    & = & \frac{1}{m} - \frac{1}{m+p} < \frac{1}{m}
  \end{eqnarray*}
  所以对于任意正实数$\varepsilon$,只要取$N>1/\varepsilon$,就能保证柯西条件成立,于是数列$S_n$有极限,不过这极限值在此处是求不出来的,在以后我们将会利用无穷级数理论,得到它的极限值,这极限值与圆周率有关:
  \[ \lim_{n \to \infty} \sum_{i=1}^n \frac{1}{i^2} = \frac{\pi^2}{6} \]
\end{example}

\begin{example}
  前$n$个正整数的阶乘的倒数和
  \[ T_n = 1 + \frac{1}{2!} + \cdots + \frac{1}{n!} \]
  它的片段和
  \begin{eqnarray*}
    T_{m+p} - T_m & = & \frac{1}{(m+1)!} + \frac{1}{(m+2)!} + \cdots + \frac{1}{(m+p)!} \\
                  & < & \frac{1}{2^{m+1}} + \frac{1}{2^{m+2}} + \cdots + \frac{1}{2^{m+p}} \\
    & = & \frac{1}{2^m} \left( 1-\frac{1}{2^p} \right) < \frac{1}{2^m}
  \end{eqnarray*}
  可见它也满足柯西收敛条件,所以这个数列也有极限,它的极限便是自然对数的底数$e$:
  \[ \lim_{n \to \infty} \sum_{i=1}^n \frac{1}{i!} = e \]
\end{example}

%%% Local Variables:
%%% mode: latex
%%% TeX-master: "../../book"
%%% End:
