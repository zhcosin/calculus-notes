
\section{一般项级数}
\label{sec:signed-series}

\subsection{交错级数}
\label{sec:alternating-sign-series}

\begin{definition}
  如果序列$\{a_n\}$的任意相邻两项的符号都相反,即整个序列交错的取正值和负值,则称其为\emph{交错序列},而对应级数$\sum_{i=1}^{\infty}a_n$为\emph{交错级数}.
\end{definition}

对于交错级数的收敛性有如下结论

\begin{theorem}
  如果数列$\{a_n\}(n=0,1,\ldots)$单调递减并趋于零,则级数$\sum_{i=0}^{\infty}(-1)^{n}a_n$收敛。
\end{theorem}

\begin{proof}[证明]
级数$\sum_{i=0}^{\infty}(-1)^{n-1}a_n$与级数$\sum_{i=0}^n(-1)^na_n$的收敛性是相同的,这里只是为了方便而让首项$a_0$的符号是正的。  

作级数的部分和$S_n=a_0-a_1+a_2-\cdots+(-1)^na_n$,显然
\[ S_{2n+1}=(a_0-a_1)+(a_2-a_3)+\cdots+(a_{2n}-a_{2n+1}) \]
由$\{a_n\}$单调递减可知$S_{2n+1}$是单调增加的正项数列,但是
\[ S_{2n+1}=a_0-(a_1-a_2)-(a_3-a_4)-\cdots-(a_{2n-1}-a_{2n})-a_{2n+1} \]
显然就有$S_{2n+1}<a_0$,即$S_{2n+1}$又有上界,所以$S_{2n+1}$收敛,同样的方法还可得出$S_{2n}$是单调递减有下界,因而也收敛,而由$S_{2n+1}=S_{2n}+a_{2n+1}$及$a_n \to 0(n \to \infty)$可知这两个子列只能收敛到相同的极限,即级数$\sum_{i=0}^{\infty}(-1)^na_n$收敛。
\end{proof}

\subsection{绝对收敛级数及其性质}
\label{sec:abs-converage-series-and-its-properties}

\begin{definition}
  对于级数$\sum_{n=1}^{\infty}a_n$,如果它的各项的绝对值相加而得的新级数$\sum_{n=1}^{\infty}|a_n|$收敛,则称原来的级数是 \emph{绝对收敛} 的.
\end{definition}

\begin{theorem}
  如果一个级数绝对收敛,则它一定收敛。
\end{theorem}

\begin{proof}[证明]
  由柯西准则,如果级数$\sum_{n=1}^{\infty}a_{n}$绝对收敛,则对于任意$\varepsilon>0$,存在正整数$N$,使得对于任意$n>N$和任意正整数$m$成立
  \[ |a_{n+1}|+|a_{n+2}|+\cdots+|a_{n+m}| < \varepsilon \]
  于是也有
  \[ |a_{n+1}+a_{n+2}+\cdots+a_{n+m}| < \varepsilon \]
  所以原来的级数也收敛。
\end{proof}

注意收敛级数并一定都是绝对收敛,如交错级数$\sum_{n=1}^{\infty}(-1)^{n-1}\frac{1}{n}$。

下面的定理提示了绝对收敛级数的一个非常重要的性质
\begin{theorem}
  如果一个级数绝对收敛,则将它的项任意重新排列后所得新级数仍然绝对收敛,且其和不变。
\end{theorem}

\begin{proof}[证明]
  对于绝对收敛的级数$\sum_{n=1}^{\infty}a_n$,将它的项任意重新排列后所得新级数记为$\sum_{n=1}^{\infty}b_n$,记$A_n=\sum_{i=1}^n|a_i|$,$B_n=\sum_{i=1}^n|b_i|$,则对于任一$B_n$,组成它的各个$b_i$在原来的级数中的下标最大值记为$m$,则显然$B_n \leqslant A_m$,于是$B_n$单调增加有上界,故级数$\sum_{n=1}^{\infty}b_n$绝对收敛。
\end{proof}

以下定理深刻提示了绝对收敛级数与非绝对收敛级数之间的区别。
\begin{theorem}[黎曼定理]
  如果一个级数收敛但非绝对收敛,则将它的项进行适当的重新排列后,可使新级数收敛到任意预先指定的实数。
\end{theorem}

%%% Local Variables:
%%% mode: latex
%%% TeX-master: "../../calculus-note"
%%% End:
