
\section{实数理论}
\label{sec:real-number-theory}

分析学的基础建立在实数的公理化体系之上,在讨论极限理论之前,先来讨论一下实数的理论。

\subsection{实数的十进制表示与大小关系}
\label{sec:decimal-system}

在人类历史上,为了计数而引进了自然数,最初以算筹的数量代表对应的数字,但这对于较大的数比较困难,为了表示数100就需要100根算筹,于是发明了十进制,这样所需的算筹数量就大大减少,之所以是十进制很可能是因为人正好有十根手指头,便于比划数字。后来为了解决多人平分食物等生活资料的问题又引进了整数之比即有理数的概念,再往后毕达哥拉斯学派根据勾股定理,发现了边长为1的正方形的对角线的长度不是有理数,引发了第一次数学危机,这次危机随着无理数的引入而得以解决。有理数与无理数一起,构成了全体实数。但在实数范围内,像$x^2+1=0$这样的代数方程没有解,为了从理论上解决这个问题而引入了虚数的概念,实数与虚数一起构成了复数,代数方程的理论在复数范围内得到彻底的解决。

本节只讨论实数。在十进制下,一个实数$x$具有如下表示:
\begin{equation}
  \label{eq:decimal-format-of-real}
 x=a_na_{n-1}\cdots a_1a_0.a_{-1}a_{-2}\cdots 
\end{equation}
其中$a_i \in \{0,1,2,3,4,5,6,7,8,9 \}$,并且最左边的数位$a_n$非零(否则省略这一位不写),十进制就是说,这个式子表示的数值其实是
\[ x=10^na_n+10^{n-1}a_{n-1}+\cdots+10a_1+a_0+\frac{a_{-1}}{10}+\frac{a_{-2}}{10^2}+\cdots \]
即\autoref{eq:decimal-format-of-real}实际表示的数值是它每一位数字与该数位上的权值之积的和,这一点是十分重要的,因为这样我们就只需要$0-9$这十个数符就可以表示出任意实数,而不必为每一个数都去发明一个对应的数符,那样既是不可能的,也是很难使用的。

在这种表示下,数位$a_0$称为个位,$a_1$称为十位,$a_2$称为百位,依次类推,在$a_0$以后的部分称为小数部分,$a_0$以及$a_0$左边的部分称为整数部分,两部分之间用小数点来分隔出明确的界限。

需要说明的是,实数十进制表示的小数部分是可以无限延伸的,但整数部分只能是有限位,并且规定,如果小数部分从某一位起全部都是零,则可以省写这些零,这样的小数称为有限小数,否则便称为无限小数。

如果无限小数的小数部分有连续重复出现的片段,例如 $0.12345678678678678\cdots$,这以后的数位全是重复的片段$678$,就称这小数为循环小数,并简写为$0.12345\dot{6}7\dot{8}$,即在循环片段的首尾两个数字上加点。如果没有这样的连续重复出现片段,则称为无限不循环小数。

关于整数的一个极为深刻的结论是
\begin{theorem}[带余除法]
  对任意两个整数$a$和$b$,其中$b$为正整数,则存在唯一一对整数$q$与$r(0\leqslant r < b)$,使得$a=qb+r$成立.这整数$q$及$r$分别称为$a$除以$b$所得的\emph{商}和\emph{余数}.
\end{theorem}

\begin{proof}[证明]
  以$b$的倍数为界点将全体实数划分为区间序列$\ldots,[-2b,-b),[-b,0),[0,b),[b,2b),\ldots$,这些左闭右开区间两两无交集,且它们的并集就是全体实数,那么整数$a$必定从属于其中某一个区间,假定是$[mb,(m+1)b)$,则取$q=m,r=a-mb$即满足定理条件,反过来,如果还有另一组$q_1$及$r_1$满足定理中条件,那么有$q_1b \leqslant a < (q_1+1)b$,这即表明$q_1=m$,从而$r_1=a_{mb}$,这就证得了商及余数的唯一性。
\end{proof}

利用带余除法,可以证明
\begin{theorem}
  有理数都是无限循环小数.
\end{theorem}

\begin{proof}[证明]
  设有理数$\frac{a}{b}$,其中$a$与$b$是整数,由于这结论与数的符号无关,所以假定这分子分母还是正的。这个证明过程其实就是两个正整数做除法的过程,思路就是在这个除法过程中,每一步所得的余数,或者是零从而被除尽,或者便要重复出现.

  先用$a$除以$b$,记商与余数分别为$q$及$r$,即$a=qb+r(0\leqslant r < b)$,如果$r>0$,再用$10r$除以$b$,所得的商与余数分别记为$q_1$与$r_1$,如果仍然有$r_1>0$,则再将$10r_1$除以$b$得到商$q_2$与余数$r_2$,依次类推,得到序列$q_i$与$r_i$,这时有$q_i(i \leqslant 1)$只能取$0$到$9$中的数字,这是因为$10r_{i-1}=q_ib+r_i$,而$0 \leqslant r_{i-1} < b$,所以$q_i$不能超过9,而由于$0 \leqslant r_i < b$,所以$r_i$也只能在集合$\{0,1,2,\ldots,b-1\}$这个有限集中取值,如果某一次取到了零$r_m=0$,则这个除法过程就结束了,而最终有
  \[ \frac{a}{b} = q + \sum_{i=0}^{m-1}\frac{q_i}{10^i} = q.q_1q_2\cdots q_{m-1} \]
  即为有限小数。如果$r_i$始终不能取到零,那么必然存在某个$i$及$j(> i)$使得$r_i=r_j$,既然出现了相同的余数,那么在分别用$10r_i$和$10r_j$去除以$b$时也会得出相同的商$q_{i+1}$和$q_{j+1}$,于是进一步出现相同的$r_{i+1}$与$r_{j+1}$,这个过程将无限重复下去,这时就有
  \[ \frac{a}{b} = q.q_1q_2 \cdots q_iq_{i+1} \cdots q_jq_{j+1} \cdots \]
  这里从$q_i$到$q_{j-1}$便是一个重复片段,为小数的循环部分(不一定是最小循环片段),即为无限循环小数。
\end{proof}

\subsection{最小自然数原理}
\label{sec:minimum-nature-number-principle}

\subsection{确界定理}
\label{sec:least-bound-theorem}

对于一个实数集,如果存在实数$M$,使得集合中的全部数$x$都满足$x \leqslant M$,则称实数$M$是这数集的一个\emph{上界},如果不等式是反向的,则称这实数是这数集的一个\emph{下界},显然,如果$M$是某个数集的上界,则比$M$大的所有实数也都是这数集的上界,对下界亦有类似结论。

如果数集既有上界又有下界,则称数集\emph{有界},有界数集的所有项的数值能够被某个区间所全部包含。

有界的另一种表述是,存在正实数$M>0$,使得数集的全部数$x$都满足$|x| \leqslant M$,这与前述说法是等价的。

\begin{definition}
对于一个有上界的实数集,如果某个实数$M$满足: (1)它是这数集的上界. (2)对于无论多么小的正实数$\epsilon$,总存在数集中的数$x$使得$x>M-\epsilon$,则称实数$M$是这数集的\emph{上确界},类似的有\emph{下确界}的定义.
\end{definition}

显然,上确界是最小的上界,下确界是最大的下界。

\begin{theorem}[确界定理]
若实数集合(无论有限集无限集)有上界,则有上确界,下界亦有相应结论。
\end{theorem}

\begin{proof}[证明]
设实数集合$A$有上界$M$,我们先构造出一个数$K$,再证明构造出的这个数正是这集合的上确界。

根据最小数原理,集合$A$中元素的整数部分有最大值,令$K$的整数部分与之相同,这整数部分记作$K_0$。

再将集合$A$中所有元素乘以10后舍去小数部分,这些新数组成的新集合记作$A_1$,这集合有上界$10M$,因此按最小数原理,它也有最大值,而且这最大值除个位以外的部分正是$K_0$(按$K_0$的定义),取这最大值的个位作为$K$的十分位。$K$的其余数位依次类推,$K$在$10^{-n}$上的数位是将集合$A$中全体元素乘以$10^n$后舍去小数部分所得新集合中最大数的个位数。

现在证明,数$K$是集合$A$的上确界,先证明它是上界,反证法,若它不是上界,则$A$中存在比它更大的数$x_0$,那么按实数大小关系定义,在比较$x_0$与$K$时,从左边开始往右比较,第一个不相同的数位上,$x_0$在该数位上的数大于$K$在该数位上的数,但这与$K$在这一数位上的数值的确定方法相矛盾,所以$K$是上界。其次需要证明,$K$是最小的上界,设$L$是一个小于$K$的实数,那么它与$K$相比,从左边开始第一个不相同的数位上,它对应的数较小,假定这数位就是$10^{-n}$,并设$K$和$L$在舍去这一数位以后的全部数位后所得的数分别是$K_n$和$L_n$,那么$K_n>L_n$,但根据$K$的确定过程可知,对于任何正整数$n$,$A$中都存在不小于$K_n$的数,自然这数也就大于$L_n$,因此$K$是最小的上界,即为上确界。
\end{proof}

\subsection{复数}
\label{sec:complex-number}



%%% Local Variables:
%%% mode: latex
%%% TeX-master: "../../book"
%%% End:
