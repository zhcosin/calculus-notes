
\section{刻画实数完备性的几个定理}
\label{sec:some-theorems-for-real-completeness}


\begin{theorem}[确界定理]
  如果数集有上界,则必有上确界,如果有下界,则必有下确界。
\end{theorem}

\begin{theorem}[单调有界定理]
  单调递增有上界的数列收敛,单调递减有下界的数列也收敛。
\end{theorem}

\begin{theorem}[柯西收敛准则]
  数列$x_n$收敛的充分必要条件是,对于任意正实数$\epsilon$,都存在正整数$N$,使得对于任意满足$n_1>N,n_2>N$的$n_1,n_2$都成立$|x_{n_1}-x_{n_2}| < \epsilon$。
\end{theorem}

\begin{theorem}[闭区间套定理]
  如果闭区间无穷序列$[a_i,b_i](i=1,2,\ldots)$满足两个条件: (1)$[a_k,b_k]\supset[a_{k+1},b_{k+1}](i=1,2,\ldots)$,(2)$\lim_{n\to\infty}(b_n-a_n)=0$,则存在唯一实数,同时位于所有闭区间内。
\end{theorem}

\begin{theorem}[有限覆盖定理]
  如果有无穷多个开区间的并集覆盖了一个闭区间,那么能够从中选取有限个开区间,它们的并集就足够覆盖这个闭区间了。
\end{theorem}

\begin{definition}
  对于一个数集和一个实数,如果在这实数的任意空心邻域内都有这数集中的数,则这实数称为这数集的\emph{聚点}。
\end{definition}

容易知道,假若$x$是数集$A$的一个聚点,则它的任意邻域内必包含数集中的无穷多个数。

\begin{theorem}[聚点定理]
  有界的无穷数集必有聚点。
\end{theorem}



%%% Local Variables:
%%% mode: latex
%%% TeX-master: "../../calculus-note"
%%% End:
