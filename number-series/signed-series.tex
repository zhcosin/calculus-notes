
\section{一般项级数}
\label{sec:signed-series}

\subsection{交错级数}
\label{sec:alternating-sign-series}

\begin{definition}
  如果序列$\{a_n\}$的任意相邻两项的符号都相反,即整个序列交错的取正值和负值,则称其为\emph{交错序列},而对应级数$\sum_{i=1}^{\infty}a_n$为\emph{交错级数}.
\end{definition}

对于交错级数的收敛性有如下结论

\begin{theorem}
  如果数列$\{a_n\}(n=0,1,\ldots)$单调递减并趋于零,则级数$\sum_{i=0}^{\infty}(-1)^{n}a_n$收敛。
\end{theorem}

\begin{proof}[证明]
级数$\sum_{i=0}^{\infty}(-1)^{n-1}a_n$与级数$\sum_{i=0}^n(-1)^na_n$的收敛性是相同的,这里只是为了方便而让首项$a_0$的符号是正的。  

作级数的部分和$S_n=a_0-a_1+a_2-\cdots+(-1)^na_n$,显然
\[ S_{2n+1}=(a_0-a_1)+(a_2-a_3)+\cdots+(a_{2n}-a_{2n+1}) \]
由$\{a_n\}$单调递减可知$S_{2n+1}$是单调增加的正项数列,但是
\[ S_{2n+1}=a_0-(a_1-a_2)-(a_3-a_4)-\cdots-(a_{2n-1}-a_{2n})-a_{2n+1} \]
显然就有$S_{2n+1}<a_0$,即$S_{2n+1}$又有上界,所以$S_{2n+1}$收敛,同样的方法还可得出$S_{2n}$是单调递减有下界,因而也收敛,而由$S_{2n+1}=S_{2n}+a_{2n+1}$及$a_n \to 0(n \to \infty)$可知这两个子列只能收敛到相同的极限,即级数$\sum_{i=0}^{\infty}(-1)^na_n$收敛。
\end{proof}

\subsection{绝对收敛级数及其性质}
\label{sec:abs-converage-series-and-its-properties}

\begin{definition}
  对于级数$\sum_{n=1}^{\infty}a_n$,如果它的各项的绝对值相加而得的新级数$\sum_{n=1}^{\infty}|a_n|$收敛,则称原来的级数是 \emph{绝对收敛} 的.
\end{definition}

\begin{theorem}
  如果一个级数绝对收敛,则它一定收敛。
\end{theorem}

\begin{proof}[证明]
  由柯西准则,如果级数$\sum_{n=1}^{\infty}a_{n}$绝对收敛,则对于任意$\varepsilon>0$,存在正整数$N$,使得对于任意$n>N$和任意正整数$m$成立
  \[ |a_{n+1}|+|a_{n+2}|+\cdots+|a_{n+m}| < \varepsilon \]
  于是也有
  \[ |a_{n+1}+a_{n+2}+\cdots+a_{n+m}| < \varepsilon \]
  所以原来的级数也收敛。
\end{proof}

注意收敛级数并一定都是绝对收敛,如交错级数$\sum_{n=1}^{\infty}(-1)^{n-1}\frac{1}{n}$。

下面的定理提示了绝对收敛级数的一个非常重要的性质
\begin{theorem}
  如果一个级数绝对收敛,则将它的项任意重新排列后所得新级数仍然绝对收敛,且其和不变。
\end{theorem}

\begin{proof}[证明]
  由于正项收敛级数任意重排后仍然收敛且和不变,因此将一个绝对收敛的(非全正项)级数的项任意重排后显然仍然是绝对收敛的。

  下面来证明,任意重排其项的顺序,其和也不会变.

  绝对收敛的级数,都可以看成由一个正项级数通过对各项添加正负号而得到, 记此正项级数为 $\sum_{n=1}^{\infty}a_n$,其和记作$A$,部分和记作$A_n$,而原来的级数可以写成 $\sum_{n=1}^ns_na_n$,其中$s_k\in\{-1,1\}$为符号因子,其部分和记作 $T_n$.

  $T_n$显然可以分为两部分,一部分是取正号的那些项之和,一部分是取负号的那些项之和,为此记
  \[
    \lambda_k=
    \begin{cases}
      1,  & \quad s_k=1 \\
      0,  & \quad s_k=-1
    \end{cases}
    , \quad
    \mu_k =
    \begin{cases}
     0, \quad & s_k=1 \\
     1, \quad & s_k=-1
    \end{cases}
  \]
  $\lambda_k$对于取正号的那些项其值为1,对于取负号的那些项其值为0,$\mu_k$则对取正号的那些项其值为0,对取负号的那些项其值为1,因此$\lambda_k$与$\mu_k$就是每一项取正负号的选择因子,现在再作两个级数$\sum_{n=1}^{\infty}\lambda_na_{n}$ 与 $\sum_{n=1}^{\infty}\mu_na_n$,显然这两个级数都是非负项级数,它们都是从正项级数$\sum_{n=1}^{\infty}a_n$中将部分项替换为0而得来.由于$ 0 \leqslant \lambda_ka_k \leqslant a_k$与 $ 0 \leqslant \mu_ka_k \leqslant a_k$,由比较判别法知这两个级数都收敛,记它们的和数分别为$P$和$Q$(都不超过$A$),而部分和分别记作$P_n$与$Q_n$.

  容易发现,在这样的定义下成立着 $s_k=\lambda_k-\mu_k$,因此原级数的部分和
  \[ T_n=\sum_{k=1}^ns_ka_k = \sum_{k=1}^n(\lambda_k-\mu_k)a_k = \sum_{k=1}^n\lambda_ka_k - \sum_{k=1}^n\mu_ka_k = P_n-Q_n \]
  于是$T_n$必以$P-Q$为其极限,从而原级数 $\sum_{n=1}^{\infty}s_na_n$收敛,且其和为 $P-Q$.

  这表明: 对一个正项的收敛级数的每一项任意添加正负符号,得到的新级数仍然收敛,并且其和等于添加正号的的子级数的和数减去添加负项的子级数的和数的差.

  现在考虑重排级数$\sum_{n=1}^{\infty}s_na_n$的项,对它的重排,对应着对级数 $\sum_{n=1}^{\infty}\lambda_na_n$与$\sum_{n=1}^{\infty}\mu_na_n$ 的重排,而后两个级数的项都是非负的,重排并不影响它们的收敛性,也不影响它们的和数,重排后的它俩仍然会收敛到$P$与$Q$,从而重排后的新级数的和仍然会是$P-Q$. 换句话说: 这个和数不取决于项的排列顺序,只取决于项的正负号的选择.
\end{proof}

以下定理深刻提示了绝对收敛级数与非绝对收敛级数之间的区别。
\begin{theorem}[黎曼定理]
  如果一个级数收敛但非绝对收敛,则将它的项进行适当的重新排列后,可使新级数收敛到任意预先指定的实数。
\end{theorem}

\subsection{幂级数}
\label{sec:power-series}

以下形状的级数称为 \emph{幂级数}:
\[ \sum_{n=0}^{\infty} a_nx^n = a_0+a_1x+a_2x^2+\cdots + a_nx^n+\cdots \]
该级数依赖于未知量$x$,其收敛性与 $x$ 的取值有关,如果当 $x=x_0$ 时上述幂级数收敛,则称上述幂级数在 $x_0$ 处收敛,显然,任意幂级数在 $x=0$ 处都收敛,此外,绝对收敛的概念同样可以照搬.

关于幂级数的收敛性,有如下结论
\begin{theorem}
  若幂级数 $\sum_{n=0}^{\infty}a_nx^n$ 在 $x=x_0(\neq 0)$处收敛,则它对于任意满足 $|x|<|x_0|$ 的 $x$ 均绝对收敛(从而更是收敛的).若它在 $x=x_0(\neq 0)$ 处发散,则对于任意符合 $|x|>|x_0|$的 $x$ 均发散.
\end{theorem}

\begin{proof}[证明]
  如果这幂级数在$x=x_0(\neq 0)$处收敛,那么对于满足 $|x|<|x_0|$的 $x$ 而言,有
  \[ |a_nx^n| = |a_nx_0^n| \cdot \left| \frac{x}{x_0} \right|^n \]
  由于级数在 $x_0$处的收敛,可知 $a_nx_0^n \to 0 (n \to \infty)$,从而 $a_nx_0^n$ 是有界的,设 $|a_nx_0^n| \leqslant M$,而 $q=\left| \frac{x}{x_0} \right| < 1$,因此
  \[ |a_nx^n| < Mq^n \]
  而级数 $\sum_{n=0}^{\infty} Mq^n$ 显然是收敛的,故而级数 $\sum_{n=0}^n|a_nx^n|$ 收敛,定理前半部分获证.

  定理的后半部分可以视前半部分的推论,因为如果这级数在 $x_0$ 处发散而在绝对值更大的 $x$ 处收敛,显然与前部分的结论是矛盾的.
\end{proof}

这定理显示能使幂级数 $\sum_{n=0}^{\infty} a_nx^n$收敛的 $x$ 的取值集合只能是以$0$为中心的某个区间(端点处的开闭暂不清楚),将这区间长度的一半就称为这幂级数的 \emph{收敛半径}.

%%% Local Variables:
%%% mode: latex
%%% TeX-master: "../calculus-note"
%%% End:
