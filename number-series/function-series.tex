
\section{函数级数}
\label{sec:function-series}

\subsection{一致收敛}
\label{sec:uniform-convergence}



\subsection{幂级数}
\label{sec:power-series}

以下形状的级数称为 \emph{幂级数}:
\[ \sum_{n=0}^{\infty} a_nx^n = a_0+a_1x+a_2x^2+\cdots + a_nx^n+\cdots \]
该级数依赖于未知量$x$,其收敛性与 $x$ 的取值有关,如果当 $x=x_0$ 时上述幂级数收敛,则称上述幂级数在 $x_0$ 处收敛,显然,任意幂级数在 $x=0$ 处都收敛,此外,绝对收敛的概念同样可以照搬.

关于幂级数的收敛性,有如下结论
\begin{theorem}
  若幂级数 $\sum_{n=0}^{\infty}a_nx^n$ 在 $x=x_0(\neq 0)$处收敛,则它对于任意满足 $|x|<|x_0|$ 的 $x$ 均绝对收敛(从而更是收敛的).若它在 $x=x_0(\neq 0)$ 处发散,则对于任意符合 $|x|>|x_0|$的 $x$ 均发散.
\end{theorem}

\begin{proof}[证明]
  如果这幂级数在$x=x_0(\neq 0)$处收敛,那么对于满足 $|x|<|x_0|$的 $x$ 而言,有
  \[ |a_nx^n| = |a_nx_0^n| \cdot \left| \frac{x}{x_0} \right|^n \]
  由于级数在 $x_0$处的收敛,可知 $a_nx_0^n \to 0 (n \to \infty)$,从而 $a_nx_0^n$ 是有界的,设 $|a_nx_0^n| \leqslant M$,而 $q=\left| \frac{x}{x_0} \right| < 1$,因此
  \[ |a_nx^n| < Mq^n \]
  而级数 $\sum_{n=0}^{\infty} Mq^n$ 显然是收敛的,故而级数 $\sum_{n=0}^n|a_nx^n|$ 收敛,定理前半部分获证.

  定理的后半部分可以视前半部分的推论,因为如果这级数在 $x_0$ 处发散而在绝对值更大的 $x$ 处收敛,显然与前部分的结论是矛盾的.
\end{proof}

这定理显示能使幂级数 $\sum_{n=0}^{\infty} a_nx^n$收敛的 $x$ 的取值集合只能是以$0$为中心的某个区间(端点处的开闭暂不清楚),将这区间长度的一半就称为这幂级数的 \emph{收敛半径}.


%%% Local Variables:
%%% mode: latex
%%% TeX-master: "../calculus-note"
%%% End:
