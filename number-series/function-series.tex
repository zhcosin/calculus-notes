
\section{函数级数}
\label{sec:function-series}

\subsection{基本概念}
\label{sec:concept-of-function-series}


如果级数的每一项的数值都依赖于某一个(共同的)变量,那么这就是一个 \emph{函数项级数},简称 \emph{函数级数},其形如
\[ \sum_{i=0}^n f_n(x) \]
当变量 $x$ 取定一个确定的数值 $x_0$ 时,函数级数的各项也就都具备了确定的数值,成为一个普通的数项级数
\[ \sum_{i=0}^n f_n(x_{0}) \]
这时就可以讨论其收敛性了,如果函数级数 $\sum_{i=0}^n f_n(x)$ 在 $x=x_0$ 时的数项级数是收敛的,就称函数级数在 $x=x_0$ 处收敛,如果函数级数在自变量从某个区间(或数集)上任意取值所得的数项级数都收敛,那么就称这函数级数在该区间(数集)上是收敛的.

如果函数级数在某个区间(或数集)上收敛,那么级数收敛到的数值显然也与自变量 $x$ 有关,这就形成一个新的函数 $f(x)$,即
\[ f(x) = \sum_{i=0}^n f_n(x) \]
注意 $f(x)$ 只有在使得函数级数收敛的那些自变量取值的数集上才有定义.

\subsection{一致收敛}
\label{sec:uniform-convergence}

与函数在区间上一致连续的概念相仿,对于函数级数来说,对于确定的很小的正实数 $\varepsilon$,满足 $|f(x) - \sum_{i=0}^m f_n(x)|<\varepsilon$ 成立的下标下限 $(m>)N$ 是与自变量 $x$ 有关的,而一致收敛则是指,对任意确定的任意小正实数$\varepsilon$,(对使得函数级数收敛的自变量的不同取值)存在共同的 $N$ 使得接近不等式成立,严格描述就是
\begin{definition}
  \label{uniform-convergence-of-function-series}
  设函数级数$\sum_{i=0}^n f_n(x)$与函数 $f(x)$ 在数集$I$上满足: 对于无论多么小的正实数 $\varepsilon$,都存在与$x$无关的正整数$N$,使得对一切 $n>N$ 成立 $|f(x) - \sum_{i=0}^m f_n(x)|<\varepsilon$,则称函数级数$\sum_{i=0}^n f_n(x)$在数集$I$上 \emph{一致收敛} 到 $f(x)$.
\end{definition}

显然,一致收敛是更强的收敛(即为收敛的充分不必要条件).


\subsection{幂级数}
\label{sec:power-series}

以下形状的函数级数称为 \emph{幂级数}:
\[ \sum_{n=0}^{\infty} a_nx^n = a_0+a_1x+a_2x^2+\cdots + a_nx^n+\cdots \]
显然,任意幂级数在 $x=0$ 处都收敛,此外,绝对收敛的概念同样可以照搬.

关于幂级数的收敛性,有如下结论
\begin{theorem}
  若幂级数 $\sum_{n=0}^{\infty}a_nx^n$ 在 $x=x_0(\neq 0)$处收敛,则它对于任意满足 $|x|<|x_0|$ 的 $x$ 均绝对收敛(从而更是收敛的).若它在 $x=x_0(\neq 0)$ 处发散,则对于任意符合 $|x|>|x_0|$的 $x$ 均发散.
\end{theorem}

\begin{proof}[证明]
  如果这幂级数在$x=x_0(\neq 0)$处收敛,那么对于满足 $|x|<|x_0|$的 $x$ 而言,有
  \[ |a_nx^n| = |a_nx_0^n| \cdot \left| \frac{x}{x_0} \right|^n \]
  由于级数在 $x_0$处的收敛,可知 $a_nx_0^n \to 0 (n \to \infty)$,从而 $a_nx_0^n$ 是有界的,设 $|a_nx_0^n| \leqslant M$,而 $q=\left| \frac{x}{x_0} \right| < 1$,因此
  \[ |a_nx^n| < Mq^n \]
  而级数 $\sum_{n=0}^{\infty} Mq^n$ 显然是收敛的,故而级数 $\sum_{n=0}^n|a_nx^n|$ 收敛,定理前半部分获证.

  定理的后半部分可以视前半部分的推论,因为如果这级数在 $x_0$ 处发散而在绝对值更大的 $x$ 处收敛,显然与前部分的结论是矛盾的.
\end{proof}

这定理显示能使幂级数 $\sum_{n=0}^{\infty} a_nx^n$收敛的 $x$ 的取值集合只能是以$0$为中心的某个区间(端点处的开闭暂不清楚),将这区间长度的一半就称为这幂级数的 \emph{收敛半径}.

\begin{example}
  借助常用基本初等函数的泰勒公式,可以罗列出以下的幂级数和对应的收敛半径.

  指数函数的幂级数
  \[ \mathrm{e}^x = 1 + x + \frac{x^2}{2!}+\cdots+\frac{x^n}{n!}+\cdots \]
  它在整个 $\mathbb{R}$ 上都收敛.
\end{example}

%%% Local Variables:
%%% mode: latex
%%% TeX-master: "../calculus-note"
%%% End:
