
\section{微分中值定理}
\label{sec:differential-mean-value-theorems}

\subsection{费马极值定理}
\label{sec:fermat-limit-value-theorem}

\begin{theorem}
  设函数$f(x)$在$x_0$的某邻域内有定义,如果它在$x_0$处取得极值(极大或者极小均可)且在$x_0$处可导,则必有$f'(x_0)=0$.
\end{theorem}

\begin{proof}[证明]
  假如函数在$x_0$处取的是极大值,用反证法证明该点处如果可导,则导数只能是零,这是因为,假设$f'(x_0)>0$,则必定存在$x_0$的某个充分小的邻域,在此邻域上恒有
  \[ \frac{f(x)-f(x_0)}{x-x_0} > 0 \]
  于是在这个充分小的邻域的右侧部分,就有$f(x)>f(x_0)$,这与函数在$x_0$处达到极大值相矛盾. 同理,如果$f'(x_0)<0$,则在$x_0$的某个充分小的左邻域上,亦必有$f(x)>f(x_0)$,同样导致矛盾,因此,如果$f'(x_0)$存在,则只能是零。
\end{proof}

%%% Local Variables:
%%% mode: latex
%%% TeX-master: "../calculus-note"
%%% End:
