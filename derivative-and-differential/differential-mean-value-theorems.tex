
\section{微分中值定理}
\label{sec:differential-mean-value-theorems}

\subsection{费马极值定理}
\label{sec:fermat-limit-value-theorem}

\begin{theorem}
  设函数$f(x)$在$x_0$的某邻域内有定义,如果它在$x_0$处取得极值(极大或者极小均可)且在$x_0$处可导,则必有$f'(x_0)=0$.
\end{theorem}

\begin{proof}[证明]
  假如函数在$x_0$处取的是极大值,用反证法证明该点处如果可导,则导数只能是零,这是因为,假设$f'(x_0)>0$,则必定存在$x_0$的某个充分小的邻域,在此邻域上恒有
  \[ \frac{f(x)-f(x_0)}{x-x_0} > 0 \]
  于是在这个充分小的邻域的右侧部分,就有$f(x)>f(x_0)$,这与函数在$x_0$处达到极大值相矛盾. 同理,如果$f'(x_0)<0$,则在$x_0$的某个充分小的左邻域上,亦必有$f(x)>f(x_0)$,同样导致矛盾,因此,如果$f'(x_0)$存在,则只能是零。
\end{proof}


\subsection{罗尔定理}
\label{sec:rolle-theorem}

\begin{theorem}[罗尔(Rolle)定理]
  设函数$f(x)$在闭区间$[a,b]$内连续,在开区间$(a,b)$内可导,且有$f(a)=f(b)$,则开区间$(a,b)$上至少存在一个点$x_0$,使得$f'(x_0)=0$.
\end{theorem}

\begin{proof}[证明]
  因为闭区间上的连续函数必定同时存在着最大值和最小值,如果两个最值相等,则函数值恒为常数,此时结论自然是成立的,如果最大值和最小值不相等,则两个最值必然有一个与两个端点处的函数值不同,从而只能在开区间内取得,于是按费马极值定理,该点处的导数便是零。
\end{proof}

\subsection{拉格朗日中值定理}
\label{sec:lagrange-middle-value-theorem}

\begin{theorem}[拉格朗日(Lagrange)中值定理]
  设函数$f(x)$在闭区间$[a,b]$上连续,在开区间$(a,b)$内可导,则存在$x_0 \in (a,b)$,使得
  \[ f'(x_0) = \frac{f(b)-f(a)}{b-a} \]
\end{theorem}

这定理的几何意义是,闭区间上的连续函数的图象上,存在某个点处的切线平行于两个端点的连线。

\begin{proof}[证明]
  两个端点相连直线对应的线性函数是
  \[ g(x) = f(a) + \frac{f(b)-f(a)}{b-a}(x-a) \]
  构造函数
  \[ L(x) = f(x)-g(x) \]
  容易验证,函数$L(x)$满足罗尔定理的条件,由罗尔定理即得结论。
\end{proof}

在证明了拉格朗日定理之后,我们注意到一个的现象,罗尔定理是拉格朗日定理的特殊情形,但由特殊情形的罗尔定理可以得到一般情形的拉格朗日定理,这与我们通常的认识相违背,但这揭示了一种特殊与一般之间的等价关系,这在数学上很多地方都可以看到这种关系,有时候为了证明一个定理,我们往往先证明定理的特殊情形,再由特殊情形通过某种变换,可以得出一般情形的结论。

\subsection{柯西中值定理}
\label{sec:cauchy-middle-value-theorem}

\begin{theorem}[柯西(Cauchy)中值定理]
  设函数$f(x)$和$g(x)$都在闭区间$[a,b]$上连续且在开区间$(a,b)$内可导,并且两个函数的导数不同时为零以及$g(a) \neq g(b)$,则存在$x_0 \in (a,b)$,使得
  \[ \frac{f'(x_0)}{g'(x_0)} = \frac{f(b)-f(a)}{g(b)-g(a)} \]
\end{theorem}

\begin{proof}[证明]
  同拉格朗日定理相仿,作函数
  \[ h(x) = f(a) + \frac{f(b)-f(a)}{g(b)-g(a)}(g(x)-g(a)) \]
  再作
  \[ C(x) = f(x) - h(x) \]
  则可以验证,$C(x)$符合罗尔定理的条件,由罗尔定理即得结论。
\end{proof}

柯西中值定理可看作是参变量曲线的拉格朗日定理,假定一条曲线由参数方程
\[
  \begin{cases}
    x = g(t) \\
    y = f(t)
  \end{cases}
\]
确定,参数范围是$a \leqslant t \leqslant b$,那么它在$t=t_0$处的切线斜率便是$\frac{f'(t_0)}{g'(t_0)}$,这定理表明曲线上存在某点,该点处的切线平行于曲线两个端点的连线。

\subsection{导函数的进一步性质}
\label{sec:some-perproties-of-derivative-function}

\begin{theorem}[导函数介值定理]
  \label{theorem:value-range-of-derived-function}
  设函数$f(x)$在闭区间$[a,b]$内可导,且$f'_+(a) \neq f'_-(b)$,则对于介于$f'_+(a)$与$f'_-(b)$之间的任意实数$k$,都存在$x_0 \in (a,b)$,使得$ f'(x_0) = k $.
\end{theorem}

\begin{proof}[证明]
  作函数$g(x)=f(x)-kx$,则$g'(x)=f'(x)-k$,显然$g'(a)g'(b)<0$,不妨假设是$g'(a)>0$而$g'(b)<0$,则函数$g(x)$在$a$的某个右邻域上单调增加,同时在$b$的某个左邻域上单调减小,因此,连续函数$g(x)$必然在开区间$(a,b)$内的某点处达到它的最大值,记此点为$x_0$,由费马极值定理可知$g'(x_0)=0$,即$f'(x_0)=k$.
\end{proof}

\begin{theorem}
  \label{theorem:derived-function-limit-as-derived-value}
  设函数$f(x)$在$x_0$的某邻域内连续,且在其空心邻域内可导,如果导数函数$f'(x)$在$x_0$处存在左右极限,则函数在$x_0$处存在左导数和右导数,并且有
  \begin{eqnarray*}
   f_-'(x_0) & = & \lim_{x \to x_0^-} f'(x)  \\
   f_+'(x_0) & = & \lim_{x \to x_0^+} f'(x) 
  \end{eqnarray*}
\end{theorem}

\begin{proof}[证明]
  只证明右侧导数的部分,左导数也是完全类似的。设导函数$f'(x)$在$x_0$处的右极限是$A$,只要证明$f(x)$在$x_0$处右可导且导数值也是$A$即可,设$x>x_0$,则函数在$[x_0,x]$上显然满足拉格朗日中值定理的条件,于是存在$x_1 \in (x_0,x)$,使得
  \[ \frac{f(x)-f(x_0)}{x-x_0} = f'(x_1) \]
  当$x \to x_0^+$时,显然亦必有$x_1 \to x_0^+$,而导函数$f'(x)$在$x_0$处有右极限为$A$,所以
  \[ \lim_{x \to x_0^+} \frac{f(x)-f(x_0)}{x-x_0} = \lim_{x_1 \to x_0^+} f'(x_1) = A \]
  这就表明函数$f(x)$在$x_0$处右可导,且导数值为$A$.
\end{proof}

这两个定理表明,导函数不可能有第一类间断点,在一定程度上具有连续函数的某些性质,也正是因此,不是任何一个函数都可以成为某个函数的导函数的。

\subsection{再证拉格朗日中值定理}
\label{sec:second-proof-for-lagrange-middle-value-theorem}

 受 \autoref{theorem:value-range-of-derived-function} 和 \autoref{theorem:derived-function-limit-as-derived-value} 启发,可以得到拉格朗日中值定理的一个新的证明方法,先证明下面的结论:
\begin{theorem}
  如果函数$f(x)$在闭区间$[a,b]$内连续且在开区间$(a,b)$内可导,且$f(a)=f(b)$,若函数 $f(x)$ 在该闭区间上不是常函数,则存在两个不同的 $\xi,\eta \in (a,b)$,使得 $f'(\xi)>0$及 $f'(\eta)<0$.
\end{theorem}

该定理的几何意义是容易理解的.

\begin{proof}
  因为函数$f(x)$在该闭区间上不是常函数,那么必在某一处 $c$ 取异于 $f(a)=f(b)$的函数值,于是函数 $f(x)$在该闭区间上不可能是单调函数,从而导函数不可能总是非负或者总是非正,于是结论成立.
\end{proof}

接下来证明拉格朗日中值定理: 若函数 $f(x)$ 在闭区间$[a,b]$内连续且在开区间$(a,b)$内可导,则存在 $\xi \in (a,b)$,使得
\[ f'(\xi) = \frac{f(b)-f(a)}{b-a} \]
\begin{proof}
  记
\[ \mu = \frac{f(b)-f(a)}{b-a} \]
作函数
\[ g(x) = f(a) +  \frac{f(b)-f(a)}{b-a} (x-a) = f(a) + \mu (x-a) \]
它的几何意义就是将 $f(x)$ 在区间 $[a,b]$ 上的图像的两个端点用直线段连接起来.再记
\[ h(x) = f(x) - g(x) \]
则 $g'(x) = \mu$ 而 $h'(x) = f'(x) - \mu$,且 $h(a)=h(b)=0$,如果 $h(x)=0$即$f(x)=g(x)$总是成立的,那么定理结论自然成立,在 $f(x)$不总是等于$g(x)$即$h(x)$不总取零值的情况下,由刚所证定理的结论可知存在两个不同的实数 $s,t \in (a,b)$,使得 $h'(s)>0$和$h'(t)<0$,即 $f'(s)>\mu$和$f'(t)<\mu$,于是由导函数介值性,存在 $\xi \in (a,b)$ 使得 $f'(\xi) = \mu$,拉格朗日定理得证.
\end{proof}

\subsection{洛必达法则}
\label{sec:L'Hopital-rule}

洛必达法则是一系列函数极限公式的统称。

第一个定理是关于$\frac{0}{0}$型的不定式的。
\begin{theorem}
  设$f(x)$与$g(x)$都在$x_0$的某个空心邻域内可导,且
  \begin{enumerate}
  \item $\lim_{x \to x_0} f(x) = \lim_{x \to x_0} g(x) = 0$.
  \item 在点$x_0$的某空心邻域内两者都可导,且$g'(x) \neq 0$.
  \item $\lim_{x \to x_0} \frac{f'(x)}{g'(x)}=A$,这里$A$可以是有限实数,也可以是无穷大或者带符号的无穷大.
  \end{enumerate}
  则有
  \[ \lim_{x \to x_0} \frac{f(x)}{g(x)} = A \]
\end{theorem}

\begin{proof}[证明]
  如果$f(x_0)$与$g(x_0)$不为零,则对它们进行连续开拓,命$f(x_0)=g(x_0)=0$,于是利用柯西中值定理,对任意$x \neq x_0$都存在介于$x_0$与$x$之间的$x_1$使得
  \[ \frac{f(x)}{g(x)} = \frac{f(x)-f(x_0)}{g(x)-g(x_0)} = \frac{f'(x_1)}{g'(x_1)} \]
  再令$x \to x_0$,亦必有$x_1 \to x_0$,于是便得出结论。
\end{proof}

\begin{example}
  今来证明
  \[ \lim_{x \to 0} \frac{1-\cos{x}}{x^2} = \frac{1}{2} \]
  可以验证分子分母两个函数符合上述定理的条件,并且导函数在$x=0$处的极限是
  \[ \lim_{x \to 0} \frac{\sin{x}}{2x} = \frac{1}{2} \]
  因此便得结论,这个极限的思路是这样来的,因为余弦函数在$x=0$处是连续的,所以我们知道$1-\cos{x}$是一个无穷小,即
  \[ \cos{x} = 1+o(1) \]
  我们将这个无穷小与$x$这个简单的无穷小进行比较,也就是考虑
  \[ \lim_{x \to 0} \frac{1-\cos{x}}{x} \]
  根据前面证明,显然这个极限是零,也就是说$1-\cos{x}$是$x$的高阶无穷小,即
  \[ \cos{x} = 1-o(x) \]
  于是我们再将它与$x^2$进行比较,前述结果表明它俩是同阶无穷小,或者说,$1-\cos{x}$与$\frac{1}{2}x^2$是等价无穷小,于是得
  \[ \cos{x} = 1-\frac{1}{2}x^2 + o(x^2) \]
  这个过程还可以继续进行下去,只要得到一个无穷小,就将它与最简单的无穷小$x,x^2,\ldots,x^n,\ldots$依次进行比较,直到得出同阶无穷小为止,然后再反复进行此步骤,依次剥离出无穷小的主要成分,剩下的是越来越高阶的无穷小,下面就是将$1-\cos{x}$与$\frac{1}{2}x^2$作差之后与$x^3$比较,仍然由洛必达法则可得
  \[ \lim_{x \to x_0} \frac{(1-\cos{x})-\frac{1}{2}x^2}{x} = 0 \]
  说明这个差是$x$的高阶无穷小,于是将它与$x^2$进行比较,得
  \[ \lim_{x \to x_0} \frac{(1-\cos{x})-\frac{1}{2}x^2}{x^2} = \lim_{x \to 0} \frac{\sin{x}-x}{2x} = \lim_{x \to 0} \frac{\sin{x}}{2x} - \frac{1}{2} = 0 \]
  说明$x^2$的阶数仍然低了,同样的办法可以知道$x^3$仍然低,这个差与$x^4$是同阶无穷小,比值的极限是$-\frac{1}{24}$,这就是说
  \[ 1-\cos{x}-\frac{1}{2}x^2 \sim -\frac{1}{24}x^4 \]
  或者写成
  \[ \cos{x} = 1-\frac{1}{2}x^2+\frac{1}{24}x^4 + o(x^4) \]
  因为余弦函数存在任意阶的导数,所以这个过程可以无限进行下去,接下来进行几步之后看起来是这样的
  \[ \cos{x} = 1 - \frac{1}{2}x^2 + \frac{1}{24}x^4 - \frac{1}{720}x^6 + \frac{1}{40320}x^8 + o(x^8) \]
  这意味着我们可以用多项式函数逐步逼近余弦函数,同样,对于正弦函数,我们也可以用同样的办法,可以得出
  \[ \sin{x} = x - \frac{1}{6}x^3 + \frac{1}{120}x^5 - \frac{1}{5040}x^7 + o(x^7) \]
  这提供了一个非常重要的思路,就是可以用多项式函数来无限逼近一个函数,这就是后面我们推导函数的多项式展开即泰勒公式的思想。
\end{example}

\begin{example}
  依照上例,从我们已知的极限
  \[ \lim_{x \to 0} \frac{\mathrm{e}^x-1}{x} = 1 \]
  出发,可以得出
  \[ \mathrm{e}^x = 1 + x + \frac{1}{2}x^2+\frac{1}{6}x^3+\frac{1}{24}x^4+\frac{1}{120}x^5+o(x^5) \]
  同样,由
  \[ \lim_{x \to 0} \frac{\ln{(1+x)}}{x} = 1 \]
  出发,可以得出
  \[ \ln{(1+x)} = x - \frac{1}{2}x^2+\frac{1}{3}x^3+\frac{1}{4}x^4-\frac{1}{5}x^5+o(x^5) \]
\end{example}

第二个是关于$\frac{\infty}{\infty}$型的不定式.
\begin{theorem}
  设函数$f(x)$与$g(x)$都在$x_0$右侧的某个空心邻域内可导,且
  \begin{enumerate}
  \item $\lim_{x \to x_0^+} f(x) = \lim_{x \to x_0^+} g(x) = \infty$.
  \item 两个函数都在$x_0$的该空心邻域内可导,且$g'(x) \neq 0$.
  \item $\lim_{x \to x_0^+} \frac{f'(x)}{g'(x)} = A$,这里$A$可以是实数,也可以是无穷大或者带符号的无穷大.
  \end{enumerate}
\end{theorem}

\begin{proof}[证明]
  在$x_0$的右侧任意取定一个数$x_1$,设$a_n$是一个数列,它的所有项都被包含在开区间$(x_0,x_1)$上并且严格减少并收敛到$x_0$,那么两个函数在这个数列上对应着两个值的数列:$f(a_n)$和$g(a_n)$,显然这两个数列都是无穷大,并且根据柯西中值定理,存在$x_n \in (a_{n+1},a_n)$,使得
  \[ \frac{f(a_{n+1})-f(a_n)}{g(a_{n+1})-g(a_n)} = \frac{f'(x_n)}{g'(x_n)} \]
  显然当$n \to \infty$时$x_n \to x_0$,因而有
  \[ \lim_{n \to \infty} \frac{f(a_{n+1})-f(a_n)}{g(a_{n+1})-g(a_n)} = A \]
  于是由Stolz定理,可知
  \[ \lim_{n \to \infty} \frac{f(a_n)}{g(a_n)} = A \]
  然后由于数列$a_n$的任意性以及数列极限与函数极限的关系,可得
  \[ \lim_{x \to x_0^+} \frac{f(x)}{g(x)} = A \]
\end{proof}

从证明过程可以看出,这个定理与Stolz定理非常相似,Stolz定理可以看作是这个定理的离散版本,而这个定理可以看作是Stolz定理的连续版本。

\subsection{插值公式}
\label{sec:interpolation-formula}



%%% Local Variables:
%%% mode: latex
%%% TeX-master: "../calculus-note"
%%% End:
