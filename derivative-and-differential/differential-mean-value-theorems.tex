
\section{微分中值定理}
\label{sec:differential-mean-value-theorems}

\subsection{费马极值定理}
\label{sec:fermat-limit-value-theorem}

\begin{theorem}
  设函数$f(x)$在$x_0$的某邻域内有定义,如果它在$x_0$处取得极值(极大或者极小均可)且在$x_0$处可导,则必有$f'(x_0)=0$.
\end{theorem}

\begin{proof}[证明]
  假如函数在$x_0$处取的是极大值,用反证法证明该点处如果可导,则导数只能是零,这是因为,假设$f'(x_0)>0$,则必定存在$x_0$的某个充分小的邻域,在此邻域上恒有
  \[ \frac{f(x)-f(x_0)}{x-x_0} > 0 \]
  于是在这个充分小的邻域的右侧部分,就有$f(x)>f(x_0)$,这与函数在$x_0$处达到极大值相矛盾. 同理,如果$f'(x_0)<0$,则在$x_0$的某个充分小的左邻域上,亦必有$f(x)>f(x_0)$,同样导致矛盾,因此,如果$f'(x_0)$存在,则只能是零。
\end{proof}


\subsection{罗尔定理}
\label{sec:rolle-theorem}

\begin{theorem}[罗尔(Rolle)定理]
  设函数$f(x)$在闭区间$[a,b]$内连续,在开区间$(a,b)$内可导,且有$f(a)=f(b)$,则开区间$(a,b)$上至少存在一个点$x_0$,使得$f'(x_0)=0$.
\end{theorem}

\begin{proof}[证明]
  因为闭区间上的连续函数必定同时存在着最大值和最小值,如果两个最值相等,则函数值恒为常数,此时结论自然是成立的,如果最大值和最小值不相等,则两个最值必然有一个与两个端点处的函数值不同,从而只能在开区间内取得,于是按费马极值定理,该点处的导数便是零。
\end{proof}

\subsection{拉格朗日中值定理}
\label{sec:lagrange-middle-value-theorem}

\begin{theorem}[拉格朗日(Lagrange)中值定理]
  设函数$f(x)$在闭区间$[a,b]$上连续,在开区间$(a,b)$内可导,则存在$x_0 \in (a,b)$,使得
  \[ f'(x_0) = \frac{f(b)-f(a)}{b-a} \]
\end{theorem}

这定理的几何意义是,闭区间上的连续函数的图象上,存在某个点处的切线平行于两个端点的连线。

\begin{proof}[证明]
  两个端点相连直线对应的线性函数是
  \[ g(x) = f(a) + \frac{f(b)-f(a)}{b-a}(x-a) \]
  构造函数
  \[ L(x) = f(x)-g(x) \]
  容易验证,函数$L(x)$满足罗尔定理的条件,由罗尔定理即得结论。
\end{proof}

在证明了拉格朗日定理之后,我们注意到一个的现象,罗尔定理是拉格朗日定理的特殊情形,但由特殊情形的罗尔定理可以得到一般情形的拉格朗日定理,这与我们通常的认识相违背,但这揭示了一种特殊与一般之间的等价关系,这在数学上很多地方都可以看到这种关系,有时候为了证明一个定理,我们往往先证明定理的特殊情形,再由特殊情形通过某种变换,可以得出一般情形的结论。

\subsection{柯西中值定理}
\label{sec:cauchy-middle-value-theorem}

\begin{theorem}[柯西(Cauchy)中值定理]
  设函数$f(x)$和$g(x)$都在闭区间$[a,b]$上连续且在开区间$(a,b)$内可导,并且两个函数的导数不同时为零以及$g(a) \neq g(b)$,则存在$x_0 \in (a,b)$,使得
  \[ \frac{f'(x_0)}{g'(x_0)} = \frac{f(b)-f(a)}{g(b)-g(a)} \]
\end{theorem}

\begin{proof}[证明]
  同拉格朗日定理相仿,作函数
  \[ h(x) = f(a) + \frac{f(b)-f(a)}{g(b)-g(a)}(g(x)-g(a)) \]
  再作
  \[ C(x) = f(x) - h(x) \]
  则可以验证,$C(x)$符合罗尔定理的条件,由罗尔定理即得结论。
\end{proof}

柯西中值定理可看作是参变量曲线的拉格朗日定理,假定一条曲线由参数方程
\[
  \begin{cases}
    x = g(t) \\
    y = f(t)
  \end{cases}
\]
确定,参数范围是$a \leqslant t \leqslant b$,那么它在$t=t_0$处的切线斜率便是$\frac{f'(t_0)}{g'(t_0)}$,这定理表明曲线上存在某点,该点处的切线平行于曲线两个端点的连线。

\subsection{导函数的进一步性质}
\label{sec:some-perproties-of-derivative-function}

\begin{theorem}[导函数介值定理]
  \label{theorem:value-range-of-derived-function}
  设函数$f(x)$在闭区间$[a,b]$内可导,且$f'_+(a) \neq f'_-(b)$,则对于介于$f'_+(a)$与$f'_-(b)$之间的任意实数$k$,都存在$x_0 \in (a,b)$,使得$ f'(x_0) = k $.
\end{theorem}

\begin{proof}[证明]
  作函数$g(x)=f(x)-kx$,则$g'(x)=f'(x)-k$,显然$g'(a)g'(b)<0$,不妨假设是$g'(a)>0$而$g'(b)<0$,则函数$g(x)$在$a$的某个右邻域上单调增加,同时在$b$的某个左邻域上单调减小,因此,连续函数$g(x)$必然在开区间$(a,b)$内的某点处达到它的最大值,记此点为$x_0$,由费马极值定理可知$g'(x_0)=0$,即$f'(x_0)=k$.
\end{proof}

\begin{theorem}
  \label{theorem:derived-function-limit-as-derived-value}
  设函数$f(x)$在$x_0$的某邻域内连续,且在其空心邻域内可导,如果导数函数$f'(x)$在$x_0$处存在左右极限,则函数在$x_0$处存在左导数和右导数,并且有
  \begin{eqnarray*}
   f_-'(x_0) & = & \lim_{x \to x_0^-} f'(x)  \\
   f_+'(x_0) & = & \lim_{x \to x_0^+} f'(x) 
  \end{eqnarray*}
\end{theorem}

\begin{proof}[证明]
  只证明右侧导数的部分,左导数也是完全类似的。设导函数$f'(x)$在$x_0$处的右极限是$A$,只要证明$f(x)$在$x_0$处右可导且导数值也是$A$即可,设$x>x_0$,则函数在$[x_0,x]$上显然满足拉格朗日中值定理的条件,于是存在$x_1 \in (x_0,x)$,使得
  \[ \frac{f(x)-f(x_0)}{x-x_0} = f'(x_1) \]
  当$x \to x_0^+$时,显然亦必有$x_1 \to x_0^+$,而导函数$f'(x)$在$x_0$处有右极限为$A$,所以
  \[ \lim_{x \to x_0^+} \frac{f(x)-f(x_0)}{x-x_0} = \lim_{x_1 \to x_0^+} f'(x_1) = A \]
  这就表明函数$f(x)$在$x_0$处右可导,且导数值为$A$.
\end{proof}

这两个定理表明,导函数不可能有第一类间断点,在一定程度上具有连续函数的某些性质,也正是因此,不是任何一个函数都可以成为某个函数的导函数的。

\subsection{再证拉格朗日中值定理}
\label{sec:second-proof-for-lagrange-middle-value-theorem}

 受 \autoref{theorem:value-range-of-derived-function} 和 \autoref{theorem:derived-function-limit-as-derived-value} 启发,如果给拉格朗日中值定理的条件再加上端点处可导的条件(有限的或带符号的无空大均可),则可以得到定理的另一种证明方法,为此我们先证明如下定理
\begin{theorem}
  如果函数$f(x)$在闭区间$[a,b]$内连续且可导(端点处存在单侧导数,导数值可为有限值也可为带符号的无穷大),且$f(a)=f(b)=0$,若两端点处的单侧导数值$f_{+}'(a)$ 与 $f_-'(b)$同号,则存在 $\xi \in (a,b)$,使得 $f'(\xi)$ 与两端点处的单侧导数值异号. 
\end{theorem}

该定理的几何意义是容易理解的.

\begin{proof}
 以下用反证法证明定理结论,不妨假定 $f_{+}'(a)$ 与 $f_-'(b)$ 都是正的(正有限值或正无穷大),如果定理结论不成立,那么就是说 $f'(x) \geqslant 0$ 总是成立的,那意味着函数 $f(x)$ 在闭区间$[a,b]$上是单调增加的。而由 $f'_{+}(a) >0$,可知在$a$的某个右邻域内存在$\eta$使得 $f(\eta)>f(a)=0$,于是$f(\eta)>0=f(b)$,与单调性矛盾,因此定理成立. 
\end{proof}

接下来证明拉格朗日中值定理(增加了端点处可导的条件): 若函数 $f(x)$ 在闭区间$[a,b]$内连续且可导(端点处单侧可导,导数值可以为有限也可以为带符号的无穷大),则存在 $\xi \in (a,b)$,使得
\[ f'(\xi) = \frac{f(b)-f(a)}{b-a} \]
\begin{proof}
  记
\[ \mu = \frac{f(b)-f(a)}{b-a} \]
若 $\mu$ 正好介于 $f_{+}'(a)$ 与 $f_-'(b)$ 之间,那么由导函数介值定理即可知结论成立,现在假定 $f_{+}'(a)$ 与 $f_-'(b)$ 位于 $\mu$ 的同侧,不妨假定都大于 $\mu$,作函数
\[ g(x) = f(a) +  \frac{f(b)-f(a)}{b-a} (x-a) = f(a) + \mu (x-a) \]
它的几何意义就是将 $f(x)$ 在区间 $[a,b]$ 上的图像的两个端点用直线段连接起来.再记
\[ h(x) = f(x) - g(x) \]
则 $g'(x) = \mu$ 而 $h'(x) = f'(x) - \mu$,且 $h(a)=h(b)=0$,由 $f_{+}'(a)>\mu$ 与 $f_-'(b)>\mu$ 知 $h_{+}'(a)>0$ 与 $h_-'(b)>0$,于是由刚所证定理的结论可知存在 $s \in (a,b)$,使得 $h'(s)<0$,即 $f'(s)<\mu$,于是 $\mu$ 就介于 $f_{+}'(a)$与 $f'(s)$之间,由导函数介值性,存在 $\xi \in (a,b)$ 使得 $f'(\xi) = \mu$,拉格朗日定理得证.
\end{proof}

\subsection{洛必达法则}
\label{sec:L'Hopital-rule}

洛必达法则是一系列函数极限公式的统称。

第一个定理是关于$\frac{0}{0}$型的不定式的。
\begin{theorem}
  设$f(x)$与$g(x)$都在$x_0$的某个空心邻域内可导,且
  \begin{enumerate}
  \item $\lim_{x \to x_0} f(x) = \lim_{x \to x_0} g(x) = 0$.
  \item 在点$x_0$的某空心邻域内两者都可导,且$g'(x) \neq 0$.
  \item $\lim_{x \to x_0} \frac{f'(x)}{g'(x)}=A$,这里$A$可以是有限实数,也可以是无穷大或者带符号的无穷大.
  \end{enumerate}
  则有
  \[ \lim_{x \to x_0} \frac{f(x)}{g(x)} = A \]
\end{theorem}

\begin{proof}[证明]
  如果$f(x_0)$与$g(x_0)$不为零,则对它们进行连续开拓,命$f(x_0)=g(x_0)=0$,于是利用柯西中值定理,对任意$x \neq x_0$都存在介于$x_0$与$x$之间的$x_1$使得
  \[ \frac{f(x)}{g(x)} = \frac{f(x)-f(x_0)}{g(x)-g(x_0)} = \frac{f'(x_1)}{g'(x_1)} \]
  再令$x \to x_0$,亦必有$x_1 \to x_0$,于是便得出结论。
\end{proof}

\begin{example}
  今来证明
  \[ \lim_{x \to 0} \frac{1-\cos{x}}{x^2} = \frac{1}{2} \]
  可以验证分子分母两个函数符合上述定理的条件,并且导函数在$x=0$处的极限是
  \[ \lim_{x \to 0} \frac{\sin{x}}{2x} = \frac{1}{2} \]
  因此便得结论,这个极限的思路是这样来的,因为余弦函数在$x=0$处是连续的,所以我们知道$1-\cos{x}$是一个无穷小,即
  \[ \cos{x} = 1+o(1) \]
  我们将这个无穷小与$x$这个简单的无穷小进行比较,也就是考虑
  \[ \lim_{x \to 0} \frac{1-\cos{x}}{x} \]
  根据前面证明,显然这个极限是零,也就是说$1-\cos{x}$是$x$的高阶无穷小,即
  \[ \cos{x} = 1-o(x) \]
  于是我们再将它与$x^2$进行比较,前述结果表明它俩是同阶无穷小,或者说,$1-\cos{x}$与$\frac{1}{2}x^2$是等价无穷小,于是得
  \[ \cos{x} = 1-\frac{1}{2}x^2 + o(x^2) \]
  这个过程还可以继续进行下去,只要得到一个无穷小,就将它与最简单的无穷小$x,x^2,\ldots,x^n,\ldots$依次进行比较,直到得出同阶无穷小为止,然后再反复进行此步骤,依次剥离出无穷小的主要成分,剩下的是越来越高阶的无穷小,下面就是将$1-\cos{x}$与$\frac{1}{2}x^2$作差之后与$x^3$比较,仍然由洛必达法则可得
  \[ \lim_{x \to x_0} \frac{(1-\cos{x})-\frac{1}{2}x^2}{x} = 0 \]
  说明这个差是$x$的高阶无穷小,于是将它与$x^2$进行比较,得
  \[ \lim_{x \to x_0} \frac{(1-\cos{x})-\frac{1}{2}x^2}{x^2} = \lim_{x \to 0} \frac{\sin{x}-x}{2x} = \lim_{x \to 0} \frac{\sin{x}}{2x} - \frac{1}{2} = 0 \]
  说明$x^2$的阶数仍然低了,同样的办法可以知道$x^3$仍然低,这个差与$x^4$是同阶无穷小,比值的极限是$-\frac{1}{24}$,这就是说
  \[ 1-\cos{x}-\frac{1}{2}x^2 \sim -\frac{1}{24}x^4 \]
  或者写成
  \[ \cos{x} = 1-\frac{1}{2}x^2+\frac{1}{24}x^4 + o(x^4) \]
  因为余弦函数存在任意阶的导数,所以这个过程可以无限进行下去,接下来进行几步之后看起来是这样的
  \[ \cos{x} = 1 - \frac{1}{2}x^2 + \frac{1}{24}x^4 - \frac{1}{720}x^6 + \frac{1}{40320}x^8 + o(x^8) \]
  这意味着我们可以用多项式函数逐步逼近余弦函数,同样,对于正弦函数,我们也可以用同样的办法,可以得出
  \[ \sin{x} = x - \frac{1}{6}x^3 + \frac{1}{120}x^5 - \frac{1}{5040}x^7 + o(x^7) \]
  这提供了一个非常重要的思路,就是可以用多项式函数来无限逼近一个函数,这就是后面我们推导函数的多项式展开即泰勒公式的思想。
\end{example}

\begin{example}
  依照上例,从我们已知的极限
  \[ \lim_{x \to 0} \frac{\mathrm{e}^x-1}{x} = 1 \]
  出发,可以得出
  \[ \mathrm{e}^x = 1 + x + \frac{1}{2}x^2+\frac{1}{6}x^3+\frac{1}{24}x^4+\frac{1}{120}x^5+o(x^5) \]
  同样,由
  \[ \lim_{x \to 0} \frac{\ln{(1+x)}}{x} = 1 \]
  出发,可以得出
  \[ \ln{(1+x)} = x - \frac{1}{2}x^2+\frac{1}{3}x^3+\frac{1}{4}x^4-\frac{1}{5}x^5+o(x^5) \]
\end{example}

第二个是关于$\frac{\infty}{\infty}$型的不定式.
\begin{theorem}
  设函数$f(x)$与$g(x)$都在$x_0$右侧的某个空心邻域内可导,且
  \begin{enumerate}
  \item $\lim_{x \to x_0^+} f(x) = \lim_{x \to x_0^+} g(x) = \infty$.
  \item 两个函数都在$x_0$的该空心邻域内可导,且$g'(x) \neq 0$.
  \item $\lim_{x \to x_0^+} \frac{f'(x)}{g'(x)} = A$,这里$A$可以是实数,也可以是无穷大或者带符号的无穷大.
  \end{enumerate}
\end{theorem}

\begin{proof}[证明]
  在$x_0$的右侧任意取定一个数$x_1$,设$a_n$是一个数列,它的所有项都被包含在开区间$(x_0,x_1)$上并且严格减少并收敛到$x_0$,那么两个函数在这个数列上对应着两个值的数列:$f(a_n)$和$g(a_n)$,显然这两个数列都是无穷大,并且根据柯西中值定理,存在$x_n \in (a_{n+1},a_n)$,使得
  \[ \frac{f(a_{n+1})-f(a_n)}{g(a_{n+1})-g(a_n)} = \frac{f'(x_n)}{g'(x_n)} \]
  显然当$n \to \infty$时$x_n \to x_0$,因而有
  \[ \lim_{n \to \infty} \frac{f(a_{n+1})-f(a_n)}{g(a_{n+1})-g(a_n)} = A \]
  于是由Stolz定理,可知
  \[ \lim_{n \to \infty} \frac{f(a_n)}{g(a_n)} = A \]
  然后由于数列$a_n$的任意性以及数列极限与函数极限的关系,可得
  \[ \lim_{x \to x_0^+} \frac{f(x)}{g(x)} = A \]
\end{proof}

从证明过程可以看出,这个定理与Stolz定理非常相似,Stolz定理可以看作是这个定理的离散版本,而这个定理可以看作是Stolz定理的连续版本。

\subsection{泰勒公式与麦克劳林公式}
\label{sec:taylor-formular}

在洛必达法则的例子中已经看到,反复应用洛必达法则可以得出一个函数的多项式逼近,这一节就来讨论这个问题。

假定函数$f(x)$在$x_0$处具有各阶导数(即直到接下来出现到的阶的导数),那么由连续性,$f(x)-f(x_0)$必然是一个无穷小($x \to x_0$),为了寻求这个无穷小的阶,将其与$x-x_0$这个基本无穷小进行比较,即求极限
\[ \lim_{x \to x_0} \frac{f(x)-f(x_0)}{x-x_0} \]
由洛必达法则,假定函数的一阶导数连续(尤其是二阶可导),上式的极限就是$f'(x_0)$,于是
\[ \frac{f(x)-f(x_0)}{x-x_0} = f'(x_0) + o(1) \]
或者改写为
\[ f(x) = f(x_0) + f'(x_0) (x-x_0) + o(x-x_0) \]
记
\[ T_1(x) = f(x_0)+f'(x_0)(x-x_0) \]
显然$T_1(x)$是一个一次多项式,并且与$f(x)$在$x_0$处具有相同的函数值以及一阶导数值,即
\[ f(x_0) = T_1(x_0), \  f'(x_0) = T_1'(x_0) \]

再考虑$f(x)-T_1(x)$,它是$x-x_0$的高阶无穷小,因此将它与$(x-x_0)^2$比较,即求极限
\[ \lim_{x \to x_0} \frac{f(x)-T_1(x)}{(x-x_0)^2} \]
假定$f(x)$二阶可导并且二阶导数连续,连续两次应用洛必达法则,得
\[ \lim_{x \to x_0} \frac{f(x)-T_1(x)}{(x-x_0)^2} = \lim_{x \to x_0} \frac{f'(x)-T_1'(x)}{2(x-x_0)} = \lim_{x \to x_0} \frac{f''(x)}{2} = \frac{1}{2} f''(x_0) \]
于是
\[ \frac{f(x)-T_1(x)}{(x-x_0)^2} = \frac{1}{2}f''(x_0) + o(1) \]
或者写成
\[ f(x) = f(x_0) + f'(x_0)(x-x_0)+\frac{1}{2}f''(x_0)(x-x_0)^2 + o((x-x_0)^2) \]
并记
\[ T_2(x) =  f(x_0) + f'(x_0)(x-x_0)+\frac{1}{2}f''(x_0)(x-x_0)^2 \]
则$T_2(x)$与$f(x)$在$x_0$处具有相同的函数值以及一阶导数和二阶导数。

同样,假定函数存在连续的三阶导数,那么还有
\[ f(x) = f(x_0) + f'(x_0)(x-x_0)+\frac{f''(x_0)}{2!}(x-x_0)^2+\frac{f'''(x_0)}{3!}(x-x_0)^3 + o((x-x_0)^3) \]
如果还具有连续的四阶导数,则还有
\[ f(x) = f(x_0) + f'(x_0)(x-x_0)+\frac{f''(x_0)}{2!}(x-x_0)^2+\frac{f'''(x_0)}{3!}(x-x_0)^3+\frac{f^{(4)}(x_0)}{4!}(x-x_0)^4 + o((x-x_0)^4) \]
照此下去,如果函数$f(x)$在$x_0$的邻域上存在直到$n$阶的连续导函数,那么
\[ f(x) = f(x_0) + f'(x_0)(x-x_0)+\frac{f''(x_0)}{2!}(x-x_0)^2+ \cdots +\frac{f^{(n)}(x_0)}{n!}x^n + o((x-x_0)^n) \]
记
\[ T_n(x) = f(x_0) + f'(x_0)(x-x_0)+\frac{f''(x_0)}{2!}(x-x_0)^2+ \cdots +\frac{f^{(n)}(x_0)}{n!}x^n \]
这个多项式就称为函数$f(x)$在$x_0$处的泰勒多项式,它的主要特征是,它与$f(x)$在$x_0$处具有相同的函数值以及一阶导数、二阶导数直到$n$阶导数(函数值也可以被视为零阶导数),即
\[ f(x_0)=T_n(x_0), \ f'(x_0) = T_n'(x_0), \  f''(x_0) = T_n''(x_0), \ \ldots, \  f^{(n)}(x_0) = T_n^{(n)}(x_0) \]
而函数的展开式则可以写成
\[ f(x) = T_n(x) + o((x-x_0)^n) \]
称为函数$f(x)$在$x_0$处的\emph{泰勒展开公式}.

于是我们得到如下的著名定理
\begin{theorem}[泰勒(Taylor)定理]
  如果函数$f(x)$在$x_0$的邻域内有直到$n$阶的导函数,作泰勒多项式
\[ T_n(x) = f(x_0) + f'(x_0)(x-x_0)+\frac{f''(x_0)}{2!}(x-x_0)^2+ \cdots +\frac{f^{(n)}(x_0)}{n!}x^n \]
则有
\[ f(x_0)=T_n(x_0), \ f'(x_0) = T_n'(x_0), \  f''(x_0) = T_n''(x_0), \ \ldots, \  f^{(n)}(x_0) = T_n^{(n)}(x_0) \]
并且有下式成立
\[ f(x) = T_n(x) + o((x-x_0)^n) \]
这就是($n$阶)\emph{泰勒(Taylor)公式},其中的余项$o((x-x_0)^n)$称为\emph{佩亚诺余项}.
\end{theorem}

我们用数学归纳法来证明.
\begin{proof}[证明]
  我们记
  \[ R(x) = f(x) - T_n(x) \]
  显然有
  \[ R(x_0) = R'(x_0) = \cdots = R^{(n)}(x_0) = 0 \]
  因此只要证明在这个条件下有
  \[ R(x) = o((x-x_0)^n) \]
  就可以了。
  
  当$n=1$时,由导数意义,有
  \[ \frac{R(x)-R(x_0)}{x-x_0} = R'(x_0) + o(1) \]
  注意到$R(x_0)=R'(x_0)=0$,所以
  \[ R(x) =  o(x-x_0) \]
  结论成立,假定定理对于$n$也是成立的,则看$n+1$的情形,此时设
  \[ S(x) = R'(x) \]
  就有
  \[ S(x_0) = S'(x_0) = \cdots = S^{(n)}(x_0) = 0 \]
  因此由归纳假设,有
  \[ S(x) = o((x-x_0)^n) \]
  再由洛必达法则,有
  \[ \lim_{x \to x_0} \frac{R(x)}{(x-x_0)^{n+1}} = \lim_{x \to x_0} \frac{S(x)}{(n+1)(x-x_0)^n} = 0 \]
  即
  \[ R(x) = o((x-x_0)^{n+1}) \]
  由归纳法,定理成立。
\end{proof}

注意到,在前面的推导过程中,我们要求函数具有直到$n$阶的连续导函数,但在这个定理的证明过程中,第$n$阶导函数是否连续并没有用到,这是为什么呢,用前面推导到二阶展开式那里的情况说明,那时有
\[ \lim_{x \to x_0} \frac{f(x)-T_1(x)}{(x-x_0)^2} = \lim_{x \to x_0} \frac{f'(x)-T_1'(x)}{2(x-x_0)} = \lim_{x \to x_0} \frac{f''(x)}{2} = \frac{1}{2} f''(x_0) \]
最后一步,求极限$\lim_{x \to x_0} f''(x_0)$,那里我们是假定二阶导函数连续,得出这个极限为$f''(x_0)$的,但实际上由导函数连续性定理可知,如果这个极限存在,设为$K$,那么函数在$x_0$处必然存在二阶导数,且$f''(x_0)=K$,所以我们不要求二阶导数在$x_0$处连续,仅要求$f''(x_0)$存在就行,只要存在就必然连续,所以定理中的条件就减弱为在$x_0$的邻域上存在直到$n$阶的导函数。

在泰勒公式中,取$x_0 = 0$,我们就得到如下的\emph{麦克劳克(Maclaurin)公式}:
\[ f(x) = f(0) + f'(0)x + \frac{f''(0)}{2!}x^2 + \cdots + \frac{f^{(n)}(0)}{n!}x^n + o(x^n) \]

\begin{example}
  既然公式是函数的多项式展开,那么多项式函数自己的展开式会是什么样呢? 设多项式函数
  \[ f(x) = a_nx^n+a_{n-1}x^{n-1}+\cdots+a_1x+a_0 \]
  先写出它的$n$阶泰勒多项式
\[ T_n(x) = f(x_0) + f'(x_0)(x-x_0)+\frac{f''(x_0)}{2!}(x-x_0)^2+ \cdots +\frac{f^{(n)}(x_0)}{n!}x^n \]
可以发现,它也是一个$n$次多项式,按泰勒定理,这两个多项式相差一个$(x-x_0)^n$的高阶无穷小,我们指出,对于$n$次多项式函数$f(x)$而言,它与它的$n$阶泰勒展开式的差值是零,也就是
\[ f(x) = T_n(x) \]
我们利用归纳法证明下面的结论,有了下面这个结论,这里的结论也就成立了。

\begin{statement}
  如果两个次数不超过$n$次的多项式$f(x)$与$g(x)$在$x_0$处直到$n$阶的导数值都相同,即
  \[ f'(x_0) = g'(x_0), \  f''(x_0) = g''(x_0), \  \ldots, \  f^{(n)}(x_0) = g^{(n)}(x_0) \]
  则这两个多项式仅相差一个常数,即$f(x)-g(x)=C$,这里$C$是一个固定的常数.
\end{statement}

显然结论对于常数多项式和一次多项式均成立,假设结论对于$\leqslant n$的情况都成立,现在假定$f(x)$和$g(x)$都是$n+1$次多项式,并且在$x_0$处两者具有相同的直到$n+1$阶的导数,那么它们的导数$f'(x)$和$g'(x)$在$x_0$处就具有相同的直到$n$阶导数,根据归纳假设,存在常数$C_1$,使得$f'(x)-g'(x)=C_1$,但由于$f'(x_0)=g'(x_0)$,所以$C_1=0$,即$f'(x)=g'(x)$,于是对于多项式$h(x)=f(x)-g(x)$有$h'(x)=0$,于是$h(x)$只能是常数多项式,即$h(x)=f(x)-g(x)=C$.
\end{example}

有了这结论之后,由于$n$次多项式$f(x)$与它的$n$阶泰勒多项式在$x_0$具有相同的直到$n$阶的导数,因此它俩仅相差一个常数,但又由于$f(x_0)=T_n(x_0)$,所以有$f(x)=T_n(x)$.

\subsection{基本函数的泰勒展式}
\label{sec:taylor-expand-for-fundation-function}

在这一小节,我们来求一些基本函数在$x=0$处的泰勒展开式,即麦克劳林展开式。

1. 幂函数$f(x)=x^p(x>0)$.

由于它已经是在$x=0$处的展开式了,所以我们考虑它在$x=1$处的泰勒展开,在作代换$t=x-1$之后,实际就是要求函数$f(t)=(1+t)^p$在$t=0$处的展开式,我们还是用$x$来替换$t$,那么根据前面已经求得的高阶导数值,有
\[ (1+x)^p = 1 + px + \frac{p(p-1)}{2!}x^2 + \cdots + \frac{p(p-1)\cdots (p-n+1)}{n!}x^n + o(x^n) \]
显然,如果$p$是正整数,由于$n$次多项式的$n$阶导数成为常数,更高阶的导数则恒零,因此这个展开式就成为二项式定理了
\[ (1+x)^n = 1+nx + \frac{n(n-1)}{2!}x^2 + \frac{n(n-1)(n-2)}{3!}x^3 + \cdots + x^n \]
因此这式子可以看作是二项式定理的推广,它被称为\emph{牛顿二项式定理}.

在式中令$p=-1$,并将$x$替换为$-x$,得
\[ \frac{1}{1-x} = 1+x+x^2 + \cdots + x^n + o(x^n) \]
同样,取$p=\frac{1}{2}$,得
\[ \sqrt{1+x} = 1+\frac{1}{2}x-\frac{1}{8}x^2+\frac{3}{48}x^3 + o(x^3) \]
取$p=-\frac{1}{2}$,得
\[ \frac{1}{\sqrt{1+x}} = 1-\frac{1}{2}x+\frac{3}{8}x^2 -\frac{15}{48}x^3 + o(x^3) \]

1. 指数函数$f(x)=\mathrm{e}^x$.

根据在高阶导数那一小节我们求得的指数函数的高阶导数公式,我们得出
\[ \mathrm{e}^x = 1+x+\frac{1}{2!}x^2 + \cdots + \frac{1}{n!}x^n + o(x^n) \]

2. 对数函数$f(x)=\ln{x}$.

由于对数函数在$x=0$处无定义,所以我们考虑它在$x=1$处的展开式,实际上就是求$\ln{(1+x)}$在$x=0$处的展开式,这时有
\[ \ln{(1+x)} = x - \frac{1}{2} x^2 + \frac{1}{3} x^3 - \cdots + \frac{(-1)^{n-1}}{n}x^n + o(x^n) \]

3. 三角函数

对正弦函数,有
\[ \sin{x} = x - \frac{1}{3!}x^3 + \frac{1}{5!} x^5 - \cdots + \frac{(-1)^n}{(2n+1)!}x^{2n+1} + o(x^{2n+1}) \]
对于余弦函数,则是
\[ \cos{x} = 1 - \frac{1}{2!}x^2 + \frac{1}{4!}x^4 - \cdots + \frac{(-1)^n}{(2n)!}x^{2n} + o(x^{2n}) \]

\subsection{余项的其它形式}
\label{sec:other-format-of-taylor-additional}



\subsection{插值公式}
\label{sec:interpolation-formula}


\subsection{微分方程}
\label{sec:differtial-equation}

我们知道指数函数$y=f(x)=\mathrm{e}^x$的导函数就是它自己,用微分的形式写出来就是
\[ \frac{\dif f(x)}{\dif x} = f(x) \]
或者简单写成
\[ \frac{\dif y}{\dif x} = y \]
或者更加简洁的写成
\[ y'=y \]
像这样把函数自身与其微分(或者导数)联系起来的方程,就称为\emph{微分方程}。

上面的例子就表明指数函数$y=\mathrm{e}^x$就满足微分方程
\[ \frac{\dif y}{\dif x} = y \]
我们还可以写出一些微分方程
\begin{align*}
  y''+y'+y&=0 \\
  y'^2+x^2&=0 \\
  y''+2y'y+y^2& = 0
\end{align*}

如果一个函数与它的导函数满足一个给定的微分方程,就称这函数是这微分方程的一个\emph{解}.例如,函数$y=\mathrm{e}^x$就是微分方程$y'=y$的一个解,但并不是唯一解,因为零函数也满足这方程。

\begin{example}
  通过求导可以验证,函数
  \[ y=C_1 \cos{x} + C_2 \sin{x} \]
  满足微分方程
  \[ y''+y=0 \]
  这里$C_1$和$C_2$是常数。
\end{example}

\begin{example}
  函数
  \[ y= C_1 \cosh{x} + C_2 \sinh{x} \]
  满足微分方程
  \[ y''-y=0 \]
\end{example}

\begin{example}
  可以验证,函数
  \[ y = C_1 \mathrm{e}^{\lambda_1 x} + C_2 \mathrm{e}^{\lambda_2 x} \]
  满足微分方程
  \[ y''-(\lambda_1+\lambda_2)y' + \lambda_1 \lambda_2 y = 0 \]
  像这样由$y$、$y'$、$y''$等各阶导数的线性组合(系数可以是常数或者自变量$x$的表达式)而成的微分方程,称为\emph{线性微分方程},如果各项的系数又是常数,则称为\emph{常系数线性微分方程}。
\end{example}

\begin{example}
  在这个例子中,我们来解下面这个微分方程
  \[ y'-\lambda y = 0 \]
  作函数$u=y\mathrm{e}^{-\lambda x}$,那么有
  \[ u'=\mathrm{e}^{-\lambda x} (y'-\lambda y) = 0 \]
  根据前面的结果,函数$u$只能是常数函数,即
  \[ y\mathrm{e}^{-\lambda x} = C \]
  从而
  \[ y = C \mathrm{e}^{\lambda x} \]
  这样就解出了这个微分方程.
\end{example}




%%% Local Variables:
%%% mode: latex
%%% TeX-master: "../calculus-note"
%%% End:
