
\section{利用导数研究函数的性质}
\label{sec:research-function-use-derivative}

\subsection{单调性与极值}
\label{sec:research-monotonicity-and-minmax-value}

导数可以用来研究函数的单调性和极值,我们先建立如下关于单调性的定理
\begin{theorem}
  如果函数$f(x)$的一阶导函数$f'(x)$在某个区间上恒满足$f'(x) \geqslant 0$,那么函数在此区间上单调不减,如果不等式反向,即$f'(x) \leqslant 0$恒成立,那么函数在此区间上单调不增。如果不等式中的等号总不成立或者至多仅在有限个点处成立,那么函数相应的是严格单调的。
\end{theorem}

\begin{proof}[证明]
  如果函数在某区间上恒成立$f'(x) \geqslant 0$,那么对于区间上任意两个不同的数$x_1$与$x_2$,设$x_1<x_2$,按照拉格朗日中值定理,存在$\xi \in (x_1,x_2)$,使得
  \[ f'(\xi) = \frac{f(x_1)-f(x_2)}{x_1-x_2} \]
  由$f'(\xi) \geqslant 0$即得$f(x_1) \leqslant f(x_2)$,于是函数单调不减。函数单调不增的证明也是完全类似的。

  显然,如果上面的证明中不等式$f'(x) \geqslant 0$中的等号永远不成立,即永远只能取大于号,那么相应的得出的结论是$f(x_1)<f(x_2)$,即是严格递增的。

  现在来证明,如果导函数$f'(x) \geqslant 0$恒成立,但等号仅在有限个点处取得,那么函数仍然是严格增加的。

  首先可以肯定函数是单调不减的了,把区间划分成多个小区间,使得每个小区间上至多只有一个导函数零点,如果函数在每个小区间上是严格增加的,而连续性又保证了端点处的接续,那么就能得出函数在整个区间上都是严格增加的,所以只需要证明导函数只在一个点处取零点的情形就可以了。

  假定导函数$f'(x)$仅在区间$[a,b]$内的某一点$c$处取零值,采用反证明法来证明函数是严格增加的,若不然,假如存在$x_1,x_2 \in [a,b]$使得$x_1<x_2$且$f(x_1)=f(x_2)$,那么$f'(x_1)$与$f'(x_2)$中至多只有一个为零,而另一个必为正值,不妨设就是$f‘(x_1)>0$,于是存在$x_1$的某个右邻域$(x_1,x_1+\delta)$,在此右邻域上恒有$f(x)>f(x_1)=f(x_2)$,只要把$\delta$限制的充分小,就能保证$x_1+\delta<x_2$,于是函数在$(x_1,x_1+\delta)$上的函数值就都大于$f(x_2)$,这与函数的单调不减是矛盾的,从而得证。
\end{proof}

\begin{example}
  关于定理中导函数恒非负,仅在有限个点处取零值也能保证函数的严格增加这一点,函数$y=x^3$提供了一个例子,它的导函数$y'=3x^2 \geqslant 0$,但仅在$x=0$处取零值,并不影响原来函数是严格增加的这一事实。
\end{example}

需要说明的是,对于函数单调不减与单调不增来讲,条件$f'(x) \geqslant 0$或者$f'(x) \leqslant 0$是充分必要条件(在函数可导的前提下),但对于严格单调来说,不等式成立且至多仅在有限个点处取等号,则只是充分条件而非必要条件,我们会举反例加以说明。

关于极值,费马极值定理已经表明,极值点处如果可导,则导数只能是零,但是如何判断导函数的零点是否是原来函数的极值点呢,有如下定理
\begin{theorem}
  设函数$f(x)$在$x_0$的邻域内可导,如果它满足以下两条,那么函数在$x_0$处取极值.
  \begin{enumerate}
  \item $f'(x_0)=0$.
  \item 存在$x_0$的某个足够小的邻域,使得函数在两侧空心邻域内各自保持恒定的符号,且两侧的符号正好相反,具体的说,如果左正右负,则函数在$x_0$处取极大值,反之,若左负右正,则函数在$x_0$处取极小值。
  \end{enumerate}
\end{theorem}

\begin{proof}[证明]
  
\end{proof}

但要注意,这个定理中的条件是充分条件,但不是必要条件。

\begin{example}[光的折射定律]
  假定光在甲、乙两种介质中的传播速度分别是$v_1$和$v_2$,如图所示,图中直线是两种介质的分界面,现在光从介质甲中的$A$点发出,经分界面折射后,经过介质乙中的$B$点,光在同一种介质中是必定沿直线传播的,我们知道,光总是按照传播用时最短的路径前进,那么问题就来了,光线应该在分界面上何处折射,才能使得传播用时最短?

  设$A$、$B$两点与分界面的距离分别记为$a$、$b$,并且沿分界面的距离是$l$,假定光线到达分界面上的点$P$处,点$P$沿分界面与$A$、$B$的距离分别是$x$和$l-x$,那么光线传播所用的时间是
  \[ f(x) = \frac{\sqrt{x^2+a^2}}{v_1}+\frac{\sqrt{(l-x)^2+b^2}}{v_2} \]
  为了求得最小值,求导并令其为零,得
  \[ f'(x) = \frac{x}{v_1\sqrt{x^2+a^2}} - \frac{l-x}{v_2 \sqrt{(l-x)^2+b^2}} \]
  根据物理意义,这最小值一定存在,设当$x=x_0$时用时最短,则必有$f'(x_0)=0$,于是有
  \[ \frac{x_0}{v_1\sqrt{x_0^2+a^2}} - \frac{l-x_0}{v_2 \sqrt{(l-x_0)^2+b^2}}\]
  设入射角为$\alpha$,折射角为$\beta$,上式即为
  \[ \frac{\sin{\alpha}}{v_1} = \frac{\sin{\beta}}{v_2} \]
  通常用入射角和折射角来标记点$P$的位置,而不是用$x_0$,上式就是光线传播最短路径所应满足的条件,其被称为光的折射定律。
\end{example}

%%% Local Variables:
%%% mode: latex
%%% TeX-master: "../calculus-note"
%%% End:
