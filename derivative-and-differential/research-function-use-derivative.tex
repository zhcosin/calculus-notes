
\section{利用导数研究函数的性质}
\label{sec:research-function-use-derivative}

\subsection{单调性与极值}
\label{sec:research-monotonicity-and-minmax-value}

\begin{example}[光的折射定律]
  假定光在甲、乙两种介质中的传播速度分别是$v_1$和$v_2$,如图所示,图中直线是两种介质的分界面,现在光从介质甲中的$A$点发出,经分界面折射后,经过介质乙中的$B$点,光在同一种介质中是必定沿直线传播的,我们知道,光总是按照传播用时最短的路径前进,那么问题就来了,光线应该在分界面上何处折射,才能使得传播用时最短?

  设$A$、$B$两点与分界面的距离分别记为$a$、$b$,并且沿分界面的距离是$l$,假定光线到达分界面上的点$P$处,点$P$沿分界面与$A$、$B$的距离分别是$x$和$l-x$,那么光线传播所用的时间是
  \[ f(x) = \frac{\sqrt{x^2+a^2}}{v_1}+\frac{\sqrt{(l-x)^2+b^2}}{v_2} \]
  为了求得最小值,求导并令其为零,得
  \[ f'(x) = \frac{x}{v_1\sqrt{x^2+a^2}} - \frac{l-x}{v_2 \sqrt{(l-x)^2+b^2}} \]
  根据物理意义,这最小值一定存在,设当$x=x_0$时用时最短,则必有$f'(x_0)=0$,于是有
  \[ \frac{x_0}{v_1\sqrt{x_0^2+a^2}} - \frac{l-x_0}{v_2 \sqrt{(l-x_0)^2+b^2}}\]
  设入射角为$\alpha$,折射角为$\beta$,上式即为
  \[ \frac{\sin{\alpha}}{v_1} = \frac{\sin{\beta}}{v_2} \]
  通常用入射角和折射角来标记点$P$的位置,而不是用$x_0$,上式就是光线传播最短路径所应满足的条件,其被称为光的折射定律。
\end{example}

%%% Local Variables:
%%% mode: latex
%%% TeX-master: "../calculus-note"
%%% End:
