
\section{数列极限}
\label{sec:limit-of-number-sequence}

\subsection{数列极限的概念}
\label{sec:concept-of-limit-for-number-sequence}

在中学数学中,我们熟知反比例函数$f(x)=1/x$的图象在向无穷远处延伸时,它会无限的向横轴靠近,当$x$取正值并无限的增大时,其第一象限的一支会无限的向$x$轴正半轴靠近,但无论$x$取多大,因为$1/x>0$,所以它始终不会与$x$轴相交,这给了我们一种“无限接近但是不会相等”的直观感受。

同样的情况还有许多,我们就准备来详细的讨论下这种“无限接近又不相等”的现象。

上面反比例函数的例子是针对函数而言的,我们先从较为简单的数列开始,同样可以得到数列$x_n=1/n$这个数列,在$n$取正整数并无限增大时,数列的值无限的接近零,但却总是大于零,我们来从这种现象中提取 \emph{极限} 的概念。

首先要指出的是这里“无限接近但不等于”中的“不等于”其实是无关紧要的,例如在数列$1/n$中把下标为偶数的项全部换成零,那么这个"无限"接近并没有被破坏,而且它仍然给我们以极限的印象,只是它在下标增大的过程中,可以无限次的取极限值,而且从这个例子中还可得知,数列的单调性也不是必要的。

我们先给出一个初步的定义: 如果数列$x_n$在随着$n$的无限增大过程中可以无限的接近一个常数$A$,则称$A$是这数列当$n$趋于无穷大时的极限。

这个定义不会令人满意,因为作为数学上的一个定义,它需要具备精确性,而这个定义中出现了“无限接近”这样含义模糊不清的描述,利用这个定义,我们很难说明一个给定常数是否是一个数列的极限,我们需要将它严格化。

所谓数列$x_n$“无限接近”于常数$A$,自然指的是差值$|x_n-A|$可以任意的小,所以我们进行第一步严格化:把数列$x_n$无限接近常数$A$严格化成差值$|x_n-A|$可以任意小,于是极限的定义可以重新叙述为: 对于数列$x_n$和常数$A$,如果数列$x_n$当$n$无限增大时差值$|x_n-A|$可以任意的小,则称常数$A$是数列$x_n$当$n$趋向于无穷大时的极限。

然后我们考虑如何刻画“可以任意的小”,那就是说,差值$|x_n-A|$可以小于任意的正实数$\varepsilon$,而不管这正实数$\varepsilon$有多小。初看起来,“可以小于任意的正实数”,似乎只要存在正整数$n$,使得$|x_n-A|<\varepsilon$就可以了,也就是如下的极限定义: 对于数列$x_n$和常数$A$,如果对于无论多么小的正实数$\varepsilon$,总存在正整数$N$,使得$|x_N-A|<\varepsilon$成立,则称常数$A$是数列$x_n$在下标趋于无穷大时的极限。

这个定义看上去似乎非常符合$1/n$这个数列的特征,不管多么小的正实数$\varepsilon$,总能找到使$1/n < \varepsilon$成立的$n$,只要$n$取的足够大。然而这个定义却有一个严重的问题,我们把数列$1/n$中下标为偶数的项全部换成1,所得到的新数列显然不应该有极限,因为它的奇数下标项趋于零而偶数下标项恒为1,按我们的直观感受,它的值并不无限靠近零,也不无限靠近1,0和1都不应该是它的极限,但是按照上面的定义,它却是符合条件的!

问题出在哪呢,仍然以这个把偶数下标项都替换为1的新数列为例,显然,存在正整数$N$使得$|x_n-A|<\varepsilon$这个条件,是无法保证数列的全部项都向常数$A$靠近的,它只能保证数列中有一部分项会向常数$A$靠近,刚才这个例子也说明了这一点,所以我们需要一个更强的能保证数列的所有项都要向常数$A$靠近,我们把存在正整数$N$使得$|x_N-A|<\varepsilon$成立,改为存在正整数$N$,使得$n>N$时$|x_n-A|<\varepsilon$恒成立,这样一来这个新数列就不满足这条件了,而原来的数列$1/n$却满足这条件。

这个新的条件,利用$n>N$时$|x_n-A|<\varepsilon$恒成立,来保证了数列向$A$靠近的总体趋势。这就是我们最终的极限定义,这个定义,从模糊到精确,别看在这几段话就给出了,实际上在历史上经过了几代数学家的努力,最后才由德国数学家魏尔斯特拉斯(Weierstrass, 1815.10.31-1897.2.19)在总结前人成果的基础上给出,这个精确定义,是人类智慧的结晶。

\begin{definition}[数列极限]
  对于实数数列${a_n}$和实数$a$,如果对于任意小的正实数$\varepsilon$,都存在某一下标$N$,使得该数列在这之后的所有项(即$n>N$)都满足
  \begin{equation}
    \label{eq:the-definition-of-sequence-limit}
    |a_n-a|<\varepsilon
  \end{equation}
  则称该数列存在极限,实数$a$称为该数列的极限。也称该数列为收敛数列,并且收敛到实数$a$,记为
  \begin{equation}
    \label{eq:limit-definition-for-number-sequence}
    \lim_{n \to \infty}x_n = a
  \end{equation}
\end{definition}

极限为零的数列称为无穷小数列,简称\emph{无穷小}。如果数列不存在有限的极限,称为数列为\emph{发散数列}。

如果数列无论对于多大的实数$M>0$,总能从某项开始,后续的全部项都有$a_n>M$,则称数列为\emph{正无穷大}。类似的也有负无穷大和(绝对值)无穷大的概念。

我们引入邻域的概念:
\begin{definition}[邻域]
以实数$x$为中心半径为$\varepsilon$的开区间$(x-\varepsilon,x+\varepsilon)$称为实数$x$的$\varepsilon-$\emph{邻域},记作$U(x,\varepsilon)$,若从中去掉实数$x$,即$(x-\varepsilon,x)\cup(x,x+\varepsilon)$称为$x$的$\varepsilon-$\emph{空心邻域},记作$\mathring{U}(x,\varepsilon)$,有时并不关心半径是多少,直接称为邻域和空心邻域,简记为$U(x)$和$\mathring{U}(x)$.
\end{definition}

借助邻域概念,我们可以这样刻画数列的极限:数列收敛到$A$,就是说,对于$A$的任意小的邻域$U(A)$,下标超过某一值以后,都将全部落在该邻域中,从而只有有限项落在该邻域外部。

\subsection{一些极限的例}
\label{sec:some-examples-about-limit-of-number-sequence}

\begin{example}
  \label{example:limit-of-1-devide-by-n-power-p}
  前面在提炼极限定义时用了$\dfrac{1}{n}$在$n$无限增大时的情形,现在利用极限定义来证明这个极限:
  \[ \lim_{n \to \infty} \frac{1}{n} = 0 \]
  对于任意小的正实数$\varepsilon$,要使$\dfrac{1}{n}<\varepsilon$,只要$n>\dfrac{1}{\varepsilon}$就可以了,于是可以取$N=1+\left[ \dfrac{1}{\varepsilon} \right]$就可以了,这就证明了它的极限是零,实际上这里的$N$,只要比$\dfrac{1}{\varepsilon}$大就都可以了,以后我们就不专门作取整处理了。

  仿此还可以得到,当$p$是正有理数时,有
  \[ \lim_{n \to \infty} \frac{1}{n^p} = 0 \]
\end{example}

\begin{example}
  \label{example:limit-of-q-power-n}
  设实数$q$满足$|q|<1$,则
  \[ \lim_{n \to \infty} q^n = 0 \]
  对于任意小的正实数$\varepsilon$,要使不等式$|q^n - 0| < \varepsilon$,只要$n>\dfrac{\ln{\varepsilon}}{\ln{|q|}}$就行了,这就证明得了结论。
\end{example}

\begin{example}
  \label{example:limit-of-1-devide-by-a-power-n}
  设实数$a>1$,则
  \[ \lim_{n \to \infty} \frac{1}{a^n} = 0 \]
  这实际上是\autoref{example:limit-of-q-power-n}的特例,但这里换一种方法来证明它.
  
  对于无论多么小的正实数$\varepsilon$,为了找到极限定义中所要求的$N$,考虑不等式
  \[ \frac{1}{a^n} < \varepsilon \]
  也就是$a^n>1/\varepsilon$,设$a=1+\lambda$,则$\lambda>0$,按二项式定理有\footnote{我们这里并没有从$a^n>1/\varepsilon$中直接使用对数来得出$n>\log_a{(1/\varepsilon)}$,这是因为尽管中学数学中已经学过对数概念,但那时还没有给出无理指数幂的定义,所以指数的定义是不完整的,因此我们无法确认,对于底数$a$,正实数$1/\varepsilon$的对数是否存在,以后我们将在\autoref{sec:irrational-power}中专门讨论指数的定义和值域问题。}
  \[ a^n = (1+\lambda)^n = 1 + n\lambda + \frac{n(n-1)}{2!}\lambda^2+\cdots+\lambda^n > 1+n \lambda \]
  所以只要$1+n\lambda>1/\varepsilon$,便能保证$a^n>1/\varepsilon$成立,也就是只需要$n > (1/\varepsilon-1) / (a-1)$就行了,所以只要选择$N>(1/\varepsilon-1)/(a-1)$就行了,这就证得了此极限。
\end{example}

\begin{example}
  \label{example:limit-of-n-devide-by-a-power-n}
  我们来建立一个基本的极限,对于大于1的正实数$a$,有
  \[ \lim_{n \to \infty} \frac{n}{a^n} = 0 \]
  事实上,仍同\autoref{example:limit-of-1-devide-by-a-power-n}一样,记$a=1+\lambda(\lambda>0)$,则
  \[ a^n=(1+\lambda)^n=\sum_{i=0}^nC_n^i\lambda^i > C_n^2 \lambda^2 \]
  因此有
  \[ \frac{n}{a^n} < \frac{n}{C_n^2 \lambda^2} = \frac{2}{(n-1)\lambda^2} \]
  在$n>2$时,又有$n-1>\dfrac{n}{2}$,所以此时更有
  \[ \frac{n}{a^n} < \frac{4}{n\lambda^2} \]
  所以对于任意小的正实数$\varepsilon$,只要$n>\max\{2,\dfrac{4}{\lambda^2\varepsilon}\}$,便有$\dfrac{n}{a^n} < \varepsilon$,这就证得所要的极限。
\end{example}

\begin{example}
  \label{example:limit-of-lnn-devide-by-n}
  在\autoref{example:limit-of-n-devide-by-a-power-n}的基础上,还可以建立下面的极限
  \[ \lim_{n \to \infty} \frac{\ln{n}}{n} = 0 \]
  对于任意小的正实数$\varepsilon$,为了使得$n$充分大时$\dfrac{\ln{n}}{n} < \varepsilon$,只要使$n<(e^{\varepsilon})^n$就可以了,而根据\autoref{example:limit-of-n-devide-by-a-power-n}中所建立起的极限,这是可以在$n$充分大时恒成立的,故此就建立了此处的极限。
\end{example}

\begin{example}
  \label{example:limit-of-a-power-n-devide-by-n-fraction}
  设实数$a>1$,则
  \[ \lim_{n \to \infty} \frac{a^n}{n!} = 0 \]
  将原式写成
  \[ a \cdot \frac{a}{2} \cdots \frac{a}{n} \]
  任取正整数$M$满足$M>a$,根据上式因式中分母与$M$的大小分成两部分(限定$n>M$)
  \[ \left( a \cdot \frac{a}{2} \cdots \frac{a}{M} \right) \left( \frac{a}{M+1} \cdot \frac{a}{M+2} \cdots \frac{a}{n} \right) \]
  将后一个括号中的因式全部放大为$\dfrac{a}{M}$,成为
  \[ \left( a \cdot \frac{a}{2} \cdots \frac{a}{M} \right) \left( \frac{a}{M} \right)^{n-M} \]
  它具有形式
  \[ c q^n \]
  其中$q=\dfrac{a}{M}$满足$|q|<1$,而$c$是一个正常数,所以此式极限是零,而原式也就以零为极限。
\end{example}

\begin{example}
  \label{example:limit-of-n-sqrt-a-when-a-greater-than-1}
  设实数$a>1$且$a \neq 1$,则 $\lim_{n \to \infty} \sqrt[n]{a} = 1$

  \begin{proof}[证明一]
    利用乘法公式$x^n-1=(x-1)(x^{n-1}+x^{n-2}+\cdots+1)$可得
    \[ \sqrt[n]{a}-1 = \frac{a-1}{(\sqrt[n]{a})^{n-1}+(\sqrt[n]{a})^{n-2}+\cdots+1} < \frac{1}{n}(a-1) \]
   于是对于任意正实数$\varepsilon$,只要取$N>\frac{a-1}{\varepsilon}$便能保证$n>N$时有$0<\sqrt[n]{a}-1<\varepsilon$,所以这极限得证。
  \end{proof}

  \begin{proof}[证明二]
    设$z_n=\sqrt[n]{a}-1$,则
    \[ a = (1+z_n)^n = 1+ nz_n+\frac{n(n-1)}{2!}z_n^2+\cdots+z_n^n > 1+ n z_n \]
    所以得到
    \[ 0<z_n<\frac{1}{n}(a-1) \]
    下同证明一.
  \end{proof}
\end{example}

\begin{example}
  我们来证明下面这个极限
  \[ \lim_{n \to \infty} \sqrt[n]{n} = 1 \]
  显然在$n>1$时有$\sqrt[n]{n}>1$,只要证明当$n$大于某一正整数时恒有
  \[ \sqrt[n]{n}<1+\varepsilon \]
  就可以了,这里$\varepsilon$是一个任意小的正实数,而这只需要
  \[ n < (1+\varepsilon)^n \]
  就行了,根据\autoref{example:limit-of-n-devide-by-a-power-n}中所得到的极限,这显然是可以办到的。
\end{example}

\begin{example}
  现在证明
  \[ \lim_{n \to \infty} \frac{1}{\sqrt[n]{n!}} = 0 \]
  任意固定一个正整数$M$,在$n>M$时显然有
  \[ n!>M!M^{n-M} \]
  即
  \[ \frac{1}{\sqrt[n]{n!}} < \frac{1}{M \sqrt[n]{\frac{M!}{M^M}}} \]
  而$\dfrac{M!}{M^M}<1$,根据\autoref{example:limit-of-n-sqrt-a-when-a-greater-than-1}的结论,当$n$充分大时,将有
  \[ \sqrt[n]{\frac{M!}{M^M}} > \frac{1}{2} \]
  从而此时有
  \[ \frac{1}{\sqrt[n]{n!}} < \frac{2}{M} \]
  即对于任何确定的$M$,当$n$充分大时都有上式成立,所以对于任意小的正实数$\varepsilon$,只要让$\dfrac{2}{M}<\varepsilon$就可以了,这就证明了结论。
\end{example}

\begin{example}
  \label{example:mean-value-of-converge-number-sequence}
  假定数列$a_n$以$A$为极限,我们来考察一下它的前$n$项的算术平均序列
  \[ M_n=\frac{x_1+x_2+\cdots+x_n}{n} \]
  作为一个数列的收敛情况。

  先可以直观的想象一下,数列以$A$为极限,那么数列的项随着下标的无限增大将与$A$无限接近,除前面的项外,越往后的连续片段的算术平均也应该与$A$无限接近,而是越着$n$的无限增大,数列最前面的与$A$相差的项对算术平均的影响也将越来越小,因此猜想$M_n$也收敛到$A$.

  事实上也的确如此,因为对于任意正实数$\varepsilon$,存在正整数$N$,使得$n>N$时恒有$|a_n-A|<\varepsilon$,即$A-\varepsilon<a_n<A+\varepsilon$,于是有
  \[ (n-N)(A-\varepsilon) < a_{N+1}+a_{N+2}+\cdots+a_n < (n-N)(A+\varepsilon) \]
  所以
  \[ \sum_{i=1}^Na_i + (n-N)(A-\varepsilon) < a_{1}+a_{2}+\cdots+a_n < \sum_{i=1}^Na_i + (n-N)(A+\varepsilon) \]
  从而
  \[ A-\varepsilon + \frac{1}{n} \left( \sum_{i=1}^Na_i - N(A-\varepsilon) \right) < M_n <  A+\varepsilon + \frac{1}{n} \left( \sum_{i=1}^Na_i - N(A+\varepsilon) \right)\]
  因为$\lim\limits_{n \to \infty}\dfrac{1}{n} = 0$,所以对于前述正实数$\varepsilon$,令
  \[ \varepsilon_1= \frac{\varepsilon}{\left| \sum_{i=1}^Na_i - N(A-\varepsilon) \right|} \]
  和正实数
  \[ \varepsilon_2= \frac{\varepsilon}{\left| \sum_{i=1}^Na_i - N(A+\varepsilon) \right|} \]
  再令$\varepsilon_0=\max\{\varepsilon_1,\varepsilon_2\}$,则存在正整数$N_0$,使得当$n>N_0$时,$\dfrac{1}{n}<\varepsilon_0$,于是在$n>N_1=\max\{N,N_0\}$时有
  \[ A-2\varepsilon < M_n < A+2\varepsilon \]
  这便说明$\lim\limits_{n \to \infty}M_n = A$,需要提醒的是这里出现的$2\varepsilon$似乎与极限定义不一致,但实际上,只要在这整个过程将$\varepsilon$替换为$\dfrac{1}{2}\varepsilon$,就可以与极限定义完全一致,而且从另一个角度来说,极限定义中只是说$\varepsilon$是任意正实数,那么如果$\varepsilon$是任意正实数,$2\varepsilon$照样可以取遍任意正实数,所以这并没有本质上的不同,关于这一点以后将不再说明了。
\end{example}

\begin{example}[无穷级数的和]
  对于数列$a_n$,我们可以作出它的前$n$项的和
  \[ S_n = \sum_{i=1}^n a_i = a_1 + a_2 + \cdots + a_n \]
  依赖于$n$的取值,这些和将构成一个新的数列$S_1,S_2,\ldots$,现在我们把加法推广到无穷个数相加的情形,称无穷个数相加的表达式
  \[ \sum_{n=1}^{\infty} a_n = a_1 + a_2 + \cdots  \]
  为\emph{无穷级数},而$S_n$称为它的第$n$个\emph{部分和},如果$S_n$有(有限的)极限,即
  \[ \lim\limits_{n \to \infty} S_n = S \]
  则称无穷级数$\sum\limits_{n=1}^{\infty} a_n$ \emph{收敛},这个极限$S$称为级数的\emph{和},记作
  \[ \sum_{n=1}^{\infty} a_n = S \]
  如果部分和数列$S_n$没有极限,则称无穷级数是\emph{发散}的。
\end{example}

\subsection{极限的性质与运算}
\label{sec:properties-and-operation-of-limit}

\begin{theorem}[极限唯一性]
  如果数列$x_n$收敛,则极限唯一。
\end{theorem}

\begin{proof}[证明]
  反证法,假若有两个数$a1$和$a2$都是数列的极限,假定$a_1<a_2$,则对于任意正数$\varepsilon > 0$,数列都能从某项起同时成立着 $|x_n-a_1| < \varepsilon$ 和 $|x_n-a_2| < \varepsilon$,于是取$\varepsilon < (a_2-a_1)/2$,则前述两个不等式因为无交集而产生矛盾。
\end{proof}

\begin{theorem}
  如果数列$x_n$收敛到实数$a$,则对于任意一个小于$a$的实数$x$,数列都能从某项起恒大于$x$,同样对于任意一个大于$a$的实数$y$,数列也能从某项起恒大于$y$。
\end{theorem}

\begin{proof}[证明]
  只要在极限的定义中取 $\varepsilon < a-x$即可得前半部分结论,同样再取$\varepsilon < y-a$即得后半部分结论。
\end{proof}

\begin{inference}[保号性]
  如果数列收敛到一个正的实数,则数列必从某项起恒保持正号,同样,若收敛到一个负的实数,则必从某项起恒保持负号。
\end{inference}


\begin{theorem}[收敛数列的有界性]
  收敛数列必有界。
\end{theorem}

\begin{proof}[证明]
  这其实从定义就可以得出了,随便取一个$\varepsilon>0$,即知数列从某项起全部落在区间$(a-\varepsilon, a+\varepsilon)$内,这里$a$是数列极限,再扩大此区间把前面的那些项(有限个)包含进来,于是数列便有界。
\end{proof}

\begin{theorem}[保不等式性]
  设数列$a_n$与数列$b_n$分别收敛到$A$与$B$,且当$n$充分大时恒有$a_n<b_n$,则必有$A \leqslant B$.
\end{theorem}

\begin{proof}[证明]
  反证法,若$A>B$,则取$\varepsilon=\dfrac{1}{2}(A-B)$,在$n$充分大时必同时有$a_n>A-\varepsilon=\dfrac{1}{2}(A+B)$以及$b_n<B+\varepsilon=\dfrac{1}{2}(A+B)$成立,这时显然有$a_n>b_n$,与定理条件矛盾,所以$A \leqslant B$.
\end{proof}

要注意的是由定理中的条件并不能得出$A<B$的结论来,例如$a_n=\dfrac{1}{n^2}$与$b_n=\dfrac{1}{n}$.

\begin{theorem}[夹逼准则,迫敛性]
 若三个数列$a_n$、$b_n$、$c_n$在$n$充分大时恒有$a_n \leqslant b_n \leqslant c_n$,并且$a_n$与$c_n$都收敛到同一极限$M$,则$b_n$亦必收敛到此极限值. 
\end{theorem}

\begin{proof}[证明]
  对任意小的正实数$\varepsilon$,显然当$n$充分大时有
  \[ M-\varepsilon < a_n < b_n < c_n < M+\varepsilon \]
  由此即得定理.
\end{proof}

\begin{theorem}
  如果数列$x_n$和$y_n$分别收敛到$x$和$y$,则数列$x_n+y_n$、$x_n-y_n$、$x_ny_n$、$x_n/y_n$都收敛,而且极限分别是$x+y$、$x-y$、$xy$、$x/y$,在商的情况要求$y \neq 0$。
\end{theorem}

这定理可以推广到任意有限个数列的情形。

\begin{proof}[证明]
  和差的情况是容易证明的,只证明积和商的情况。

  先证明乘积的情形,由
  \begin{equation*}
    |x_ny_n-xy| = |(x_ny_n-xy_n) + (xy_n-xy)| \leqslant |y_n||x_n-x| + |x| |y_n-y|
  \end{equation*}
  任取$\varepsilon > 0$,则存在$N>0$,使得$n>N$时同时恒有$|x_n-x|<\varepsilon$和$|y_n-y|<\varepsilon$\footnote{本来对同一个$\varepsilon$,两个数列的$N$是不同的,但是可以取比这两个$N$都大的$N$,这时就同时有那两个不等式。},另外再由收敛数列的有界性,存在$M>0$,使得$ |y_n| < M$,于是就有 $|x_ny_n-xy| < (M+|x|)\varepsilon$,所以$x_ny_n$收敛到$xy.$

  再来证明商的情况,先证明一个结论,如果数列$y_n$收敛到一个非零实数$y$,那么数列$1/y_n$必收敛,且收敛到$1/y$,这是因为
  \begin{equation*}
    \left| \frac{1}{y_n} - \frac{1}{y} \right| = \left| \frac{y_n-y}{yy_n} \right|
  \end{equation*}
  对于任意正实数$\varepsilon>0$,上式的分子能从某一个下标$N$开始恒小于$\varepsilon$,同时再取另外一个正实数$|y|/2$,数列能从某项起恒有$|y_n|>|y|/2$,于是从某个下标开始,上式就能恒小于$2\varepsilon / y^2$,所以数列$1/y_n$收敛到$1/y$,再将$x_n/y_n$视为$x_n$乘以$1/y_n$并利用乘积的结果,便得商的情形。
\end{proof}


\begin{example}
  给定数列$x_n$的前两项$x_1$与$x_2$,以后的项由公式
  \[ x_{n+1}=\frac{1}{2}(x_n+x_{n-1}) \]
  确定,我们来讨论一下它的收敛情况。

  递推公式可以改写成
  \[ x_{n+1}-x_n = -\frac{1}{2}(x_n-x_{n-1}) \]
  因此,递推下去有
  \[ x_{n+1}-x_n = \left( -\frac{1}{2} \right)^{n-1}(x_2-x_1) \]
  于是
  \begin{eqnarray*}
    x_n & = & x_1 + \sum_{k=2}^n(x_k-x_{k-1})  \\
        & = & x_1 + (x_2-x_1)\sum_{k=2}^n \left( -\frac{1}{2} \right)^{k-2} \\
    & = & x_1 + (x_2-x_1) \cdot \frac{2}{3}\left( 1- \left( - \frac{1}{2} \right)^{n-1} \right)
  \end{eqnarray*}
  于是有
  \[ \lim_{n \to \infty} x_n = x_1 + \frac{2}{3}(x_2-x_1) = \frac{x_1+2x_2}{3} \]
\end{example}

\begin{example}[几何级数]
  设实数$q$满足$|q|<1$,称级数
  \[ \sum_{n=0}^{\infty} q^n = 1+q+q^2+\cdots  \]
  为\emph{几何级数},它的部分和
  \[ S_n = \sum_{i=0}^n q^i = \frac{1-q^{n+1}}{1-q} \]
  因为$|q|<1$,所以可以得出
  \[ \lim_{n \to \infty} S_n = \frac{1}{1-q} \]
  也就是说,几何级数的和是
  \[ \sum_{n=0}^{\infty} q^n = \frac{1}{1-q} \]
\end{example}

\begin{inference}
  \label{inference:limit-of-power-of-number-sequence}
  若数列$a_n$收敛到实数$A$,则对于固定的整数$m$,数列$a_n^m$收敛到实数$A^m$.
\end{inference}

\begin{example}
  \label{example:limit-of-n-sqrt-a}
  设实数$a>0$且$a \neq 1$,证明极限$\lim_{n \to \infty} \sqrt[n]{a} = 1$.

  我们在 \autoref{example:limit-of-n-sqrt-a-when-a-greater-than-1}中已经证明了$a>1$时的情形,现在假设$0<a<1$,则有
  \[ \lim_{n \to \infty} \sqrt[n]{a} = \lim_{n \to \infty} \frac{1}{\sqrt[n]{\frac{1}{a}}} = 1 \]
\end{example}

\begin{example}
  \label{example:limit-of-n-power-m-devide-by-a-power-n}
  在\autoref{example:limit-of-n-devide-by-a-power-n}中,我们已经证明了下面的极限
  \[ \lim_{n \to \infty} \frac{n}{a^n} = 0 \]
  其中$a>1$,现在我们将它推广,设$m$是一个正整数,则
  \[ \lim_{n \to \infty} \frac{n^m}{a^n} = 0 \]
  这是因为,由于$\sqrt[m]{a}>1$,根据\autoref{example:limit-of-n-devide-by-a-power-n}中的结论,有
  \[ \lim_{n \to \infty} \frac{n}{(\sqrt[m]{a})^n} = 0 \]
 再由\autoref{inference:limit-of-power-of-number-sequence},便得出
  \[ \lim_{n \to \infty} \frac{n^m}{a^n} = 0 \]
  这表明在$n$趋于无限大时,多项式与指数相比,它将变得微不足道,也就是说,指数是比多项式更高阶的无穷大。
\end{example}


\subsection{无穷小与无穷大}
\label{sec:infinite-small-and-great}

\begin{definition}
  如果数列收敛到零,则称它在$n$无限增大时是一个\emph{无穷小}.
\end{definition}

\begin{definition}
  如果数列$a_n$满足: 对任意大的正实数$M$,总存在正整数$N$,使得当$n>N$时恒有$|a_n|\geqslant M$,则称这数列在$n$无限增大时是一个\emph{无穷大},如果$a_n$在$n$充分大时还保持着一定的符号,则称之为\emph{正无穷大}或者\emph{负无穷大},此时也称数列收敛到正无穷或者收敛到负无穷,记作$\lim\limits_{n \to \infty} a_n = + \infty$或者$\lim\limits_{n \to \infty} a_n = -\infty$.
\end{definition}

显然,当$a_n$是非零的无穷小时,$\dfrac{1}{a_n}$是无穷大,反之也对。

例如$\dfrac{1}{n}$便是无穷小的例子,而$n^2$则是一个无穷大的例子。

为讨论的方便,我们引入无穷大的邻域概念.
\begin{definition}
  正无穷大的邻域是开区间$(a,+\infty)$,负无穷大的邻域是开区间$(-\infty,a)$.
\end{definition}

\begin{example}
  今来考虑关于$n$的多项式
  \[ a_mn^m+a_{m-1}n^{m-1} + \cdots a_1n+a_0 \]
  在$n \to \infty$时的极限情况。

  将最高次幂提出来,多项式写成下面的形状
  \[ n^m \left( a_m+\frac{a_{m-1}}{n}+\frac{a_{m-2}}{n^2} + \cdots + \frac{a_0}{n^m} \right) \]
  在$n \to \infty$时,显然括号中的部分将趋于极限值$a_m$,但外面有一个无穷大因子$n^m$,所以我们得到结论,如果最高次项系数为正,则多项式将趋于正无穷,如果这系数为负,则它将趋于负无穷。
\end{example}

\begin{example}
  再来考察关于$n$的有理式子的极限,即关于$n$的两个多项式之比的极限:
  \[ \frac{a_mn^m+a_{m-1}n^{m-1} + \cdots a_1n+a_0}{b_ln^l+b_{l-1}n^{l-1} + \cdots b_1n+b_0} \]
  这里$a_m \neq 0$,$b_l \neq 0$.

  如果$m=l$,则分子分母同时除以最高次幂变形为
  \[ \frac{a_m+\frac{a_{m-1}}{n}+\cdots+\frac{a_0}{n^m}}{b_m+\frac{b_{m-1}}{n}+\cdots+\frac{b_0}{n^m}} \]
  显然它的极限是$\dfrac{a_m}{b_m}$,即最高次项系数之比。

  如果$m<l$,则分子分母同时除以$n^l$后,原式成为
  \[ \frac{\frac{a_m}{n^{l-m}}+\cdots+\frac{a_0}{n^l}}{b_m+\frac{b_{m-1}}{n}+\cdots+\frac{b_0}{n^l}} \]
  分子趋于$0$,而分母趋于$b_m$,因此此时极限为零。

  同理,当$m>l$时,分子分母同时除以$n^l$后,分母趋于有限的极限$b_m$,但分母却是无穷大,所以此时它的极限是无穷大,其符号由分子分母的最高次项系数共同决定,相同时为正无穷,相反时为负无穷。
\end{example}

\begin{definition}
   设数列$a_n$与$b_n$都是无穷小,即在$n$无限增大时都收敛到零,\\
  (1). 若$\lim\limits_{n \to \infty} \dfrac{a_n}{b_n} = 0$,则称$a_n$是$b_n$的\emph{高阶无穷小},记作$a_n=o(b_n)$. \\
  (2). 若存在正实数$U$和$V$,使得当$n$充分大时有$U\leqslant \left| \dfrac{a_n}{b_n} \right| \leqslant V$,则称$a_n$与$b_n$是\emph{同阶无穷小},特别的,如果$\lim\limits_{n \to \infty} \dfrac{a_n}{b_n} = 1$,则称它俩是\emph{等价无穷小},记作$a_n \sim b_n$.
\end{definition}

例如,$\dfrac{1}{n^2}$是$\dfrac{1}{n}$的高阶无穷小,而$\dfrac{1}{n}$与$\dfrac{1+\sin{n}}{n}$则是同阶无穷小,这个概念体现了无穷小之间的比较,需要说明的是,两个无穷小之间并不必然有阶的高低之分,因为它俩之比可以不收敛。


关于无穷小,还有如此结论
\begin{theorem}
  如果数列$a_n$是无穷小,数列$b_n$有界,则数列$a_nb_n$是无穷小。
\end{theorem}

\begin{proof}[证明]
  由条件,存在正实数$M$,使得$b_n$中的所有项都满足$|b_n|\leqslant M$,所以$|a_nb_n| \leqslant M |a_n|$总是成立的,而$a_n$为无穷小,所以对于无论多么小的正实数$\varepsilon$,数列$a_n$总能从某项起恒满足$|a_n|<\varepsilon/M$,从而$|a_nb_n|<\varepsilon$,于是结论成立。
\end{proof}

由此知道,数列$\dfrac{\sin{n}}{n}$是无穷小.

\begin{theorem}
  有限个无穷小之和仍是无穷小.
\end{theorem}

利用极限定义即可简单证明,略去。

关于同阶无穷小,则有如下结论
\begin{theorem}
  若$a_n$与$b_n$是同阶无穷小,$c_n$是任一数列,则数列$a_nc_n$与$b_nc_n$的收敛性情况相同,即要么都收敛要么都发散,并且在收敛的情况下,如果又是等价无穷小,则还具有相同的极限值.
\end{theorem}

这定理关于等价无穷小的结论是非常有用的,它表明在计算极限的过程中,可以将一个无穷因子替换为与它等价的无穷小。

与无穷小类似,我们可以定义高阶无穷大的概念。

\begin{definition}
  设数列$x_n$和$y_n$在$n\to\infty$时都是无穷大,如果$\lim\limits_{n \to \infty} \left| \dfrac{x_n}{y_n} \right| = +\infty$,则称$x_n$是$y_n$的\emph{高阶无穷大}.
\end{definition}

$x_n$是$y_n$的高阶无穷大,就意味着当$n\to\infty$时,$y_n$的大小与$x_n$相比完全可以忽略不计,尽管$y_n$自身也是无穷大。

\begin{example}
  \label{example:infinite-large-compare}
  根据\autoref{example:limit-of-lnn-devide-by-n}、\autoref{example:limit-of-a-power-n-devide-by-n-fraction}、\autoref{example:limit-of-n-power-m-devide-by-a-power-n}的结果,我们可以知道,在对数、多项式、指数、阶乘这些无穷大中,后面的无穷大,都是前面的无穷大的高阶无穷大,如果我们用$x_n \prec y_n$来表示$y_n$是$x_n$的高阶无穷大,那么有
  \[ \ln{n} \prec n \prec n^2 \prec \cdots \prec n^m \prec a^n \prec n! \]
  式中$m>2$是正整数,$a>1$为一个实数。
\end{example}

\subsection{Stolz 定理}
\label{sec:stolz-theorem}

\begin{theorem}[Stolz 定理]
  对于两个数列$a_n$和$b_n$,其中$b_n$是一个(至少从某一项开始)严格增加到正无穷或者严格减小到负无穷的数列,如果$\lim\limits_{n\to\infty}\dfrac{a_{n+1}-a_n}{b_{n+1}-b_n} = M$,那么有$\lim\limits \dfrac{a_n}{b_n} = M$,这里的$M$可以是一个实数,也可以是正无穷或者负无穷。
\end{theorem}

\begin{proof}[证明]
  先证明$b_n$严格增加到正无穷以及$l$是一个有限数的情况,此时由两个数列增量之比收敛到$l$,所以对于任意小的正数$\varepsilon$,当$n$充分大时($n \geqslant N$)恒有
  \[ (l-\varepsilon)(b_{n+1}-b_n) < a_{n+1}-a_n < (l+\varepsilon)(b_{n+1}-b_n) \]
  累加可得
  \[ (l-\varepsilon)(b_{n}-b_N) < a_{n}-a_N < (l+\varepsilon)(b_{n}-b_N) \]
  三边同时加上$a_N$再除以$b_n$可得($b_n$必定从某一项开始恒保持正号)
  \[ l-\varepsilon+\frac{a_N-(l-\varepsilon)b_N}{b_n} < \frac{a_n}{b_n} < l+\varepsilon+\frac{a_N-(l+\varepsilon)b_N}{b_n}\]
  因为$b_n$收敛到正无穷,所以当$n$充分大时,上式左右两边以$b_n$为分母的两项的绝对值可以任意小,从而当$n$充分大时就有
  \[ l-2\varepsilon < \frac{a_n}{b_n} < l + 2\varepsilon \]
  这就表明两个数列之比也收敛到$l$.

  要说明的是,上述证明过程中,$N$的取值并不是恒定不变的,每出现一次“当$n$充分大时”之类的字眼,就意味着$N$可能需要取更大的值,以使得前面的不等式与新引入的不等关系能够同时成立。

  再证明$l$是正无穷大的情形,这时可以利用已经证明了的有限数的结果,因为这时显然有
  \[ \lim_{n \to \infty} \frac{b_{n+1}-b_n}{a_{n+1}-a_n} = 0 \]
  只需要说明$a_n$能够从某项起单调增加到正无穷大就可以了。对于任意大的正实数$M$,当$n$充分大时显然有
  \[ a_{n+1}-a_n > M(b_{n+1}-b_n) \]
  而显然从某一个正整数开始恒有$b_{n+1}-b_n>1$,所以此时有$a_{n+1}-a_n>M$,这便说明$a_n$能够从某一项开始单调增加到正无穷了,于是便可以得到我们要的结果。
\end{proof}

\begin{inference}
  如果数列$x_n$收敛到$A$,则$(x_1+x_2+\cdots+x_n)/n$也收敛到$A$,反之亦然。
\end{inference}

这正是我们在\autoref{example:mean-value-of-converge-number-sequence}中所得到的结论,那里是用极限定义证明了这个事实,现在有了 Stolz 定理,它就是一个显而易见的结论了。

\begin{proof}[证明]
  只要在 Stolz 定理中取$a_n=x_1+x_2+\cdots+x_n$,$b_n=n$便得此结论。
\end{proof}

\begin{example}
  我们证明,对任意确定的正整数$m$,有
  \[ \lim_{n \to \infty} \frac{1^m+2^m+\cdots+n^m}{n^{m+1}} = \frac{1}{m+1} \]
  只要取$a_n=1^m+2^m+\cdots+n^m$与$b_n=n^{m+1}$,有
  \[ \lim_{n \to \infty} \frac{a_{n+1}-a_n}{b_{n+1}-b_n} = \lim_{n \to \infty} \frac{(1+n)^m}{(1+n)^{m+1}-n^{m+1}} = \lim_{n \to \infty} \frac{n^m+\cdots}{(m+1)n^m+\cdots} = \frac{1}{m+1} \]
  根据Stolz定理,便得原式亦有相同极限。

  在初等数学中,我们已经知道自然数的幂和$S(n,m)=\sum\limits_{i=1}^ni^m$是一个关于$n$的$m+1$次多项式,并且它的最高次项系数就是$\dfrac{1}{m+1}$,所以这里的极限是毫不意外的。
\end{example}

\begin{example}
  现在把上一例的结果进行加强,上例中求出了
  \[ \lim_{n \to \infty} \frac{1^m+2^m+\cdots+n^m}{n^{m+1}} = \frac{1}{m+1} \]
  那么左端的数列减去极限值便是一个无穷小,这个无穷小的阶是如何的呢,现在我们拿它与$\dfrac{1}{n}$进行比较,即要求极限
  \[ \lim_{n \to \infty} n \left( \frac{1^m+2^m+\cdots+n^m}{n^{m+1}} - \frac{1}{m+1} \right) \]
  将原式通分变形为
  \[ \frac{(m+1)(1^m+2^m+\cdots+n^m)-n^{m+1}}{(m+1)n^m} \]
  将分子作为数列$x_n$,分母作为数列$y_n$,显然$y_n$是单调增加到正无穷的,而
  \begin{eqnarray*}
    \frac{x_{n+1}-x_n}{y_{n+1}-y_n} & = & \frac{(m+1)(n+1)^m-((n+1)^{m+1}-n^{m+1})}{(m+1)((n+1)^m-n^m)} \\
    & = & \frac{(m+1)\sum_{i=0}^mC_m^in^i-\sum_{i=0}^mC_{m+1}^in^i}{(m+1)\sum_{i=0}^mC_m^in^i}
  \end{eqnarray*}
  可以发现,分子分母都是关于$n$的$m-1$次多项式,因此极限应对应的最高次项系数之比,即
  \[ \frac{(m+1)C_m^{m-1}-C_{m+1}^{m-1}}{(m+1)C_m^{m-1}} = \frac{1}{2} \]
  于是根据Stolz定理,就有
  \[ \lim_{n \to \infty} n \left( \frac{1^m+2^m+\cdots+n^m}{n^{m+1}} - \frac{1}{m+1} \right) = \frac{1}{2} \]
  这就是说,上例中的数列与极限作差的无穷小,与$\dfrac{2}{n}$是等价无穷小.
\end{example}

\begin{example}
  我们证明一个关于调和级数与对数之间关系的一个重要极限:
  \[ \lim_{n \to \infty} \frac{1+\frac{1}{2}+\cdots+\frac{1}{n}}{\ln{n}} = 1 \]
  取$a_n=1+\dfrac{1}{2}+\cdots+\dfrac{1}{n}$,$b_n=\ln{n}$,则
  \[ \lim_{n \to \infty} \frac{a_{n+1}-a_n}{b_{n+1}-b_n} = \lim_{n \to \infty} \frac{1}{(n+1)\ln{\left( 1+ \frac{1}{n} \right)}} = 1 \]
  显然$b_n$是单调增加到正无穷的,因此由Stolz定理即得所要证的极限。

  这个例子的意义在于,它揭示了调和级数的增长速度,与对数是同一级别的,以后我们还将看到,它俩不但增长速度接近,事实上它俩之差会趋于一个固定值(欧拉常数$c=0.577\cdots$).
\end{example}


\subsection{单调有界定理}
\label{sec:theorem-of-monotone-bounded}

\begin{theorem}
  单调递增有上界的数列必定收敛,而且收敛到它的上确界。单调递减有下界的数列也类似。
\end{theorem}

\begin{proof}[证明]
  只证明单调递增有上界的情况,假如数列$x_n$就是这样的数列,它的上确界是$M$,则数列中的全部项都满足$x_n \leqslant M$,另外,对于任意小的正实数$\varepsilon$,由上确界定义,总存在某个$x_N$满足$x_N>M-\varepsilon$,再由单调性即知对于$n>N$恒有$M-\varepsilon < x_n \leqslant M < M+\varepsilon$,所以$M$就是这数列的极限。
\end{proof}

\begin{example}
  考虑下面的具有$n$重根号的式子
  \[ \sqrt{2+\sqrt{2+\cdots+\sqrt{2}}} \]
  为求其无限个根号时的极限,用数列来归纳的定义它,$x_1=\sqrt{2}$,$x_{n+1}=\sqrt{2+x_n}$.

  显然这是一个正项数列,由$x_n<2$可得到$x_{n+1}=\sqrt{2+x_n}<\sqrt{2+2}=2$,又$x_1<2$,因此数列有上界,又
  \[ x_{n+1}^2-x_n^2=2+x_n-x_n^2=(2-x_n)(1+x_n)>0 \]
  所以数列单调增加,于是这数列有极限,设其极限为$x$,则在递推式两边取极限得$\sqrt{2+x}=x$,解之得$x=2$,即它收敛到2.

  可以看到,如果$x_1$的取值大于2,则这数列将单调减少并恒大于2,但会收敛到2.

  类似的,对于递推数列$x_{n+1}=\sqrt{c+x_n}$,这里$c>0$为常数,如果数列收敛,设极限为$x$,显然$x=\sqrt{c+x}$,即$x=\dfrac{1+\sqrt{1+4c}}{2}$,又如果$x_n<x$,则显然$x_{n+1}<x$,如果$x_n>x$,则$x_{n+1}>x$,也就是说,只要首项$x_1$在$x$的某一边,则数列的全部项都在同一侧,而
  \[ x_{n+1}^2-x_n^2=c+x_n-x_n^2=\left( x_n - \frac{1+\sqrt{1+4c}}{2} \right)\left( \frac{1-\sqrt{1+4c}}{2}-x_n \right) \]
  所以如果$x_1<x=\dfrac{1+\sqrt{1+4c}}{2}$,则$x_n$单调增加有上界,如果$x_1>x$,则数列单调减少有下界,无论哪种情况,都会收敛到$\dfrac{1+\sqrt{1+4c}}{2}$.
\end{example}

\begin{example}
  序列
  \[ 1+\frac{1}{2^2}+\cdots+\frac{1}{n^2} \]
  显然是单调增加的,又
  \begin{eqnarray*}
    && 1+\frac{1}{2^2}+\cdots+\frac{1}{n^2} \\
    & < & 1 + \frac{1}{1\times 2}+\cdots+\frac{1}{(n-1)n} \\
    & = & 1+\left( 1-\frac{1}{2} \right)+\cdots+\left( \frac{1}{n-1}-\frac{1}{n} \right) \\
    & = & 2-\frac{1}{m} < 2
  \end{eqnarray*}
  因此它有上界,于是收敛,也就是说,级数$\lim\limits_{n=1}^{\infty}\dfrac{1}{n^2}$收敛,以后会看到,这个和是$\dfrac{\pi^2}{6}$.
\end{example}

\begin{example}
  序列
  \[ 1+\frac{1}{1!} + \frac{1}{2!} + \cdots + \frac{1}{n!} \]
  它显然是单调增加的,而关于它上有界,则有两种放缩方式可以证明,其一是利用$\dfrac{1}{n!}<\dfrac{1}{2^{n-1}}(n \geqslant 2)$,其二是利用$\dfrac{1}{n!}<\dfrac{1}{(n-1)n}=\dfrac{1}{n-1}-\dfrac{1}{n}$,因此它也有上界,从而有极限,这就是说: 级数$\sum\limits_{n=0}^{\infty} \dfrac{1}{n!}$收敛,以后将看到,它会收敛到$\mathrm{e}$,自然对数的底数。
\end{example}

\begin{example}
  序列
  \[ \left( 1-\frac{1}{2} \right) \left( 1-\frac{1}{4} \right) \cdots \left( 1-\frac{1}{2^n} \right) \]
  显然是单调减少有下界的,因此它收敛,不过它的极限,却不能从递推式取极限而得出来了。
\end{example}

\begin{example}[等差-等比中项]
  \label{example:arithmetic-gemotry-mid-term}
  考虑两个数列$a_n$和$b_n$,任取$a_1=a>0$,$b_1=b>0$,以后的项由下式确定
  \[ a_{n+1} = \sqrt{a_nb_n}, \  b_{n+1}=\frac{a_n+b_n}{2} \]
  现考察一下这两个数列的极限情况。

  显然,两个数列的所有项都是正的,而且无论两个数列的第一项如何取值,从第二项起便恒有$a_n \leqslant b_n$,因此
  \[ a_{n+1} = \sqrt{a_nb_n} \geqslant a_n \]
  即$a_n$从第二项开始单调增加(不减),同时有
  \[ b_{n+1} = \frac{a_n+b_n}{2} \leqslant b_n \]
  即$b_n$从第二项开始单调减少(不增),另一方面,容易发现两个数列都是有界的:
  \[ a_2 \leqslant a_3 \leqslant \cdots \leqslant a_n \leqslant \cdots \leqslant b_n \leqslant \cdots \leqslant b_3 \leqslant b_2  \]
  即$a_n$是单调增加有上界,而$b_n$是单调减少有下界,因此它俩都有极限,再在$a_n$的递推式两端取极限便知它们有着共同的极限值,毫无疑问,这个极限值是由两个数列的首项$a$和$b$所唯一确定的,它称为$a$和$b$的\emph{等差-等比中项},但要想在这里求出它的具体表达式,却是不可能的,它不可能用目前我们已知的任何表达式写出来,以后将会看到,它可以利用椭圆积分表出。
\end{example}

\begin{example}[调和-算术中项]
  \label{example:harmonic-arithmetic-mid-term}
  与\autoref{example:arithmetic-gemotry-mid-term}相仿,不过现在我们将两个平均数分别替换为调和平均与算术平均:
  \[ a_{n+1} = \frac{2}{\frac{1}{a_n}+\frac{1}{b_n}}, \  b_{n+1} = \frac{a_n+b_n}{2} \]
  类似的可得出
  \[ a_2 \leqslant a_3 \leqslant \cdots \leqslant a_n \leqslant \cdots \leqslant b_n \leqslant \cdots \leqslant b_3 \leqslant b_2  \]
  因此这两个数列仍有极限,这极限称为数$a$与$b$的\emph{调和-算术平均},与算术-几何平均不同的是,调和-算术平均有简单表达式,因为恒有关系
  \[ a_{n+1}b_{n+1} = \frac{2a_nb_n}{a_n+b_n} \cdot \frac{a_n+b_n}{2} = a_nb_n \]
  因此恒有$a_nb_n=a_1b_1=ab$,所以这个极限是$\sqrt{ab}$,即两个数的调和-算术平均,就是两者的几何平均。
\end{example}

\begin{example}
  数列
  \[ \frac{1}{n+1}+\frac{1}{n+2}+\cdots+\frac{1}{2n} \]
  它共有$n$个加数,且依次减小,把所有加数都放大为第一个,则得出它小于$\dfrac{n}{n+1}<1$,这表明它有上界,再看单调性,当$n$变为$n+1$时,末尾会增加两项$\dfrac{1}{2n+1}$和$\dfrac{1}{2n+2}$,但前面会少一项$\dfrac{1}{n+1}$,因此其增量为
  \[ \frac{1}{2n+1}+\frac{1}{2n+2}-\frac{1}{n+1} > 2 \cdot \frac{1}{2n+2} - \frac{1}{n+1} = 0 \]
  故它又是递增的,因此有极限,且极限不超过1,以后将看到(\autoref{example:limit-of-sum-of-i/n+1-i-in-(1-n)}),它的极限是$\ln{2}$.
\end{example}

\begin{example}
  \label{limit-of-cos-pi-frac-n}
  这里我们建立如下两个极限
  \[ \lim_{n\to\infty}\cos{\frac{\pi}{n}}=1 \]
  及
  \[ \lim_{n\to\infty}\sin{\frac{\pi}{n}}=0 \]
  我们先证明$n$为$2$的幂的情况:
  \[ \lim_{n\to\infty}\cos{\frac{\pi}{2^n}}=1 \]
  及
  \[ \lim_{n\to\infty}\sin{\frac{\pi}{2^n}}=0 \]

  依单调性, 序列 $\cos{\frac{\pi}{2^n}}$ 是单调增加的,且有上界,因此必有极限,只是要求出此极限,还需要别的方法。

  
  由余弦半角公式
  \[ \cos{\frac{\theta}{2}} = \sqrt{ \frac{1+\cos{\theta}}{2} } \]
  可得
  \begin{equation*}
    \begin{split}
        & 1-\cos{\frac{\pi}{2^{n+1}}} \\
        =& 1-\sqrt{\frac{1+\cos{\frac{\pi}{2^{n}}}}{2}} \\
      = & \frac{1-\cos{\frac{\pi}{2^{n}}}}{2 \left( 1+\sqrt{\frac{1+\cos{\frac{\pi}{2^{n}}}}{2}} \right)} \\
      < & \frac{1}{2} \left( 1-\cos{\frac{\pi}{2^{n}}} \right)
    \end{split}
  \end{equation*}
  于是
  \[ 0 < 1-\cos{\frac{\pi}{2^{n+1}}} < \frac{1}{2^{n-1}} \left( 1- \cos{\frac{\pi}{2}} \right) = \frac{1}{2^{n-1}} \]
  这便表明
  \[ \lim_{n\to\infty}\cos{\frac{\pi}{2^n}}=1 \]
  同时有
  \[ \lim_{n\to\infty}\cos{\frac{\pi}{2^n}} = \lim_{n\to_{infty}}\sqrt{1-\cos^2 \frac{\pi}{2^n} } = 0 \]
  在后文讲述柯西收敛准则后,将可得到
  \[ \lim_{n\to\infty}\cos{\frac{\pi}{n}}=1 \]
  及
  \[ \lim_{n\to\infty}\sin{\frac{\pi}{n}}=0 \]
\end{example}

\subsection{闭区间套定理}
\label{sec:theorem-of-closed-interval-sequence}

\begin{theorem}[闭区间套定理]
  \label{closed-interval-sequence-theorem}
  设闭区间序列$[a_n,b_n](n=1,2,\ldots)$ 满足:
  \begin{enumerate}
  \item $[a_1,b_1] \supseteq [a_2,b_2] \supseteq \cdots [a_n,b_n] \supseteq [a_{n+1},b_{n+1}] \supseteq \cdots$
  \item $\lim\limits_{n \to \infty} (b_n-a_n) = 0$
  \end{enumerate}
    则存在唯一实数$x$,使得$x \in [a_n,b_n](n=1,2,\ldots)$.
\end{theorem}

\begin{proof}[证明]
  显然$a_n$是一个单调增加有上界的序列,所有的$b_m$都是它的上界,同样,$b_n$是单调减少有下界的数列,所有的$a_m$都是它的下界,因此两个数列均有极限,分别记为$A$和$B$,由$a_n < b_n$恒成立知$A \leqslant B$,又如果$A<B$,那么有$a_n\leqslant A < B \leqslant b_n$,这表明$b_n-a_n>B-A$,这与区间长度趋于零相矛盾,所以只能$A=B$,并且它同时位于所有的闭区间上,如果还有另一个与之不同的点也能同处于所有的闭区间上,则显然与区间长度趋于零相矛盾,所以这个共同极限是唯一同时位于所有区间上的数。
\end{proof}

\subsection{柯西收敛准则}
\label{sec:cauchy-convergence-rule}

\begin{theorem}[柯西收敛准则]
  数列$x_n$收敛的充分必要条件是,对于任意正实数$\varepsilon$,总存在正整数$N>0$,使得任意$n_1>N$和任意$n_2>N$及任意恒有$|x_{n_1}-x_{n_2}| < \varepsilon$。
\end{theorem}

\begin{proof}[证明]
  先证必要性. 如果数列$x_n$收敛到$x$,那么对于任意正实数$\varepsilon$,都有正整数$N$,使得$n>N$时恒有$|x_n-x|<\varepsilon / 2$,于是对于任意$n_1>N$及$n_2>N$,便有$|x_{n_1}-x_{n_2}|=|(x_{n_1}-x)- (x_{n_2}-x)|\leqslant |x_{n_1}-x|+|x_{n_2}-x|<\varepsilon / 2+\varepsilon / 2 = \varepsilon$。必要性得证。

  再证充分性,先任取一个正实数$\varepsilon_1$,数列中的项能够从某一项开始全部落在一个宽度为$2\varepsilon_1$的闭区间$I_1$,再取$\varepsilon_2 = \dfrac{1}{2}\varepsilon_1$,则数列又能够从另一项(一般来说需要向后推移若干个下标)开始尽数落在一个宽度为$2\varepsilon_2$的闭区间$I_2$上,且$I_1 \supseteq I_2$,依次类推,得出一个闭区间序列$I_1,I_2,\ldots$,其中每一个闭区间都能包含数列从某一项开始之后的全部项,且这些闭区间的长度趋于零,因此按闭区间套定理,存在唯一实数$M$同时处于所有的闭区间上,它将是这数列的极限,这是因为,对于任意正实数$\varepsilon$,显然存在某个闭区间$I_m$,使得$I_m \subset (M-\varepsilon,M+\varepsilon)$,而数列能够从某一项开始全部落在闭区间$I_m$上,也就从这一项开始全部落在$M$的$\varepsilon$邻域内,所以$M$就是这数列的极限。
\end{proof}

我们还可以得出柯西收敛准则的否定叙述:数列$a_n$发散的充分必要条件是,存在正实数$r$,使得无论对于多大的正整数$N$,总是存在两个正整数$n_1$和$n_2$,使得$|a_{n_1}-a_{n_2}| \geqslant r$.

\begin{example}
  前$n$个正整数的平方倒数和
  \[ S_n = 1 + \frac{1}{2^2} + \cdots + \frac{1}{n^2} \]
  对它的片段有
  \begin{eqnarray*}
    S_{m+p}-S_m  & = & \frac{1}{(m+1)^2} + \frac{1}{(m+2)^2} + \cdots + \frac{1}{(m+p)^2} \\
                 & < & \frac{1}{m(m+1)} + \frac{1}{(m+1)(m+2)} + \cdots + \frac{1}{(m+p-1)(m+p)} \\
                 & = & \left( \frac{1}{m} - \frac{1}{m+1} \right) + \left( \frac{1}{m+1} - \frac{1}{m+2} \right) + \cdots + \left( \frac{1}{m+p-1} - \frac{1}{m+p} \right) \\
    & = & \frac{1}{m} - \frac{1}{m+p} < \frac{1}{m}
  \end{eqnarray*}
  所以对于任意正实数$\varepsilon$,只要取$N>1/\varepsilon$,就能保证柯西条件成立,于是数列$S_n$有极限,不过这极限值在此处是求不出来的,在以后我们将会利用无穷级数理论,得到它的极限值,这极限值与圆周率有关:
  \[ \lim_{n \to \infty} \sum_{i=1}^n \frac{1}{i^2} = \frac{\pi^2}{6} \]
\end{example}

\begin{example}
  前$n$个正整数的阶乘的倒数和
  \[ T_n = 1 + \frac{1}{2!} + \cdots + \frac{1}{n!} \]
  它的片段和
  \begin{eqnarray*}
    T_{m+p} - T_m & = & \frac{1}{(m+1)!} + \frac{1}{(m+2)!} + \cdots + \frac{1}{(m+p)!} \\
                  & < & \frac{1}{2^{m+1}} + \frac{1}{2^{m+2}} + \cdots + \frac{1}{2^{m+p}} \\
    & = & \frac{1}{2^m} \left( 1-\frac{1}{2^p} \right) < \frac{1}{2^m}
  \end{eqnarray*}
  可见它也满足柯西收敛条件,所以这个数列也有极限,它的极限便是自然对数的底数$e$:
  \[ \lim_{n \to \infty} \sum_{i=1}^n \frac{1}{i!} = e \]
\end{example}

\begin{example}[级数收敛的柯西准则]
  因为级数收敛等同于它的部分和数列收敛,据此可以得出级数收敛的柯西准则:级数$\sum_{n=1}^{\infty} a_n$收敛的充分必要条件是,对于任意小的正实数$\varepsilon$,总存在正整数$n$和正整数$p$,使得$|a_{n+1}+a_{n+2} + \cdots + a_{n+p}|<\varepsilon$。

  同样可以得出它的反而叙述:级数$\sum_{n=1}^{\infty} a_n$发散的充分必要条件是,存在一个正实数$r$,使得无论对于多大的$N$,总存在该下标之后的某一片段和满足$|a_{n+1}+a_{n+2}+\cdots+a_{n+p}| \geqslant r$.
\end{example}

\begin{example}[调和级数的发散性]
  \label{example:non-convergency-of-harmonic-series-proof-by-cauchy}
  在\autoref{example:non-convergency-of-harmonic-series-proof-by-monotone}中,我们已经证明了调和级数
  \[ \sum_{n=1}^{\infty} \frac{1}{n} = 1+\frac{1}{2} + \frac{1}{3} + \cdots  \]
  的发散性,现在我们可以用柯西收敛准则来说明,但是所用的放缩仍然与那里是一致的,根据柯西准则,为了说明它的发散性,只要证明对于存在某个正实数$r$,无论对于多大的正整数$N$,总存在两个正整数$n_1$与$n_2$,使得$|H_{n_1}-H_{n_2}|\geqslant r$就可以了。

  仍然由
  \[ \frac{1}{2^m+1} + \frac{1}{2^m+2} + \cdots + \frac{1}{2^{m+1}} > 2^m \cdots \frac{1}{2^{m+1}} = \frac{1}{2} \]
可知,对于任意正整数$N$,只要取一个$n_1=2^{m_1}>N$,再取$n_2=2^{m_1+1}$,就有$|H_{n_1}-H_{n_2}|>\dfrac{1}{2}$,于是部分和数列$H_n$发散,即调和级数发散。

要说明的是,这里的放缩方式不是唯一的,比如说从级数中的任一项开始,作片段和
\[ \frac{1}{n+1} + \frac{1}{n+2} + \cdots + \frac{1}{n+m} > \frac{m}{n+m} \]
为了让右端能够大于某个正整数,取$m=n$,即片段的长度要依赖于片段中首项的下标,下标越大,片段越长,这里有
\[ \frac{1}{n+1} + \frac{1}{n+2} + \cdots + \frac{1}{2n} > \frac{1}{2} \]
于是我们将调和级数按下标划分片段,1和2分别为单独的片段,从3到4为一段,从5到8为一段,9到16为一段,这样划分之后,每一段上的和都大于$\dfrac{1}{2}$,但是这样的划分与前面的放缩其实是一致的,但是我们可以改为这里的$m$,以得出不同的划分,例如取$m=2n$,就得
\[ \frac{1}{n+1} + \frac{1}{n+2} + \cdots + \frac{1}{3n} > \frac{2}{3} \]
于是片段的划分方式是,1是单独的片段,从2到3为一段,从4到9为一段,从10到27为一段,依次类推,每个片段上的倒数和都大于$\dfrac{2}{3}$,于是级数发散。
\end{example}

\subsection{聚点定理}
\label{sec:accumulate-point-theorem}

\begin{definition}
  对于无穷数集$A$和某个实数$x$,若$x$的任意小的邻域$(x-\varepsilon,x_{\varepsilon})$内都包含了$A$中的无穷多个数,则称数$x$是数集$A$的一个\emph{聚点}.
\end{definition}

例如,零是数集$\left\{ \dfrac{1}{n} \right\}$的一个聚点。显然,定义中的邻域改成空心邻域也没有什么影响。

聚点可以在数集中,也可以不在集合中,事实上,如果聚点在集合中,那就从集合中除去此数,它仍然是新数集的聚点,并不受影响。

\begin{theorem}
  任意有界的无限数集,至少有一个聚点。
\end{theorem}

\begin{proof}[证明]
  因为数集有界,设它的全部元素被限于某个闭区间$I_1$中,它含有无限个元素,则将此闭区间对半切割为两个子闭区间,则至少其中一个子闭区间上含有原数集中的无穷多个元素,记此子闭区间为$I_2$,同样再对$I_2$进行对半切割,则可以得出一个含有原数集中无穷多个元素的子闭区间$I_2$,依次类推,得到一个闭区间序列,该序列中每个闭区间都包含了原数集中的无穷多个元素,并且这闭区间序列符合闭区间套定理的条件,因此由这闭区间序列确定出一个数$K$,它就是原数集的一个聚点,这是因为,对于任意小的正实数$\varepsilon$,显然前述闭区间中存在某个闭区间$I_m \in (K-\varepsilon,K+\varepsilon)$,而$I_m$中包含了原数集的无穷个元素,因此$K$的邻域也就包含了原数集中的无穷个元素,即为它的一个聚点。
\end{proof}

从数列中剔除掉一些项,剩下的项按照原来的顺序排列而成的新数列,称为原数列的一个\emph{子数列}.
\begin{inference}[博雷尔-魏尔斯特拉斯引理]
  任何有界数列,必存在收敛的子数列。
\end{inference}

\begin{proof}[证明]
  将数列的所有项作成一个有界数集,如果该数集是有限集,则表明数列中有无穷多个项取相同的数值,那么取这些项作成的子列就是收敛的,如果此数集是无限集,由聚点定理知它至少有一个聚点$x$,那么任意取定一个正实数$\varepsilon_1$,则数集中有无穷多个数落在$(x-\varepsilon_1,x+\varepsilon_1)$内并且不等于$x$,任取其中一个记为$a_{n_1}$,即它在原数列中的下标是$n_1$,则再取$\varepsilon_2 = \min \{ \dfrac{1}{2}\varepsilon_1, |x-a_{n_1}| \}$,又可以选落在区间$(x-\varepsilon_2,x+\varepsilon)$中并且不等于$x$的项$a_{n_2}$,这里$n_2$的选法要保证$n_2>n_1$,因为是有无穷多个数落在区间内,所以这总是可以办到的。依次类推,得出一个子列$a_{n_1},a_{n_2},\ldots$,显然这个子列是收敛到$x$的。
\end{proof}

现在讨论下无穷聚点,在讨论之前,我们先用邻域重新描述一下聚点:数$x$是无穷数集$A$的一个聚点的充分必要条件是,$x$的任意邻域都包含了数集$A$中的无穷多个数。我们已经引进过,正无穷的邻域是开区间$(a,+\infty)$,负无穷的邻域是开区间$(-\infty,a)$,因此我们把聚点的这个用邻域的定义照搬到正无穷和负无穷上,就得出无穷聚点的定义。

\begin{definition}
  如果对于任意实数$a$,正无穷的邻域$(a,+\infty)$中总是包含着无穷数集$A$中的无穷多个数,则称正无穷是数集$A$的一个\emph{无穷聚点},同理可以定义负的无穷聚点. 
\end{definition}

在引入了无穷聚点之后,聚点定理中的数集有界的要求就可以去掉了,因为如果数集无界,显然它必然有无穷聚点。同样,博雷尔-魏尔斯特拉斯引理中数列有界的限制也就可以去掉了,以后我们所说的聚点,均可以是有限的也可以是无限的,在这种意义上,我们把前面的相关定理修正为:
\begin{theorem}
  任意无穷数集必有聚点,且必有最大聚点和最小聚点.
\end{theorem}
必有聚点这是显然的,所有聚点组成一个集合,这集合有上下确界(可以无穷),这上下确界也必然是其聚点,关于这一点我们需要详细的证明,即要证明下面的定理
\begin{theorem}
  设$A$是一个无穷数集,而$E$是它的聚点集合,如果$E$也有聚点,则这聚点也必定是$A$的聚点.换句话说,集合$E$具有特点:它的聚点必定也在自身中.
\end{theorem}

\begin{proof}[证明]
 只证明有限的情况,设$x$是聚点$E$的一个有限聚点, 则它的任意小空心邻域内必然包含了无穷多个$E$中的点,而根据聚点定义,显然这个空心邻域也就必然包含了$A$中的无穷多个点,因此它也是$A$的聚点.
\end{proof}

聚点集的这种特性称为是闭的,我们定义
\begin{definition}
  如果一个无穷数集的所有聚点也在这集合中,则称这集合是\emph{闭集}.
\end{definition}

与闭集相对,我们还有开集的概念,为此先引入\emph{内点}和\emph{界点}的定义.
\begin{definition}
  设$A$是一个数集,如果$x \in A$并且存在$x$的某个邻域$(x-\delta,x+\delta)$整个都被$A$所包含,则称为$x$是$A$的一个\emph{内点}.
\end{definition}

\begin{definition}
  设$A$是一个数集,$x$是一个实数,如果$x$的任意小的空心邻域内都同时包含有$A$中的点和不在$A$中的点,则称$x$是$A$的一个\emph{界点}.
\end{definition}

下面定义开集:
\begin{definition}
  如果一个数集中的任一个点都是这数集的内点,则称这数集是\emph{开集}.
\end{definition}

注意闭集与开集并不互斥,比如实数集$\mathbb{R}$,它既是开集又是闭集.

\subsection{上极限与下极限}
\label{sec:upper-limit-and-lower-limit}

这一节利用聚点的概念来定义上极限与下极限,并在此基础上讨论极限存在的条件。

首先我们证明一个结论,前面已经知道,有界数集存在聚点,在此基础上还有如下结论
\begin{theorem}
  有界数集存在最大的聚点和最小的聚点。
\end{theorem}

\begin{proof}[证明]
  记有界数集$A$的聚点集合是$B$,设$A$的上下确界分别是$M$和$L$,显然任何大于$M$或者小于$L$的数都不可能是$A$的聚点,也就是说,$B$也以$M$和$L$为上界和下界(但不能保证是确界),记$B$的上确界为$P$,我们来证明,$P$也是$A$的聚点,也就是$P \in B$.

  对于任意正实数$\varepsilon$,存在$b_1 \in B$,使得$P-\varepsilon < b_1 < P$,但$b_1$是$A$的一个聚点,因此它的任意邻域内都含有$A$中的无限个元素,取适当的邻域,使其被包含于开区间$(P-\varepsilon,P)$中,于是开区间$(P-\varepsilon,P)$也就包含了数集$A$中的无限多个元素,从而$P$是数集$A$的聚点,同理,$B$的下确界$Q$也是数集$A$的聚点,它们分别是数集$A$的最大聚点和最小聚点。
\end{proof}

如果考虑到无穷聚点,则上述定理仍然成立。

以后我们会称数列的聚点,它是指由数列所有项的数值所组成的数集(若有数值相同的项,则该值按重复次数计算)的聚点,以后不再特别说明,显然有如下定理。

\begin{theorem}
  数$A$是数列$\{a_n\}$的一个聚点的充分必要条件是,这数列存在收敛到$A$的子数列.
\end{theorem}

\begin{definition}
  数列$a_n$的最大聚点(可以无穷)称为数列的\emph{上极限},记作$\varlimsup\limits_{n \to \infty}a_n$,而最小的聚点称为它的\emph{下极限},记作$\varliminf\limits_{n \to \infty} a_n$.
\end{definition}

显然,数列必同时存在上极限与下极限(可以无穷)。

\begin{theorem}
数列存在极限的充分必要条件是,它的上极限与下极限相等。  
\end{theorem}

\begin{proof}[证明]
  必要性,如果数列$a_n$收敛到$A$,显然$A$是它的唯一聚点,上极限与下极限都是它。

  充分性,设上极限与下极限都是$A$,显然$A$的任意邻域内含有数列中的无限多项,而在此邻域外只能含有数列中的有限多项,否则在此邻域外还有别的聚点存在,所以数列只能收敛到此值。
\end{proof}



\subsection{数$\mathrm{e}$}
\label{sec:a-import-sequence-limit}

这一小节我们来证明下面这个数列有极限:
\[ x_n=\left( 1+\frac{1}{n} \right)^n \]

\begin{proof}[证明一]\footnote{这个证明来自参考文献\cite{olympic-math}.}
  由多元均值不等式,把$x_n$看成$n$个$(1+1/n)$的乘积,再添加上一个因数1构成$n+1$个数的乘积,有
  \[ \left( 1+\frac{1}{n} \right)^n = 1 \cdot \left( 1+\frac{1}{n} \right)^n < \left( \frac{1+n\left( 1+\frac{1}{n} \right)}{n+1} \right)^{n+1} = \left( 1+\frac{1}{n+1} \right)^{n+1} \]
  这便表明它是递增的。

  下证它是有上界的,把$n+1$拆分成$\frac{5}{6}$和$n$个$1+\frac{1}{6n}$,由均值不等式得
  \[ n+1 = \frac{5}{6} + n \left( 1+\frac{1}{6n} \right) > (n+1)\sqrt[n+1]{\frac{5}{6} \cdot \left( 1+\frac{1}{6n} \right)^n} \]
  整理即得
  \[ \left( 1+\frac{1}{6n} \right)^n < \frac{6}{5} \]
  所以
  \[ \left( 1+\frac{1}{6n} \right)^{6n} < \left( \frac{6}{5} \right)^6 < 3 \]
  于是由单调性便知
  \[ \left( 1+\frac{1}{n} \right)^n < 3 \]
  所以数列$x_n$单调增加且有上界,故此存在极限。
\end{proof}

\begin{proof}[证明二]\footnote{这个证明来自于参考文献\cite{math-analysis}.}
  把$x_n$按二项式定理展开得
  \begin{eqnarray*}
    x_n & = & 1+\sum_{i=1}^n C_n^i \frac{1}{n^i} \\
    & = & 1+ \sum_{i=1}^n \frac{1}{i!}\left( 1-\frac{1}{n} \right) \left( 1-\frac{2}{n} \right)\cdots \left( 1-\frac{i-1}{n} \right)
  \end{eqnarray*}
  易见对于$x_{n+1}$而言,在上式的基础上会多出$i=n+1$的一个正项,并且其它项是把上式中每一个项中的每一个因子$1-\frac{i}{n}$更换为更大的因子$1-\frac{i}{n+1}$,所以$x_{n+1}>x_n$,这是一个递增的数列.

  将每一个项中的所有$(1-i/n)$因子全部放大为1,则有
  \[  x_n < 1+\frac{1}{1!}+\frac{1}{2!}+\cdots+\frac{1}{n!} \]
  接下来有两种放缩方式都可以证明它有上限:
  \[ \frac{1}{k!} < \frac{1}{k(k-1)} = \frac{1}{k-1} - \frac{1}{k} \]
  和
  \[ \frac{1}{k!} < \frac{1}{2^{k-1}} \]
  于是
  \[ x_n < 2 + \sum_{i=2}^n \left( \frac{1}{i-1}-\frac{1}{i} \right) = 3-\frac{1}{n} < 3 \]
  或者
  \[ x_n < 2 + \sum_{i=2}^n \frac{1}{2^{i-1}} = 3-\frac{1}{2^{n-1}} < 3 \]
  所以数列单调递增有上界,故此有极限.
\end{proof}

\begin{proof}[证明三]\footnote{这个证明来自参考文献\cite{math-analysis}.}
  设实数$a>b>0$,有
  \[ a^{n+1}-b^{n+1}=(a-b)(a^n+a^{n-1}b+\cdots+ab^{n-1}+b^n) < (n+1)a^n(a-b) \]
  即
  \begin{equation}
    \label{eq:example-equation-a-n-b-n}
   b^{n+1}>a^n(a-(n+1)(a-b))=a^n((n+1)b-na) 
  \end{equation}
  在此式中取$a=1+\dfrac{1}{n}$, $b=1+\dfrac{1}{n+1}$,由于$(n+1)b-na=1$,故得
  \[ \left( 1+\frac{1}{n+1} \right)^{n+1} < \left( 1+\frac{1}{n} \right)^n \]
  即数列$x_n$单调增加,再在前式中取$a=1+\dfrac{1}{2n}$,$b=1$,则$(n+1)b-na=\dfrac{1}{2}$,则得
  \[ \frac{1}{2} \left( 1 + \frac{1}{2n} \right)^n < 1 \]
  即
  \[ \left( 1 + \frac{1}{2n} \right)^{2n} < 4 \]
  因为$x_n$是递增的,又对于任意正整数$n$,都存在大于它的偶数,所以得到
  \[ \left( 1 + \frac{1}{n} \right)^n < 4 \]
  故数列有界,从而存在极限。
\end{proof}

我们用字母$\mathrm{e}$来表示这个数列的极限,它的值是
\[ \lim_{n \to \infty} \left( 1+\frac{1}{n} \right)^n = \mathrm{e}= 2.718281828459045\cdots \]

\begin{example}
  这个极限值$\mathrm{e}$是一个非常重要的常数,对于数列$x_n=\left( 1+\frac{1}{n} \right)^n$,我们有如下一个解释,假如银行一年期定期存款的年利率是r(通常是3\%左右),那么如果将本金1万元存入银行,一年后所得收入是$1+r$万元,而如果我们每个月按月利率$\frac{r}{12}$计算一次月利息,并按复利方式(即把上一期的本金和利息一起计入下一期的本金)计算,那么一年后收入将是$\left( 1+\frac{r}{12} \right)^{12}$万元,由伯努利不等式可知,这个值比一年算一次利息的收入大了些,于是我们猜想,如果我们不嫌麻烦,按日利率$\frac{r}{365}$每天计算一次利息,那么一年后的收入则是$\left( 1+\frac{r}{365} \right)^{365}$,这个值又比之前按月计息的收入高了些(数列$x_n$的单调性),进一步想象,如果我们继续减小计息周期,每分钟一次、每秒一次,甚至每毫秒计息一次,我们的收入会不会变得无穷大呢?这不是一本万利的事情吗,然而
  \[ \lim_{n \to \infty} \left( 1+\frac{r}{n} \right)^n = \left( \lim_{n \to \infty} \left( 1+\frac{r}{n} \right)^{\frac{n}{r}} \right)^r = \mathrm{e}^r \]
  这表明,在无限缩小计息周期的时候,总收入是有一个上确界的,会无限的逼近$\mathrm{e}^r$但永远不会达到它,这一本万利的好事只能是个梦罢了。在这个意义上,也可以认为,常数$\mathrm{e}$是个自然界中客观存在的常量,它还是自然对数的底数,之所以选择它作为对数的底,是因为涉及对数的很多公式,在选择它作底的时候,往往能够使公式具有最简单的形式。
\end{example}

在前述证明二中我们已经证得
\[ x_n < 1 + \frac{1}{1!} + \frac{1}{2!} + \cdots + \frac{1}{n!} = y_n \]
并且证明二就是利用了$y_n$有上界而得出$x_n$有上界的,显然$y_n$也是单调增加的,所以$y_n$也有极限,其实它也收敛到$\mathrm{e}$,在此证明这一点。

在证明二的过程中已经得到
\[ x_n = 1 + \sum_{i=1}^n \frac{1}{i!}\left( 1-\frac{1}{n} \right) \left( 1-\frac{2}{n} \right) \cdots \left( 1-\frac{i-1}{n} \right) \]
任选整数$k$满足$0<k<n$,并在上式右端的求和中舍弃$i>k$的加项,得(注意求和的上限)
\begin{equation}\label{eq:e-xn-yk}
 x_n > 1 + \sum_{i=1}^k \frac{1}{i!}\left( 1-\frac{1}{n} \right) \left( 1-\frac{2}{n} \right) \cdots \left( 1-\frac{i-1}{n} \right) 
\end{equation}
固定整数$k$不变,则右端是有限个数之和,让$n$无限增大两端取极限,得
\[ \mathrm{e} \geqslant 1 + \frac{1}{1!} + \frac{1}{2!} + \cdots + \frac{1}{k!} = y_k \]
于是得出
\[ x_n < y_n \leqslant \mathrm{e} \]
由夹逼准则,就得出
\[ \lim_{n \to \infty} \left( 1 + \frac{1}{1!} + \frac{1}{2!} + \cdots + \frac{1}{n!} \right) = \mathrm{e} \]
或者说,有下面的级数
\begin{equation}
  \label{eq:series-1-devide-by-n-fractor-is-e}
 \sum_{n=0}^{\infty} \frac{1}{n!} = \mathrm{e} 
\end{equation}

现在考虑一个问题,我们是由$x_n<y_n \leqslant \mathrm{e}$ 以及$\lim\limits_{n \to \infty} x_n = \mathrm{e}$得出$\lim\limits_{n \to \infty} y_{n}=\mathrm{e}$的,但如果我们根据$y_n$单调增加并有上界从而定义$\mathrm{e}$为它的极限,那又如何说明$x_n$也收敛到$\mathrm{e}$呢,此时由$x_n<y_n\leqslant \mathrm{e}$以及$y_n$收敛到$\mathrm{e}$显然是无法得出$x_n$的极限的。

这点也不难,此时在\autoref{eq:e-xn-yk}中固定$k$并令$n\to\infty$有$\lim\limits_{n \to \infty} x_n > y_k$,又$x_n<y_n$,故得
\[ x_n < y_n < \lim_{n \to \infty} x_n \]
由于$x_n$与$y_n$都收敛,对上式取极限就得
\[ \lim_{n \to \infty} x_n \leqslant \lim_{n \to \infty}y_n \leqslant \lim_{n \to \infty} x_n \]
于是就有
\[ \lim_{n \to \infty} x_n = \lim_{n \to \infty} y_n = \mathrm{e} \]

现在再考虑另一个数列
\[ z_n = \left( 1 + \frac{1}{n} \right)^{n+1} \]
显然它有下界,我们再证明它是单调递减的,由伯努利不等式(\autoref{theorem:bernoulli-inequality})有
\[ \left( \frac{n^2}{n^2-1} \right)^n = \left( 1 + \frac{1}{n^2-1} \right)^n > 1 + \frac{n}{n^2-1} > 1 + \frac{1}{n} \]
整理即得
\[ \left( 1 + \frac{1}{n-1} \right)^n > \left( 1+ \frac{1}{n} \right)^{n+1} \]
即$z_n>z_{n+1}$,于是它单调减少,从而收敛,又根据
\[ z_n=\left( 1+\frac{1}{n} \right)x_n \]
知它与$x_n$有共同的极限
\[ \lim_{n \to \infty} \left( 1 + \frac{1}{n} \right)^{n+1} = \mathrm{e} \]
事实上,$x_n$与$z_n$构成一个闭区间套序列,而由这区间套确定的实数就是$\mathrm{e}$。

关于$x_n$趋于极限的速度,我们有
\[ 0 < e-\left( 1 + \frac{1}{n} \right)^n < \frac{3}{n} \]
\begin{proof}[证明一]
  根据前面的结论,数列
  \[ \left( 1+\frac{1}{n} \right)^{n+1} \]
  单调减少并收敛到$\mathrm{e}$,因此
  \[ 0 < \mathrm{e} - \left( 1+\frac{1}{n} \right)^n < \left( 1+\frac{1}{n} \right)^{n+1} - \left( 1+\frac{1}{n} \right)^{n} = \frac{1}{n} \left( 1+\frac{1}{n} \right)^{n} < \frac{3}{n} \]
  即得结论.
\end{proof}

\begin{proof}[证明二]
  我们只要证明数列
  \[ \left( 1 + \frac{1}{n} \right)^n + \frac{3}{n} \]
  单调减少并收敛到$\mathrm{e}$就可以了。

  为了证明单调减少,只要证明
  \[ \left( 1 + \frac{1}{n} \right)^n + \frac{3}{n} > \left( 1 + \frac{1}{n+1} \right)^{n+1} + \frac{3}{n+1} \]
  即要证
  \[ \left( 1 + \frac{1}{n+1} \right)^{n+1} - \left( 1 + \frac{1}{n} \right)^n < \frac{3}{n(n+1)} \]
  设$a>b>0$,则
  \[ a^n-b^n=(a-b)(a^{n-1}+a^{n-2}b+\cdots+ab^{n-2}+b^{n-1})>nb^{n-1}(a-b) \]
  取$a=1+\dfrac{1}{n}$,$b=1+\dfrac{1}{n+1}$,有
  \[ \left( 1 + \frac{1}{n} \right)^{n} - \left( 1 + \frac{1}{n+1} \right)^{n} >\frac{1}{n+1}\left( 1+\frac{1}{n+1} \right)^{n-1} \]
  所以
  \begin{eqnarray*}
    && \left( 1 + \frac{1}{n+1} \right)^{n+1} - \left( 1 + \frac{1}{n} \right)^n \\
    & = & \left( 1 + \frac{1}{n+1} \right)\left( 1 + \frac{1}{n+1} \right)^{n} - \left( 1 + \frac{1}{n} \right)^n \\
    & = & \left( 1 + \frac{1}{n+1} \right)^{n} - \left( 1 + \frac{1}{n} \right)^n + \frac{1}{n+1} \left( 1 + \frac{1}{n+1} \right)^{n} \\
    & < & -\frac{1}{n+1}\left( 1+\frac{1}{n+1} \right)^{n-1} + \frac{1}{n+1} \left( 1 + \frac{1}{n+1} \right)^{n} \\
    & = & \frac{1}{(n+1)^2}\left( 1+\frac{1}{n+1} \right)^{n-1} \\
    & < & \frac{3}{(n+1)^2} < \frac{3}{n(n+1)}
  \end{eqnarray*}
  于是就证明了它是单调减少的,显然它有下界(比如零是它的一个下界),所以它有极限,而
  \[ \lim_{n \to \infty} \left( \left(1+\frac{1}{n} \right)^n + \frac{3}{n} \right) = \lim_{n \to \infty}\left( 1+\frac{1}{n} \right)^n + \lim_{n \to \infty} \frac{3}{n} = \mathrm{e} + 0 = \mathrm{e} \]
    于是便得结论
\[ 0 < e-\left( 1 + \frac{1}{n} \right)^n < \frac{3}{n} \]
\end{proof}

利用\autoref{eq:series-1-devide-by-n-fractor-is-e},我们可以用不多的几项就得出$\mathrm{e}$精确度较高的近似值,这是因为阶乘的增长速度相当快,比指数还快,因此这个级数收敛到它的和的速度也是相当快。为此考虑用它的第$n$个部分和去近似$\mathrm{e}$时的误差限,我们有
\begin{eqnarray*}
  y_{n+m}-y_n & = & \frac{1}{(n+1)!}+\frac{1}{(n+2)!}+\cdots+\frac{1}{(n+m)!} \\
              & = & \frac{1}{n!} \left( \frac{1}{n+1}+\frac{1}{(n+1)(n+2)}+\cdots+\frac{1}{(n+1)(n+2)\cdots(n+m)} \right) \\
              & < & \frac{1}{n!} \left( \frac{1}{n+1}+\frac{1}{(n+1)^2}+\cdots+\frac{1}{(n+1)^m} \right) \\
              & = & \frac{1}{n!} \cdot \frac{1}{n} \left(1-\frac{1}{(n+1)^m} \right)
\end{eqnarray*}
固定$n$,令$m \to \infty$,就得到(下式本来应该是小于等于,但只要在上面取更强的放缩,比如是提$\dfrac{1}{(n+1)!}$而不是$\dfrac{1}{n!}$,取极限之后就可以得到一个小于等于$\dfrac{1}{nn!}$的量)
\[ \mathrm{e}-y_n < \frac{1}{nn!} \]
于是对于任意正整数$n$,存在实数$\theta_n \in (0,1)$,使得
\begin{equation}
  \label{eq:e-n-fractor}
 \mathrm{e} = 1+\frac{1}{1!}+\frac{1}{2!}+\cdots+\frac{1}{n!}+\frac{\theta_n}{nn!} 
\end{equation}

利用\autoref{eq:e-n-fractor},我们可以证明$\mathrm{e}$是无理数,因为如果它是有理数,则有
 \[ \frac{m}{n} = 1+\frac{1}{1!}+\frac{1}{2!}+\cdots+\frac{1}{n!}+\frac{\theta_n}{nn!} \]
注意这里是先用有理数$\dfrac{p}{q}$表示$\mathrm{e}$,再将右端展开到对应的$n=q$的,然后只要在此式的两端同时乘以$n!$,左端得出一个整数,但是右端却是一个整数加上一个真分数$\dfrac{\theta_n}{n}$,矛盾,所以$\mathrm{e}$是无理数,如此也可以说明\autoref{eq:e-n-fractor}中的$\theta_n$不可能取1.

现在利用\autoref{eq:e-n-fractor}估算一下,只要取$n=20$,就有
\[ \frac{1}{20\times 20!}=\frac{1}{48658040163532800000}= 2.0551\ldots \times 10^{-20} < 3 \times 10^{-20} \]
再考虑前面前面20项的舍入误差,每一项都保留20位小数,此时的舍入误差$<20\times 10^{-20}$,因此用前20项的和去逼近$\mathrm{e}$的误差不超过$23 \times 10^{-20}$,因此至少具有17位精确小数。

\subsection{欧拉常数$C$}
\label{sec:euler-constant}

我们已经证明了
\[ \left( 1+\frac{1}{n} \right)^n < \mathrm{e} < \left( 1+\frac{1}{n} \right)^{n+1} \]
对上式取以$\mathrm{e}$为底数的对数得
\[ n \ln{\left( 1+\frac{1}{n} \right)} < 1 < (n+1)\ln{\left( 1+\frac{1}{n} \right)} \]
即
\[ \frac{1}{n+1} < \ln{\left( 1+\frac{1}{n} \right)} < \frac{1}{n} \]
而
\[ \ln{\left( 1+\frac{1}{n} \right)} = \ln{(n+1)}-\ln{n} \]
所以
\begin{equation}
  \label{eq:1-devided-by-n-ln-n-approximation}
 \ln{(n+1)}-\ln{n} < \frac{1}{n} < \ln{n}-\ln{(n-1)} 
\end{equation}
这是关于对数与自然数倒数的一个重要不等式,以后还会用到它,在式中取$n=1,2,\ldots$进行累加(右端要从2开始),可得
\[ \ln{(n+1)} < 1 + \frac{1}{2} + \cdots + \frac{1}{n} < 1+\ln{n} \]
于是得
\[ \ln{(n+1)}-\ln{n} < 1 + \frac{1}{2} + \cdots + \frac{1}{n} - \ln{n} < 1 \]
这表明序列
\[ x_n = 1 + \frac{1}{2} + \cdots + \frac{1}{n} - \ln{n} \]
是有上界的,而作差便知它是单调增加的,因此它收敛,这个结论属于欧拉,所以它的极限被称为\emph{欧拉常数},其值是
\[ C=0.577216\ldots \]
利用欧拉常数,调和级数可以写成
\[ H_n = 1 + \frac{1}{2} + \cdots + \frac{1}{n} = \ln{n} + C + \gamma_n \]
这里$\gamma_n$是一个无穷小。

我们早已证明过,在$n \to \infty$时,
\[ 1 + \frac{1}{2} + \cdots + \frac{1}{n} \]
与$\ln{n}$是等价无穷大,而这里的结果显示,不仅如此,这两者的差值将趋于常数,这表明用对数来逼近调和级数时的误差始终是一个常数,而相对误差则渐趋于零。

\begin{example}
  \label{example:limit-of-sum-of-i/n+1-i-in-(1-n)}
  利用欧拉常数,可以求下式的极限
  \[ \frac{1}{n+1}+\frac{1}{n+2}+\cdots+\frac{1}{2n} \]
  因为
  \[ H_n = \ln{n} + C + \gamma_n, \  H_{2n}=\ln{2n}+C+\gamma_{2n} \]
  所以
  \[ \frac{1}{n+1}+\frac{1}{n+2}+\cdots+\frac{1}{2n} = \ln{2n}-\ln{n}+\gamma_{2n}-\gamma_n \]
  两边取极限便得
  \[ \lim_{n \to \infty} \left( \frac{1}{n+1}+\frac{1}{n+2}+\cdots+\frac{1}{2n} \right) = \ln{2} \]

  利用\autoref{eq:1-devided-by-n-ln-n-approximation}也可以证明这个极限,因为可以得到
  \[ \ln{(2n+1)}-\ln{(n+1)} < \frac{1}{n+1}+\frac{1}{n+2}+\cdots+\frac{1}{2n} < \ln{2n}-\ln{n} \]
  即
  \[ \ln{\frac{2n+1}{n+1}} < \frac{1}{n+1}+\frac{1}{n+2}+\cdots+\frac{1}{2n} < \ln{2} \]
  对不等式取极限便得结论。
\end{example}

\subsection{圆的面积之讨论}
\label{sec:pi}

考虑圆的面积,以$r$表其半径,根据平面图形的坐标伸缩变换,可知其面积$P$与以其半径为边长的正方形面积$r^2$成正比,即存在一个固定的正实数$\tau$,使得所有圆的面积都满足
\[ P=\tau r^2 \]
暂时我们还不能将它与圆周率$\pi$等同起来.

用内接和外切正多边形对圆进行内填外包,其内接正$n$边形的面积是
\[ S_n=\frac{1}{2}r^2\cdot n \sin{\frac{2\pi}{n}} \]
外切正$n$边形的面积是
\[ T_n = r^2 \cdot n \tan{\frac{\pi}{n}} \]
显然有
\[ S_n<P<T_n \]
另一方面容易知道
\[ S_{2n}>S_{n}, T_{2n}<T_n \]
于是数列 $S_{2^n}$ 严格递增有上界而 $T_{2^n}}$ 严格递减有下界,故两者都有极限,分别记为 $S$ 和 $T$,即
\[ S=\lim_{n\to\infty}S_{2^n}, T=\lim_{n\to\infty}T_{2^n} \]
如果能证明 $S=T$,那么我们就有理由认为这共同的极限就是圆的面积.

内接正$n$边形和外切正$n$边形的面积相差部分被夹在一个圆环中,这个圆环的内外两个半径分别是$r\cos{\frac{\pi}{n}}$和$\frac{r}{\cos{\frac{\pi}{n}}}$,于是有
\[ T_n-S_n<\tau r^2 \left( \frac{1}{\cos^2 \frac{\pi}{n}} - \cos^2 \frac{\pi}{n} \right) \]
于是
\[ T_{2^n}-S_{2^n}<\tau r^2 \left( \frac{1}{\cos^2 \frac{\pi}{2^n}} - \cos^2 \frac{\pi}{2^n} \right) \]
在 \autoref{limit-of-cos-pi-frac-n} 中已经得到 $ \lim_{n\to\infty}\cos{\frac{\pi}{2^n}} = 1 $ ,因此有
\[ \lim_{n\to\infty} (T_{2^n}-S_{2^n}) = 0 \]
这就表明 $S=T$,于是
\[ \lim_{n\to\infty}S_{2^n} = \tau r^2 = \lim_{n\to\infty}T_{2^n} \]
或者
\[ \lim_{n\to\infty} 2^n \sin{\frac{\pi}{2^n}} = \tau = \lim_{n\to\infty} 2^n \tan{\frac{\pi}{2^n}} \]
以后我们将证明 $\tau$ 实际上就等于圆周率 $\pi$.

\subsection{无穷级数}
\label{sec:infinite-series}

在上一小节我们已经看到下面两个极限是存在的的
\[ \lim_{n\to\infty} \left( 1+\frac{1}{2^2}+\cdots+\frac{1}{n^2} \right) \]
与
\[ \lim_{n\to\infty} \left( 1 + \frac{1}{2!}+\cdots+\frac{1}{n!} \right) \]
这很自然的引出了无限个实数相加的和的概念。

一般地,对于一个无限数列 $a_1,a_2,\ldots,a_n,\ldots$,,作出它的所有项的无限和式
\[ a_1+a_2+\cdots+a_n+\cdots \]
为节约篇幅,以后我们都简记为
\[ \sum_{n=1}^{\infty}a_n \]
称为一个 \emph{无穷级数},或者简称为\emph{级数},即无穷级数是一个无限长的加式。很自然的我们有如下定义
\begin{definition}[无穷级数收敛性]
  对无穷级数 $\sum_{n=1}^{\infty}a_n$,称
  \[ \sum_{k=1}^na_k=a_1+a_2+\cdots+a_n \]
  为它的(前$n$项的)\emph{部分和},若 $\lim_{n\to\infty}\sum_{k=1}^na_k$ 存在,则称无穷级数$\sum_{n=1}^{\infty}a_n$ \emph{收敛},或者说它是一个\emph{收敛级数},而极限$S$称为是这无穷级数的和,即
  \[ \sum_{n=1}^{\infty}a_n = \lim_{n\to\infty}\sum_{k=1}^na_k  \]
  反之,若极限 $\lim_{n\to\infty}\sum_{k=1}^na_k$不存在,则称无穷级数$\sum_{n=1}^{\infty}a_n$ \emph{发散},或者说它是一个\emph{发散级数}.
\end{definition}

无穷级数的意义在于,我们将加法从有限个数相加推广到了无限个数相加。

\begin{example}
  级数
  \begin{equation*}
    \begin{split}
      & \sum_{n=1}^{\infty}\frac{1}{n(n+1)} \\
      = &  \lim_{n\to\infty} \sum_{k=1}^{n}\frac{1}{k(k+1)} \\
    = &  \lim_{n\to\infty} \sum_{k=1}^{n} \left( \frac{1}{k} - \frac{1}{k+1} \right)  \\
    = & \lim_{n\to\infty} \left( 1 - \frac{1}{n+1} \right) \\
    =  & 1
    \end{split}
  \end{equation*}
\end{example}

\begin{example}
  当$|q|<1$时,级数(注意下标从0开始)
  \begin{equation*}
    \begin{split}
      &  \sum_{n=0}^{\infty}q^n \\
      = & \lim_{n\to\infty} \sum_{k=0}^{n}q^k \\
      = & \lim_{n\to\infty} \frac{1-q^{n+1}}{1-q} \\
      = & \frac{1}{1-q}
    \end{split}
  \end{equation*}
  这级数称为\emph{几何级数}.

  如果下标从1开始,那么由于少了第一项,就有
  \[ \sum_{n=1}^{\infty}q^n = \frac{1}{1-q}-1 = \frac{q}{1-q} \]

  取 $q=\frac{1}{2}$,即得
  \[ \frac{1}{2} + \frac{1}{2^2} + \frac{1}{2^3} + \cdots = 1 \]
  取 $q=\frac{1}{3}$,得
  \[ \frac{1}{3} + \frac{1}{3^2} + \frac{1}{3^3} + \cdots = \frac{1}{2} \]
\end{example}

  \begin{example}
    实际上任何一个极限$\lim_{n\to\infty}a_n$都可以表示成为级数,由以下恒等式
    \[ a_n = a_1+(a_2-a_1)+(a_3-a_2)+\cdots+(a_n-a_{n-1}) \]
   即知 
    \[ \lim_{n\to\infty}a_n = a_1+\lim_{n\to\infty}\sum_{k=1}^{n-1}(a_{k+1}-a_{k}) = a_1+\sum_{n=1}^{\infty}(a_{n+1}-a_n) \]
  \end{example}


\begin{example}[调和级数的发散性]
  \label{example:non-convergency-of-harmonic-series-proof-by-monotone}
  在这个例子中,我们来考虑\emph{调和级数}
  \[ \sum_{n=1}^{\infty} \frac{1}{n} = 1+\frac{1}{2} + \frac{1}{3} + \cdots  \]
  的收敛性,它的部分和
  \[ H_n = 1 + \frac{1}{2} + \cdots + \frac{1}{n} \]
  虽然$H_n$随着$n$的增大,它的增量逐渐减小,但可以证明它可以大于任意正实数,因为我们有
  \begin{eqnarray*}
    \frac{1}{3} + \frac{1}{4} & > & 2 \times \frac{1}{4} = \frac{1}{2} \\
    \frac{1}{5} + \frac{1}{6} + \frac{1}{7} + \frac{1}{8} & > & 4 \times \frac{1}{8} = \frac{1}{2} \\
    ...... && \\
    \frac{1}{2^m+1} + \frac{1}{2^m+2} + \cdots \frac{1}{2^{m+1}} & > & 2^m \times \frac{1}{2^{m+1}} = \frac{1}{2}
  \end{eqnarray*}
  因此
  \begin{eqnarray*}
    H_{2^m} & = & 1+ \frac{1}{2} + \sum_{i=1}^m \left( \frac{1}{2^i+1} + \frac{1}{2^i+2} + \cdots \frac{1}{2^{i+1}} \right) \\
    & > & 1 + \frac{1}{2} + \frac{1}{2}m
  \end{eqnarray*}
  这就表明$H_{2^m}$是一个无穷大,再由$H_n$的单调性,可知它是一个正无穷大,也就是说调和级数是发散的。
\end{example}

依上一小节的结论,以下两个级数是收敛的
\[ \sum_{n=1}^{\infty}\frac{1}{n^2}\]
\[ \sum_{n=1}^{\infty}\frac{1}{n!} \]
对于两者的和,现在先给出结论
\[ \sum_{n=1}^{\infty}\frac{1}{n^2} = \frac{\pi^2}{6}\]
\[ \sum_{n=1}^{\infty}\frac{1}{n!} = \mathrm{e} \]
第一个和的推导将在以后给出,第二个在前文已经得出.

由数列的单调有界定理可得
\begin{theorem}
  如果实数项级数$\sum_{n=1}^{\infty}a_n$的项都是正的(或者从某一个下标开始以后的项都是正的),且部分和$\sum_{k=1}^na_k$有上界,则该级数收敛.
\end{theorem}

因为级数收敛等同于其部分和数列的收敛,因此级数也有如下的柯西收敛准则
\begin{theorem}
  级数$\sum_{n=1}^{\infty}a_n$收敛的充分必要条件是,对于无论多么小的正实数$\varepsilon$,都存在某个下标$N$,使得级数在这个下标以后的任意片段之和的绝对值都小于$\varepsilon$,即任意的$n>N$及$m\in \mathbb{N^{*}}$,都有
  \[ |a_{n}+a_{n+1}+\cdots+a_{n+m}|<\varepsilon \]
\end{theorem}

\begin{example}
  调和级数的发散性
\end{example}

由柯西收敛准则可得
\begin{theorem}
 级数$\sum_{n=1}^{\infty}a_n$收敛的必要条件是$\lim_{n\to\infty}a_n=0$ 
\end{theorem}

现提出如下定义
\begin{definition}
  如果级数$\sum_{n=1}^{\infty}|a_n|$收敛,则称级数$\sum_{n=1}^{\infty}a_n$ \emph{绝对收敛}.
\end{definition}

在这定义下,立即可得
\begin{theorem}
  绝对收敛的级数必然收敛.
\end{theorem}
证明很简单,利用下面的不等式再结合柯西收敛准则即可得出
\[ |a_{n}+a_{n+1}+\cdots+a_{n+m}| < |a_{n+1}|+|a_{n+2}|+\cdots+|a_{n+m}| \]


关于每一项都是正值的级数的收敛性,有以下重要的比较判别法
\begin{theorem}[比较判别法]
  对于两个正项实数级数$\sum_{n=1}^{\infty}a_n$和$\sum_{n=1}^{\infty}b_n$,如果对每个正整数$n$都成立$a_n \leqslant b_n$(或者从某一项开始后续所有项都满足),那么如果$\sum_{n=1}^{\infty}b_n$收敛则$\sum_{n=1}^{\infty}a_n$必然也收敛,如果$\sum_{n=1}^{\infty}a_n$发散则$\sum_{n=1}^{\infty}b_n$必然也发散.
\end{theorem}
由柯西收敛准则是容易证明的,这里从略.

\begin{example}
  讨论级数
  \[ \sum_{n=1}^{\infty}\frac{1}{n^{\alpha}} \]
  的收敛性,这里$\alpha>0$.

  我们已经知道,当$\alpha=1$时级数发散,当$\alpha=2$时级数收敛,因此按照比较判别法,$0<\alpha<1$时级数都发散,而当$\alpha>2$时级数都收敛,唯有$1<\alpha<2$的情况还不明确,然而在这种情况下仿照前文对调和级数的处理方法,我们有
  \begin{align*}
    \frac{1}{3^{\alpha}+\frac{1}{4^{\alpha}}} & < \frac{2}{2^{\alpha}} \\
    \frac{1}{5^{\alpha}}+\frac{1}{6^{\alpha}}+\frac{1}{7^{\alpha}}+\frac{1}{8^{\alpha}} & < \frac{4}{4^{\alpha}} \\
     ... & \\
    \frac{1}{(2^n+1)^{\alpha}}+\cdots+\frac{1}{(2^{n+1})^{\alpha}} & < \frac{2^n}{(2^n)^{\alpha}}
  \end{align*}
  但是级数
  \[ \sum_{n=1}^{\infty} \frac{2^n}{(2^n)^{\alpha}} = \sum_{n=1}^{\infty} \frac{1}{(2^{\alpha-1})^n} \]
  是收敛的,因此级数
  \[ \sum_{n=1}^{\infty} \left( \frac{1}{(2^n+1)^{\alpha}}+\cdots+\frac{1}{(2^{n+1})^{\alpha}} \right) \]
  收敛,也即
  \[ \sum_{n=1}^{\infty}\frac{1}{n^{\alpha}} \]
  在$1<\alpha<2$时也收敛.

  最终结论就是,在$0<\alpha<1$时级数发散,当$\alpha>1$时级数收敛.
\end{example}




\subsection{复数数列的极限与复级数}
\label{sec:limit-of-complex-number-sequence}

与实数数列相仿,我们可以得到复数数列的极限定义
\begin{definition}
  设有无限复数数列$z_1,z_2,\ldots$,如果存在复数$Z$,使得对于无论多么小的正实数$\varepsilon$,都能从某一个下标$N$开始,后续所有的项都满足 $|z_n-Z|<\varepsilon$,就称这复数数列当$n\to\infty$时存在极限为$Z$,或者称该数列 \emph{收敛}到$Z$,记为
  \[ \lim_{n\to\infty}z_n = Z \]
\end{definition}

由于复数与复平面上的点一一对应,因此复数的收敛可以与二维平面上点列的收敛联系起来.

复数数列的实数和虚部各组成一个实数数列,记$z_n=x_n+iy_n$,$Z=x+iy$, 注意到
\[ |z_n-z|=\sqrt{(x_n-x)^2+(y_n-y)^2} \]
因此有 $|x_n-x|<|z_n-Z|$及$|y_n-y|<|z_n-Z|$,所以如果 $z_n$ 收敛到$Z$,则它的实部和虚部分别收敛到$Z$的实数和虚部.

反过来,如果复数数列$z_n$的实部$x_n$和虚部$y_n$作为两个实数数列分别收敛到$x$和$y$,记$Z=x+iy$,那么对于任意小的正实数$\varepsilon$,能找到一个共同的下标$N$,使得后续所有的$x_n$和$y_n$都满足$|x_n-x|<\varepsilon$和$|y_n-y|<\varepsilon$,于是有
\[ |z_n-Z|=\sqrt{(x_n-x)^2+(y_n-y)^2}<\sqrt{2}\varepsilon \]
这就表明$z_n$收敛到$Z=x+iy$,于是得到
\begin{theorem}
  复数数列$z_n=x_n+iy_n$收敛到复数$Z=x+iy$的充分必要条件是它的实部$x_n$和虚部$y_n$两个实数数列分别收敛到$x$和$y$.
\end{theorem}

\begin{example}
  复数数列$z_n=z^n(z\in \mathbb{C}, n\in \mathbb{N})$当$|z|<1$时收敛到0,如果从复平面上加以考察,则$z^n$的模长按等比数列减小,而辐角则以等差数列增大,其所代表的点列分布于一列半径逐次减小并以原点为圆心的同心圆上,但点在圆上则按旋转的方式分布,并收敛到原点.
\end{example}

复数数列收敛也有柯西收敛法则
\begin{theorem}
  复数数列$z_n$收敛的充分必要条件是,对于任意小的正实数$\varepsilon$,都能从某一个下标$N$开始,后续任意两项$z_m$、$z_n$ 都成立 $ |z_n-z_m| < \varepsilon $
\end{theorem}
从复平面上理解,点列收敛,则能从某一个点开始,后续任意两点之间的距离可以任意小.


和实数数列相仿,容易证明,收敛的复数数列必然有界,且满足极限的和差积商的运算法则。但由于复数不能比较大小,保号性与单调有界相关的结论并不能照搬到复数数列上来.

闭区间套定理也可以推广到复数,不过需要把闭区间换成闭矩形套或者闭圆套,对于闭矩形套的情形,只要分别对实部和虚部应用实数数列的闭区间套定理就可以得到证明. 对于闭圆套的情形,这些圆的内接正方形就是一个闭矩形套,这样也就得到了证明.

与实数级数的定义类似有复级数概念
\begin{definition}
如果复数数列$z_n$的前$n$项$z_1+z_2+\cdots+z_n$和当$n\to\infty$时收敛到$Z$,就称无穷级数$\sum_{n=1}^{\infty}z_n=z_1+z_2+\cdots+$收敛到$Z$,记为
\[ \sum_{n=1}^{\infty}z_n=Z \]
\end{definition}

\begin{example}
  由恒等式
  \[ 1+z+z^2+\cdots+z^n = \frac{1-z^{n+1}}{1-z} \]
  可知,当$|z|<1$时,有
  \[ \sum_{n=0}^{\infty}z^n = \frac{1}{1-z} \]
\end{example}

复级数收敛的柯西收敛准则如下
\begin{theorem}
  复级数$\sum_{n=1}^{\infty}z_n$收敛的充分条件条件是,对任意小的正实数$\varepsilon$,存在某个下标$N$,使得该下标之后的任意长的片段$z_n+z_{n+1}+\cdots+z_{n+m}(n>N, m \in \mathbb{N})$都满足
  \[ |z_n+z_{n+1}+\cdots+z_{n+m}| < \varepsilon \]
\end{theorem}

%%% Local Variables:
%%% mode: latex
%%% TeX-master: "../calculus-note"
%%% End:
